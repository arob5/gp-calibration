\documentclass[12pt]{article}
\RequirePackage[l2tabu, orthodox]{nag}
\usepackage[main=english]{babel}
\usepackage[rm={lining,tabular},sf={lining,tabular},tt={lining,tabular,monowidth}]{cfr-lm}
\usepackage{amsthm,amssymb,latexsym,gensymb,mathtools,mathrsfs}
\usepackage[T1]{fontenc}
\usepackage[utf8]{inputenc}
\usepackage[pdftex]{graphicx}
\usepackage{epstopdf,enumitem,microtype,dcolumn,booktabs,hyperref,url,fancyhdr}
\usepackage{algorithmic}
\usepackage[ruled,vlined,commentsnumbered,titlenotnumbered]{algorithm2e}

% Plotting
\usepackage{pgfplots}
\usepackage{xinttools} % for the \xintFor***
\usepgfplotslibrary{fillbetween}
\pgfplotsset{compat=1.8}
\usepackage{tikz}

% Custom Commands
\newcommand*{\norm}[1]{\left\lVert#1\right\rVert}
\newcommand*{\abs}[1]{\left\lvert#1\right\rvert}
\newcommand*{\suchthat}{\,\mathrel{\big|}\,}
\newcommand{\E}{\mathbb{E}}
\newcommand{\Var}{\mathrm{Var}}
\newcommand{\R}{\mathbb{R}}
\newcommand{\N}{\mathcal{N}}
\newcommand{\Ker}{\mathrm{Ker}}
\newcommand{\Cov}{\mathrm{Cov}}
\newcommand{\Prob}{\mathbb{P}}
\DeclarePairedDelimiterX\innerp[2]{(}{)}{#1\delimsize\vert\mathopen{}#2}
\DeclareMathOperator*{\argmax}{argmax}
\DeclareMathOperator*{\argmin}{argmin}
\DeclarePairedDelimiter{\ceil}{\lceil}{\rceil}

\setlist{topsep=1ex,parsep=1ex,itemsep=0ex}
\setlist[1]{leftmargin=\parindent}
\setlist[enumerate,1]{label=\arabic*.,ref=\arabic*}
\setlist[enumerate,2]{label=(\alph*),ref=(\alph*)}

% For embedding images
\graphicspath{ {./images/} }

% Specifically for paper formatting 
\renewcommand{\baselinestretch}{1.2} % Spaces manuscript for easy reading

% Formatting definitions, propositions, etc. 
\newtheorem{definition}{Definition}
\newtheorem{prop}{Proposition}
\newtheorem{lemma}{Lemma}
\newtheorem{thm}{Theorem}
\newtheorem{corollary}{Corollary}

% Title and author
\title{Calibration of Terrestrial Carbon Models}
\author{Andrew Roberts}

\begin{document}

\maketitle
\tableofcontents
\newpage

\section{General Setting: Computer Model Calibration}
\section{Motivating Example: Very Simple Ecosystem Model}
\subsection{Relation to the General Setting}

\section{Parameter Calibration Basics}
\subsection{Correlated Outputs}
In the previous section, we assumed independence between model outputs, which resulted in a convenient likelihood. To be clear, we applied the independence assumption
\begin{align}
p(y|\theta, \Sigma_\epsilon) = \prod_{j = 1}^{p} p(y_j|\theta, \sigma_{\epsilon_j}^2)
\end{align}
where $y \in \R^p$ is a single observation of the $p$ outputs. In the case that this assumption is inappropriate, the previous models may lead to inaccurate 
quantification of uncertainties. In this case, we consider an explicit model for the covariance between model outputs. We begin by introducing a simple model 
that generalizes $\Sigma_\epsilon = \text{diag}\left(\sigma_{\epsilon_1}^2, \dots, \sigma_{\epsilon_p}^2\right)$ from a diagonal to an arbitrary, dense covariance 
matrix. Since $\Sigma_\epsilon$ is now parameterized by $\frac{p(p+1)}{2}$ parameters, this model is only computational feasible when the number of outputs is 
quite small. Nevertheless, it is a good starting point to motivate more complicated models. 




\section{Surrogate Modeling}
\section{Multi-Site Hierarchical Models}

\section{Appendix}

\section{TODOs}
\begin{itemize}
\item Contrast setting with KOH setting that has both $X$ and $\theta$ variables. 
\item Standardize my notation in how I'm subscripting vs. superscripting variables to index observations vs. outputs. 
\end{itemize}


\end{document}


