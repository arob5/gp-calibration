\documentclass[12pt]{article}
\RequirePackage[l2tabu, orthodox]{nag}
\usepackage[main=english]{babel}
\usepackage[rm={lining,tabular},sf={lining,tabular},tt={lining,tabular,monowidth}]{cfr-lm}
\usepackage{amsthm,amssymb,latexsym,gensymb,mathtools,mathrsfs}
\usepackage[T1]{fontenc}
\usepackage[utf8]{inputenc}
\usepackage[pdftex]{graphicx}
\usepackage{epstopdf,enumitem,microtype,dcolumn,booktabs,hyperref,url,fancyhdr}
\usepackage{algorithmic}
\usepackage[ruled,vlined,commentsnumbered,titlenotnumbered]{algorithm2e}

% Plotting
\usepackage{pgfplots}
\usepackage{xinttools} % for the \xintFor***
\usepgfplotslibrary{fillbetween}
\pgfplotsset{compat=1.8}
\usepackage{tikz}

% Custom Commands
\newcommand*{\norm}[1]{\left\lVert#1\right\rVert}
\newcommand*{\abs}[1]{\left\lvert#1\right\rvert}
\newcommand*{\suchthat}{\,\mathrel{\big|}\,}
\newcommand{\E}{\mathbb{E}}
\newcommand{\Var}{\mathrm{Var}}
\newcommand{\R}{\mathbb{R}}
\newcommand{\N}{\mathcal{N}}
\newcommand{\Ker}{\mathrm{Ker}}
\newcommand{\Cov}{\mathrm{Cov}}
\newcommand{\Prob}{\mathbb{P}}
\DeclarePairedDelimiterX\innerp[2]{(}{)}{#1\delimsize\vert\mathopen{}#2}
\DeclareMathOperator*{\argmax}{argmax}
\DeclareMathOperator*{\argmin}{argmin}
\DeclarePairedDelimiter{\ceil}{\lceil}{\rceil}

\setlist{topsep=1ex,parsep=1ex,itemsep=0ex}
\setlist[1]{leftmargin=\parindent}
\setlist[enumerate,1]{label=\arabic*.,ref=\arabic*}
\setlist[enumerate,2]{label=(\alph*),ref=(\alph*)}

% For embedding images
\graphicspath{ {./images/} }

% Specifically for paper formatting 
\renewcommand{\baselinestretch}{1.2} % Spaces manuscript for easy reading

% Formatting definitions, propositions, etc. 
\newtheorem{definition}{Definition}
\newtheorem{prop}{Proposition}
\newtheorem{lemma}{Lemma}
\newtheorem{thm}{Theorem}
\newtheorem{corollary}{Corollary}

% Title and author
\title{Calibration of Terrestrial Carbon Models}
\author{Andrew Roberts}

\begin{document}

\maketitle
\tableofcontents
\newpage

\section{General Setting: Computer Model Calibration}
\section{Motivating Example: Very Simple Ecosystem Model}
\subsection{Relation to the General Setting}

\section{Parameter Calibration Basics}
\subsection{Correlated Outputs}
In the previous section, we assumed independence between model outputs, which resulted in a convenient likelihood. To be clear, we applied the independence assumption
\begin{align}
p(y|\theta, \Sigma_\epsilon) = \prod_{j = 1}^{p} p(y_j|\theta, \sigma_{\epsilon_j}^2)
\end{align}
where $y \in \R^p$ is a single observation of the $p$ outputs. In the case that this assumption is inappropriate, the previous models may lead to inaccurate 
quantification of uncertainties. In this case, we consider an explicit model for the covariance between model outputs. We begin by introducing a simple model 
that generalizes $\Sigma_\epsilon = \text{diag}\left(\sigma_{\epsilon_1}^2, \dots, \sigma_{\epsilon_p}^2\right)$ from a diagonal to an arbitrary, dense covariance 
matrix. Since $\Sigma_\epsilon$ is now parameterized by $\frac{p(p+1)}{2}$ parameters, this model is only computational feasible when the number of outputs is 
quite small. Nevertheless, it is a good starting point to motivate more complicated models. 

Under a Gaussian observation model, the likelihood for a single observation $y_i \in \R^p$ now becomes 
\begin{align}
p(y_i|\theta, \Sigma_\epsilon) &= N_p(y_i|f(i, \theta), \Sigma_\epsilon).
\end{align}
The likelihood across all observations is thus
\begin{align}
p(Y|\theta, \Sigma_\epsilon) &= \prod_{i = 1}^{n} N_p(y_i|f(i, \theta), \Sigma_\epsilon). \label{corr_output_lik}
\end{align}
We should emphasize that we are still assuming independent observation noise across \textit{time}; this model only accounts for correlations across outputs 
independent of time. The assumption that this output covariance $\Sigma_\epsilon$ is constant across time can be relaxed, at the cost of further increasing the 
complexity of the model. A numerically stable implementation of the log of [\ref{corr_output_lik}] is provided in the appendix [\ref{corr_output_lik_implementation}].


\section{Surrogate Modeling}
\section{Multi-Site Hierarchical Models}

\section{Appendix}

\subsection{Correlated Gaussian Likelihood Implementation} \label{corr_output_lik_implementation}
We consider coding the log of [\ref{corr_output_lik}] efficiently, which is especially important when it will need to be evaluated at each iteration of an
MCMC algorithm. The full log-likelihood is given by 
\begin{align}
\log p(Y|\theta, \Sigma_\epsilon) &= -\frac{1}{2} \log(2\pi) -\frac{n}{2} \log \det(\Sigma_\epsilon) - \frac{1}{2} \sum_{i = 1}^{n} (y_i - f(i, \theta))^T \Sigma_\epsilon^{-1} (y - f(i, \theta)) \label{log_lik_corr_outputs}
\end{align}
Due to the assumption that the output covariance $\Sigma_\epsilon$ is constant across observations, we need only calculate the Cholesky decomposition 
$\Sigma_\epsilon = LL^T$ a single time per likelihood evaluation. Denoting $e_i := y - f(i, \theta) \in \R^p$, the summation in the third term can then be computed as
\begin{align}
\sum_{i = 1}^{n} e_i^T \Sigma_\epsilon^{-1} e_i &= \sum_{i = 1}^{n} e_i^T (L L^T)^{-1} e_i \\
									&= \sum_{i = 1}^{n} (L^{-1} e_i)^T (L^{-1} e_i) \nonumber \\
									&=  \sum_{i = 1}^{n} \alpha_i^T \alpha_i, \text{ where } \alpha_i := L^{-1}e_i, \nonumber \\
									&= \alpha^T \alpha, \text{ where } \alpha := (\alpha_1^T, \dots, \alpha_n^T)^T \in \R^{np}, \nonumber
\end{align}
where the $\alpha_i$ can be computed efficiently using forward substitution. Define $E \in \R^{p \times n}$ as the matrix with columns $e_1, \dots, e_n$. 
Then $L^{-1}E$ has columns $\alpha_1, \dots, \alpha_n$. Thus, we can form $\alpha$ by stacking the columns of $L^{-1}E$ into a long vector 
columnwise. 

Recall that the inference algorithm considered for the correlated output model consisted of a Metropolis-within-Gibbs sampler. In this algorithm, the likelihood 
[\ref{log_lik_corr_outputs}] must only be evaluated during the $\theta$ sampling step, conditional on $\Sigma_\epsilon$. Therefore, the second term containing the 
determinant may be treated as a constant and need not be evaluated. However, for completeness we provide the computation of this term using the Cholesky 
factor $L$ below. 
\begin{align}
\log \det(\Sigma_\epsilon) = \log \det(LL^T) &= \log \det(L)^2 \\
								  &= 2 \log \prod_{j = 1}^{p} L_{jj} \\
								  &= 2 \sum_{j = 1}^{p} \log L_{jj}
\end{align}
The full, normalized log-likelihood can thus be implemented as
\begin{align}
\log p(Y|\theta, \Sigma_\epsilon) &= -\frac{1}{2} \log(2\pi) - n\sum_{j = 1}^{p} \log L_{jj}  - \frac{1}{2} \alpha^T \alpha
\end{align}


\section{TODOs}
\begin{itemize}
\item Contrast setting with KOH setting that has both $X$ and $\theta$ variables. 
\item Standardize my notation in how I'm subscripting vs. superscripting variables to index observations vs. outputs. Also how to 
index $f(\theta)$ when referring to outputs and observations at different time steps. 
\item Establish notation $y_i$ for observations and $Y_j$ for time series of output $j$. 
\end{itemize}


\end{document}


