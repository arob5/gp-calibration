\documentclass[12pt]{article}
\RequirePackage[l2tabu, orthodox]{nag}
\usepackage[main=english]{babel}
\usepackage[rm={lining,tabular},sf={lining,tabular},tt={lining,tabular,monowidth}]{cfr-lm}
\usepackage{amsthm,amssymb,latexsym,gensymb,mathtools,mathrsfs}
\usepackage[T1]{fontenc}
\usepackage[utf8]{inputenc}
\usepackage[pdftex]{graphicx}
\usepackage{epstopdf,enumitem,microtype,dcolumn,booktabs,hyperref,url,fancyhdr}
\usepackage{algorithmic}
\usepackage[ruled,vlined,commentsnumbered,titlenotnumbered]{algorithm2e}
\usepackage{bbm}

% Plotting
\usepackage{pgfplots}
\usepackage{xinttools} % for the \xintFor***
\usepgfplotslibrary{fillbetween}
\pgfplotsset{compat=1.8}
\usepackage{tikz}

% Local custom commands. 
%!TEX root = SMA-book.tex

% Useful info on newcomand: https://tex.stackexchange.com/questions/117358/newcommand-argument-confusion


% --------------------------------------------------------------------------------------------------
% General Math.  
% --------------------------------------------------------------------------------------------------

% Common math commands. 
\newcommand*{\abs}[1]{\left\lvert#1\right\rvert}
\newcommand{\R}{\mathbb{R}}
\newcommand*{\suchthat}{\,\mathrel{\big|}\,}
\newcommand{\Exp}[1]{\exp\mathopen{}\left\{#1\right\}\mathclose{}}

\DeclareMathOperator*{\argmax}{argmax}
\DeclareMathOperator*{\argmin}{argmin}
\DeclarePairedDelimiterX\innerp[2]{(}{)}{#1\delimsize\vert\mathopen{}#2}
\DeclarePairedDelimiter{\ceil}{\lceil}{\rceil}

% Linear algebra.
\newcommand*{\norm}[1]{\left\lVert#1\right\rVert}
\newcommand{\bx}{\mathbf{x}}
\newcommand{\bw}{\mathbf{w}}
\newcommand{\oneVec}[1]{\boldsymbol{1}_{#1}}

% Probability. 
\newcommand{\E}{\mathbb{E}}
\newcommand{\Var}{\mathrm{Var}}
\newcommand{\Cov}{\mathrm{Cov}}
\newcommand{\Prob}{\mathbb{P}}

% --------------------------------------------------------------------------------------------------
% Bayesian Inverse Problems. 
% --------------------------------------------------------------------------------------------------

% General inverse problems.

\newcommand{\spar}{u} % Parameter (scalar, non-bold)
\newcommand{\bpar}{\mathbf{u}} % Parameter vector (bold)
\newcommand{\parSpace}{\mathcal{U}} % Parameter space
\newcommand{\fwd}{G} % Forward model
\newcommand{\Npar}{D} % Dimension of parameter space
\newcommand{\Nobj}{P} % Number of output variables/data constraints/objectives
\newcommand{\llik}{\mathcal{L}} % Log-likelihood 
\newcommand{\priorDens}{\pi_0} % Prior density 
\newcommand{\postDens}{\pi} % Posterior density
\newcommand{\objIdx}{p} % Primary index for indexing objective (i.e. data constraint or output variable) in multi-objective problems.
\newcommand{\indexObj}[2][\objIdx]{{#2}^{(#1)}} % Adds objective index to a variable.
\newcommand{\designMat}[1][\designIdx]{\mathbf{U}_{#1}} % Design matrix of input parameters (in rows) at which response is observed. 

% Gaussian likelihood, Inverse Gamma prior model. 

\newcommand{\IGShape}{\alpha} % Shape parameter for inverse Gamma prior.
\newcommand{\IGScale}{\beta} % Scale parameter for inverse Gamma prior. 
\newcommand{\sdObs}{\sigma} % Standard deviation for observation error of single objective in Gaussian likelihood. 
\newcommand{\CovObs}{\boldsymbol{\Sigma}} % Time-independent covariance matrix across all output variables/objectives
\newcommand{\SSR}{\Phi} % Sum of squared error (SSR) function, as a function of the parameter. 

\newcommand{\SSRVecPredOut}[2]{\boldsymbol{\SSR}_{#1}^{(#2)}} % TODO: change this 


% --------------------------------------------------------------------------------------------------
% Gaussian Processes. 
% --------------------------------------------------------------------------------------------------

% Generic GP commands. 
\newcommand{\GP}{\mathcal{GP}} % GP distribution.
\newcommand{\f}{f} % Generic univariate target function on which GP prior is placed. 
\newcommand{\fObs}[1][\designIdx]{\mathbf{\f}_{#1}} % Vector of observed target function evaluations.  
\newcommand{\GPMean}{\mu} % Generic GP prior mean function
\newcommand{\GPKer}{k} % Generic GP prior covariance function
\newcommand{\designData}{\mathcal{D}} % Set of training/design inputs and responses  
\newcommand{\GPMeanPred}[1][\designIdx]{\GPMean_{#1}} % GP predictive mean (conditioned on data).
\newcommand{\GPKerPred}[1][\designIdx]{\GPKer_{#1}} % GP predictive covariance function (conditioned on data). 
\newcommand{\KerMat}[1][\designIdx]{\mathbf{K}_{#1}} % Generic kernel matrix, explicitly showing dependence on number of design points.

\newcommand{\bbf}{\mathbf{f}} % Bold version of target function; e.g. to be used to denote a vector of function evaluations. 
\newcommand{\GPMeanOut}[1]{\mu^{(#1)}} % Univariate GP prior mean function for specific output variable 
\newcommand{\GPMeanPredOut}[2]{\mu_{#1}^{(#2)}} % Univariate GP mean, conditioned on data, for specific output variable
\newcommand{\GPKerOut}[1]{k^{(#1)}} % Univariate GP prior covariance function for specific output variable  
\newcommand{\GPKerPredOut}[2]{k_{#1}^{(#2)}} % Univariate GP covariance, conditioned on data, for specific output variable

\newcommand{\KerMatOut}[2]{\mathbf{K}_{#1}^{(#2)}} % Kernel matrix for specific output 

% GP emulators for high-dimensional outputs using basis function approach (Higdon et al). 
\newcommand{\fwdOutputMat}{\mathbf{M}} % In the case Nout = 1, this is the Ntime x Ndesign matrix of computer model outputs at the design points. 
\newcommand{\basisOutputVec}{\mathbf{v}} % Basis vector for representing computer model output. 
\newcommand{\basisMat}{\mathbf{V}} % Matrix containing basis vectors for computer model output as columns. 
\newcommand{\NbasisVec}{R} % Number of basis vectors used in representing computer model output. 
\newcommand{\basisVecIdx}{r} % Main index for indexing basis vectors used in representing computer model output. 
\newcommand{\basisVecWeight}{w} % Weight scaling basis vectors used in representing computer model output; these are emulated by GPs. 
\newcommand{\basisWeightMat}{\mathbf{W}} % Matrix of weights, of dimension Ndesign x NbasisVec
\newcommand{\centerVec}{\mathbf{m}} % Vector used to center the computer model outputs. 
\newcommand{\scaleVec}{\mathbf{s}} %  Vector used to scale the computer model outputs. 

% Loss emulation approach; emulating the L2 error functions. 
\newcommand{\SSRObs}{\mathbf{\SSR}} % Vector of observed SSR responses. 

% --------------------------------------------------------------------------------------------------
% Dynamical Models. 
% --------------------------------------------------------------------------------------------------

% ODEs

\newcommand{\timeIdx}{t} % Primary index for time step
\newcommand{\Ntime}{T} % Number of time steps
\newcommand{\NTimeOut}[1]{T_{#1}} % Number of time steps \NTimeOut -> \indexObj{\Ntime}
\newcommand{\indexTime}[2][\timeIdx]{{#2}_{#1}} % Adds time index to a variable.
\newcommand{\state}{y} % Generic state (scalar, non-bold)
\newcommand{\bstate}{\mathbf{y}} % Generic state vector (bold)
\newcommand{\stateOut}[1]{\bstate^{(#1)}} % State vector for specific output variable
\newcommand{\stateTime}[1]{\bstate_{#1}} % State vector at specific time across all outputs 
\newcommand{\stateTimeOut}[2]{\state_{#1}^{(#2)}} % State at specific time and for specific output
\newcommand{\stateMat}{\mathbf{Y}} % Matrix of observed states across all time steps (rows) and output variables (columns) 
\newcommand{\bforce}{\mathbf{w}} % Forcing input (bold)
\newcommand{\fwdOne}{g} % One-step flow map

% --------------------------------------------------------------------------------------------------
% Experimental/Sequential Design and Optimization. 
% --------------------------------------------------------------------------------------------------

\newcommand{\Ninit}{N_0} % Initial number of design points/size of space-filling design 
\newcommand{\Ndesign}{N} % Total number of design points  
\newcommand{\designIdx}{n} % Primary index for design points (sequential design)
\newcommand{\batchIdx}{b}  % Primary index for rounds in sequential batch design
\newcommand{\indexDesign}[2][\designIdx]{{#2}_{#1}} % Adds design iteration index to a variable.
\newcommand{\indexDesignObj}[3][\objIdx]{{#2}^{(#1)}_{#3}} % Adds design iteration index and objective/constraint index to a variable.
\newcommand{\currParMax}[1]{\bpar_{#1}^{\text{max}}} % Value of design point (calibration parameter) that currently has highest posterior density
\newcommand{\currMax}[1]{\postDens_{#1}^{\text{max}}} % Current max value of posterior density
\newcommand{\acq}[1][]{\mathcal{A}_{#1}} % Acquisition function

% --------------------------------------------------------------------------------------------------
% Markov Chain Monte Carlo. 
% --------------------------------------------------------------------------------------------------

% Generic MCMC commands. 
\newcommand{\CovProp}{\mathbf{C}} % Proposal covariance matrix
\newcommand{\NMCMC}{T_{\text{MCMC}}} % Proposal covariance matrix
\newcommand{\accProbMH}{\alpha} % Metropolis-Hastings acceptance probability



\setlist{topsep=1ex,parsep=1ex,itemsep=0ex}
\setlist[1]{leftmargin=\parindent}
\setlist[enumerate,1]{label=\arabic*.,ref=\arabic*}
\setlist[enumerate,2]{label=(\alph*),ref=(\alph*)}

% For embedding images
\graphicspath{ {./images/} }

% Specifically for paper formatting 
\renewcommand{\baselinestretch}{1.2} % Spaces manuscript for easy reading

% Formatting definitions, propositions, etc. 
\newtheorem{definition}{Definition}
\newtheorem{prop}{Proposition}
\newtheorem{lemma}{Lemma}
\newtheorem{thm}{Theorem}
\newtheorem{corollary}{Corollary}

% Title and author
\title{Gaussian Process Accelerated MCMC Algorithm}
\author{Andrew Roberts}

\begin{document}

\section{Setup and Notation}
We begin by briefly introducing the statistical calibration model for parameter calibration. Let 
\begin{align*}
\fwd: \R^{\Npar} \to \R^{\Ntime \times \Nobj}
\end{align*}
denote the forward model which maps unknown calibration parameters $\bpar \in \parSpace \subseteq \R^{\Npar}$ to model predictions 
$\fwd(\bpar) \in \R^{\Ntime \times \Nobj}$. The outputs consist of time series for $\Nobj$ different outputs, each of length $\Ntime$. 
Individual outputs, corresponding to columns of the matrix $\fwd(\bpar)$, are denoted by $\indexObj{\fwd}(\bpar)$. 
The task is to infer the value of the parameters $\bpar$ by comparing predictions $\fwd(\bpar)$ to noisy 
observations $\{\indexObj{\bstate}\}_{\objIdx=1}^{\Nobj}$ of the true system being modeled, with observation dimensions 
$\indexObj{\bstate} \in \R^{\indexObj{\Ntime}}$ potentially varying due to missing data. We write 
$\stateMat := \{\indexObj[1]{\bstate}, \dots, \indexObj[\Nobj]{\bstate} \}$ to denote all observed data. 

We connect the observations to the model predictions via the (log) likelihood 
\begin{align}
\llik(\bpar, \CovObs) 
&:= \log p(\stateMat|\bpar, \CovObs) \\
&= \log \prod_{\objIdx=1}^{\Nobj} \Gaussian(\indexObj{\bstate} | \indexObj{\fwd}(\bpar), \sdObs^2_{\objIdx}) \\
&= -\frac{1}{2} \sum_{\objIdx=1}^{\Nobj} \indexObj{\Ntime} \log(2\pi \sdObs^2_{\objIdx}) - \frac{1}{2} \sum_{\objIdx=1}^{\Nobj} \frac{\indexObj{\SSR}(\bpar)}{\sdObs^2_{\objIdx}}
\end{align}
where we denote the \textit{model-data misfit} by 
\begin{align}
\indexObj{\SSR}(\bpar) := \norm{\indexObj{\bstate} - \indexObj{\fwd}(\bpar)}_2^2,
\end{align}
 and write $\CovObs := \{\sdObs^2_1, \dots, \sdObs^2_{\Nobj}\}$ to denote the variance parameters for each output. 




Let $\llik(\bpar, \CovObs) := \llik(\bpar |\CovObs, \bstate)$ denote 
a log-likelihood function which quantifies discrepancy between the forward model predictions and observed data. The 
explicit dependence on $\bstate$ dropped from the notation as the data is assumed constant throughout the analysis. We use $\CovObs$ to denote 
\textit{likelihood parameters}; that is, any unknown parameters other than $\bpar$ used to define the likelihood. This paper will focus on Gaussian likelihoods, 
where the likelihood parameters take the form of an unknown positive-definite covariance matrix 
\begin{align}
\llik(\bpar, \CovObs) &= \log \Gaussian_{\Ntime}\left(\bstate | \fwd(\bpar), \CovObs \right) \nonumber \\
			       &= -\frac{\Ntime}{2} \log(2\pi) - \frac{1}{2} \log \det(\CovObs) - \frac{1}{2} (\bstate - \fwd(\bpar))^\top \CovObs^{-1} (\bstate - \fwd(\bpar)) \label{llik}
\end{align}
In addition to the likelihood, the Bayesian framework also requires a prior distribution $\priorDens(\bpar, \CovObs)$, 
which encodes existing knowledge about both the calibration and likelihood parameters. 
The Bayesian inverse problem is solved by obtaining the 
posterior distribution
\begin{align*}
\postDens(\bpar, \CovObs) &:= \frac{\Exp{\llik(\bpar, \CovObs)}\priorDens(\bpar, \CovObs)}{\int \Exp{\llik(\bpar, \CovObs)}\priorDens(\bpar, \CovObs) d\bpar d\CovObs},
\end{align*}
which synthesizes the prior knowledge with the observed data. Typically, interest centers on the marginal posterior for $\bpar$, which marginalizes over the nuisance 
parameters $\CovObs$,
\begin{align*}
\postDens(\bpar) &:= \int_{\parSpace} \postDens(\bpar, \CovObs) d\CovObs.
\end{align*}
For notation, we let $\logUnPost(\bpar, \CovObs)$ denote the unnormalized log posterior density, such that 
\begin{align*}
\logUnPost(\bpar, \CovObs) := \llik(\bpar, \CovObs) + \Log{\priorDens(\bpar, \CovObs)}.
\end{align*}
The posterior distribution is typically summarized via samples and sample-based approximations of expectations of interest, i.e. $\E_{\bpar \sim \postDens}[\phi(\bpar)]$.
However, the standard approach of applying a Markov Chain Monte Carlo (MCMC) algorithm to produce samples from $\postDens$ often requires 
$\BigO(10^4)$ or more iterations, each of which requires evaluating $\logUnPost(\bpar, \CovObs)$, which 
in turn entails computing a forward model evaluation $\fwd(\bpar)$. In the  case where the forward model takes the form of an expensive computer simulation, this 
can render MCMC intractable. To address this issue, a variety of techniques have emerged which replace the expensive computations 
$\logUnPost(\bpar, \CovObs)$ with a cheaper statistical approximation known as an \textit{emulator} or \textit{surrogate}. This paper focuses on Gaussian process 
emulators, which are briefly introduced below before returning to the question of model emulation. 


\end{document}