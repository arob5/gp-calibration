\documentclass[12pt]{article}
\RequirePackage[l2tabu, orthodox]{nag}
\usepackage[main=english]{babel}
\usepackage[rm={lining,tabular},sf={lining,tabular},tt={lining,tabular,monowidth}]{cfr-lm}
\usepackage{amsthm,amssymb,latexsym,gensymb,mathtools,mathrsfs}
\usepackage[T1]{fontenc}
\usepackage[utf8]{inputenc}
\usepackage[pdftex]{graphicx}
\usepackage{caption}
\usepackage{subcaption}
\usepackage{epstopdf,enumitem,microtype,dcolumn,booktabs,hyperref,url,fancyhdr}

% Plotting
\usepackage{pgfplots}
\usepackage{xinttools} % for the \xintFor***
\usepgfplotslibrary{fillbetween}
\pgfplotsset{compat=1.8}
\usepackage{tikz}

% Custom Commands
\newcommand*{\norm}[1]{\left\lVert#1\right\rVert}
\newcommand*{\abs}[1]{\left\lvert#1\right\rvert}
\newcommand*{\suchthat}{\,\mathrel{\big|}\,}
\newcommand{\E}{\mathbb{E}}
\newcommand{\Var}{\mathrm{Var}}
\newcommand{\R}{\mathcal{R}}
\newcommand{\N}{\mathcal{N}}
\newcommand{\Ker}{\mathrm{Ker}}
\newcommand{\Cov}{\mathrm{Cov}}
\newcommand{\Prob}{\mathbb{P}}
\DeclarePairedDelimiterX\innerp[2]{(}{)}{#1\delimsize\vert\mathopen{}#2}
\DeclareMathOperator*{\argmax}{argmax}
\DeclareMathOperator*{\argmin}{argmin}
\def\R{\mathbb{R}}
\DeclarePairedDelimiter\ceil{\lceil}{\rceil}
\DeclarePairedDelimiter\floor{\lfloor}{\rfloor}
\newcommand*{\vertbar}{\rule[-1ex]{0.5pt}{2.5ex}} % For lines in matrix to represent columns
\newcommand*{\horzbar}{\rule[.5ex]{2.5ex}{0.5pt}} % For lines in matrix to represent rows

\setlist{topsep=1ex,parsep=1ex,itemsep=0ex}
\setlist[1]{leftmargin=\parindent}
\setlist[enumerate,1]{label=\arabic*.,ref=\arabic*}
\setlist[enumerate,2]{label=(\alph*),ref=(\alph*)}

% Specifically for paper formatting 
\renewcommand{\baselinestretch}{1.2} % Spaces manuscript for easy reading

% Formatting definitions, propositions, etc. 
\newtheorem{definition}{Definition}
\newtheorem{prop}{Proposition}
\newtheorem{lemma}{Lemma}
\newtheorem{thm}{Theorem}
\newtheorem{corollary}{Corollary}
\newtheorem{notation}{Notation}

\begin{document}

\begin{center}
\Large
Questions on GP Emulation/Calibration Methodology 
\end{center}

\section{Methodological Foundations}
I have been using Robert Gramacy's free online \href{https://bobby.gramacy.com/surrogates/}{textbook} on GPs and surrogate modeling as a primary reference on calibration of 
computer models. It is quite recent and nicely synthesizes many of the important papers on the subject. I do use my own notation here though, as I find that his can contain a somewhat
excessive number of sub and superscripts. I'll just walk through my understanding of the methodology, asking questions along the way. 

\subsection{Defining the Parameters of Interest}
The \textit{computer model} (i.e. \textit{simulator} or \textit{process-based model}, as referred to in the PEcAn papers) is deterministic and hence can be modeled as a function
$f: \mathcal{P} \to \R$, where $\mathcal{P} \subset \R^d$ is the parameter space of the computer model. For now I am assuming scalar output for simplicity, although I know that 
most of the PEcAn models have many outputs. The Kennedy and O'Hagan calibration framework (summarized in Gramacy's book) distinguishes between two types of parameters, so 
that 
\[\mathcal{P} = \mathcal{X} \times \mathcal{U}\]
where $\mathcal{X}$ is the space of \textit{observational parameters} and $\mathcal{U}$ the space of \textit{calibration parameters}. The former refers to those parameters that are 
observed in the real-world data and can also be included in the computer model. The calibration parameters cannot be observed in the real-world; this can mean that they have no 
real-world meaning (e.g. the mesh size in the computer model) or that they do exist in the real-world but cannot be directly observed. To my understanding, the observation parameters
are just like covariates in a typical regression, in the sense that the analysis will be conducted conditional on the values of the observation parameters. The calibration parameters are 
the true objects of interest; we seek to tune these parameters using field data as well as understand the uncertainty in the computer model as a function of the calibration parameters. 

\subsubsection{Questions}
\begin{itemize}
\item Is the distinction between observation and calibration parameters relevant for the PEcAn models? My understanding is that what you refer to as \textit{process-model parameters}
refers to calibration parameters. I would assume there is also some notion of observation parameters. 
\end{itemize}

\subsection{Statistical Model Relating Computer Simulation and Observational Data}



\section{PEcAn Code}


\end{document} 





