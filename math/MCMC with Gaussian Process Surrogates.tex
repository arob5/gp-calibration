\documentclass[12pt]{article}
\RequirePackage[l2tabu, orthodox]{nag}
\usepackage[main=english]{babel}
\usepackage[rm={lining,tabular},sf={lining,tabular},tt={lining,tabular,monowidth}]{cfr-lm}
\usepackage{amsthm,amssymb,latexsym,gensymb,mathtools,mathrsfs}
\usepackage[T1]{fontenc}
\usepackage[utf8]{inputenc}
\usepackage[pdftex]{graphicx}
\usepackage{epstopdf,enumitem,microtype,dcolumn,booktabs,hyperref,url,fancyhdr}
\usepackage{algorithmic}
\usepackage[ruled,vlined,commentsnumbered,titlenotnumbered]{algorithm2e}
\usepackage{bbm}

% Plotting
\usepackage{pgfplots}
\usepackage{xinttools} % for the \xintFor***
\usepgfplotslibrary{fillbetween}
\pgfplotsset{compat=1.8}
\usepackage{tikz}

% Local custom commands. 
%!TEX root = SMA-book.tex

% Useful info on newcomand: https://tex.stackexchange.com/questions/117358/newcommand-argument-confusion


% --------------------------------------------------------------------------------------------------
% General Math.  
% --------------------------------------------------------------------------------------------------

% Common math commands. 
\newcommand*{\abs}[1]{\left\lvert#1\right\rvert}
\newcommand{\R}{\mathbb{R}}
\newcommand*{\suchthat}{\,\mathrel{\big|}\,}
\newcommand{\Exp}[1]{\exp\mathopen{}\left\{#1\right\}\mathclose{}}

\DeclareMathOperator*{\argmax}{argmax}
\DeclareMathOperator*{\argmin}{argmin}
\DeclarePairedDelimiterX\innerp[2]{(}{)}{#1\delimsize\vert\mathopen{}#2}
\DeclarePairedDelimiter{\ceil}{\lceil}{\rceil}

% Linear algebra.
\newcommand*{\norm}[1]{\left\lVert#1\right\rVert}
\newcommand{\bx}{\mathbf{x}}
\newcommand{\bw}{\mathbf{w}}
\newcommand{\oneVec}[1]{\boldsymbol{1}_{#1}}

% Probability. 
\newcommand{\E}{\mathbb{E}}
\newcommand{\Var}{\mathrm{Var}}
\newcommand{\Cov}{\mathrm{Cov}}
\newcommand{\Prob}{\mathbb{P}}

% --------------------------------------------------------------------------------------------------
% Bayesian Inverse Problems. 
% --------------------------------------------------------------------------------------------------

% General inverse problems.

\newcommand{\spar}{u} % Parameter (scalar, non-bold)
\newcommand{\bpar}{\mathbf{u}} % Parameter vector (bold)
\newcommand{\parSpace}{\mathcal{U}} % Parameter space
\newcommand{\fwd}{G} % Forward model
\newcommand{\Npar}{D} % Dimension of parameter space
\newcommand{\Nobj}{P} % Number of output variables/data constraints/objectives
\newcommand{\llik}{\mathcal{L}} % Log-likelihood 
\newcommand{\priorDens}{\pi_0} % Prior density 
\newcommand{\postDens}{\pi} % Posterior density
\newcommand{\objIdx}{p} % Primary index for indexing objective (i.e. data constraint or output variable) in multi-objective problems.
\newcommand{\indexObj}[2][\objIdx]{{#2}^{(#1)}} % Adds objective index to a variable.
\newcommand{\designMat}[1][\designIdx]{\mathbf{U}_{#1}} % Design matrix of input parameters (in rows) at which response is observed. 

% Gaussian likelihood, Inverse Gamma prior model. 

\newcommand{\IGShape}{\alpha} % Shape parameter for inverse Gamma prior.
\newcommand{\IGScale}{\beta} % Scale parameter for inverse Gamma prior. 
\newcommand{\sdObs}{\sigma} % Standard deviation for observation error of single objective in Gaussian likelihood. 
\newcommand{\CovObs}{\boldsymbol{\Sigma}} % Time-independent covariance matrix across all output variables/objectives
\newcommand{\SSR}{\Phi} % Sum of squared error (SSR) function, as a function of the parameter. 

\newcommand{\SSRVecPredOut}[2]{\boldsymbol{\SSR}_{#1}^{(#2)}} % TODO: change this 


% --------------------------------------------------------------------------------------------------
% Gaussian Processes. 
% --------------------------------------------------------------------------------------------------

% Generic GP commands. 
\newcommand{\GP}{\mathcal{GP}} % GP distribution.
\newcommand{\f}{f} % Generic univariate target function on which GP prior is placed. 
\newcommand{\fObs}[1][\designIdx]{\mathbf{\f}_{#1}} % Vector of observed target function evaluations.  
\newcommand{\GPMean}{\mu} % Generic GP prior mean function
\newcommand{\GPKer}{k} % Generic GP prior covariance function
\newcommand{\designData}{\mathcal{D}} % Set of training/design inputs and responses  
\newcommand{\GPMeanPred}[1][\designIdx]{\GPMean_{#1}} % GP predictive mean (conditioned on data).
\newcommand{\GPKerPred}[1][\designIdx]{\GPKer_{#1}} % GP predictive covariance function (conditioned on data). 
\newcommand{\KerMat}[1][\designIdx]{\mathbf{K}_{#1}} % Generic kernel matrix, explicitly showing dependence on number of design points.

\newcommand{\bbf}{\mathbf{f}} % Bold version of target function; e.g. to be used to denote a vector of function evaluations. 
\newcommand{\GPMeanOut}[1]{\mu^{(#1)}} % Univariate GP prior mean function for specific output variable 
\newcommand{\GPMeanPredOut}[2]{\mu_{#1}^{(#2)}} % Univariate GP mean, conditioned on data, for specific output variable
\newcommand{\GPKerOut}[1]{k^{(#1)}} % Univariate GP prior covariance function for specific output variable  
\newcommand{\GPKerPredOut}[2]{k_{#1}^{(#2)}} % Univariate GP covariance, conditioned on data, for specific output variable

\newcommand{\KerMatOut}[2]{\mathbf{K}_{#1}^{(#2)}} % Kernel matrix for specific output 

% GP emulators for high-dimensional outputs using basis function approach (Higdon et al). 
\newcommand{\fwdOutputMat}{\mathbf{M}} % In the case Nout = 1, this is the Ntime x Ndesign matrix of computer model outputs at the design points. 
\newcommand{\basisOutputVec}{\mathbf{v}} % Basis vector for representing computer model output. 
\newcommand{\basisMat}{\mathbf{V}} % Matrix containing basis vectors for computer model output as columns. 
\newcommand{\NbasisVec}{R} % Number of basis vectors used in representing computer model output. 
\newcommand{\basisVecIdx}{r} % Main index for indexing basis vectors used in representing computer model output. 
\newcommand{\basisVecWeight}{w} % Weight scaling basis vectors used in representing computer model output; these are emulated by GPs. 
\newcommand{\basisWeightMat}{\mathbf{W}} % Matrix of weights, of dimension Ndesign x NbasisVec
\newcommand{\centerVec}{\mathbf{m}} % Vector used to center the computer model outputs. 
\newcommand{\scaleVec}{\mathbf{s}} %  Vector used to scale the computer model outputs. 

% Loss emulation approach; emulating the L2 error functions. 
\newcommand{\SSRObs}{\mathbf{\SSR}} % Vector of observed SSR responses. 

% --------------------------------------------------------------------------------------------------
% Dynamical Models. 
% --------------------------------------------------------------------------------------------------

% ODEs

\newcommand{\timeIdx}{t} % Primary index for time step
\newcommand{\Ntime}{T} % Number of time steps
\newcommand{\NTimeOut}[1]{T_{#1}} % Number of time steps \NTimeOut -> \indexObj{\Ntime}
\newcommand{\indexTime}[2][\timeIdx]{{#2}_{#1}} % Adds time index to a variable.
\newcommand{\state}{y} % Generic state (scalar, non-bold)
\newcommand{\bstate}{\mathbf{y}} % Generic state vector (bold)
\newcommand{\stateOut}[1]{\bstate^{(#1)}} % State vector for specific output variable
\newcommand{\stateTime}[1]{\bstate_{#1}} % State vector at specific time across all outputs 
\newcommand{\stateTimeOut}[2]{\state_{#1}^{(#2)}} % State at specific time and for specific output
\newcommand{\stateMat}{\mathbf{Y}} % Matrix of observed states across all time steps (rows) and output variables (columns) 
\newcommand{\bforce}{\mathbf{w}} % Forcing input (bold)
\newcommand{\fwdOne}{g} % One-step flow map

% --------------------------------------------------------------------------------------------------
% Experimental/Sequential Design and Optimization. 
% --------------------------------------------------------------------------------------------------

\newcommand{\Ninit}{N_0} % Initial number of design points/size of space-filling design 
\newcommand{\Ndesign}{N} % Total number of design points  
\newcommand{\designIdx}{n} % Primary index for design points (sequential design)
\newcommand{\batchIdx}{b}  % Primary index for rounds in sequential batch design
\newcommand{\indexDesign}[2][\designIdx]{{#2}_{#1}} % Adds design iteration index to a variable.
\newcommand{\indexDesignObj}[3][\objIdx]{{#2}^{(#1)}_{#3}} % Adds design iteration index and objective/constraint index to a variable.
\newcommand{\currParMax}[1]{\bpar_{#1}^{\text{max}}} % Value of design point (calibration parameter) that currently has highest posterior density
\newcommand{\currMax}[1]{\postDens_{#1}^{\text{max}}} % Current max value of posterior density
\newcommand{\acq}[1][]{\mathcal{A}_{#1}} % Acquisition function

% --------------------------------------------------------------------------------------------------
% Markov Chain Monte Carlo. 
% --------------------------------------------------------------------------------------------------

% Generic MCMC commands. 
\newcommand{\CovProp}{\mathbf{C}} % Proposal covariance matrix
\newcommand{\NMCMC}{T_{\text{MCMC}}} % Proposal covariance matrix
\newcommand{\accProbMH}{\alpha} % Metropolis-Hastings acceptance probability


\newcommand{\bphi}{\boldsymbol{\phi}}

\setlist{topsep=1ex,parsep=1ex,itemsep=0ex}
\setlist[1]{leftmargin=\parindent}
\setlist[enumerate,1]{label=\arabic*.,ref=\arabic*}
\setlist[enumerate,2]{label=(\alph*),ref=(\alph*)}

% For embedding images
\graphicspath{ {./images/} }

% Specifically for paper formatting 
\renewcommand{\baselinestretch}{1.2} % Spaces manuscript for easy reading

% Formatting definitions, propositions, etc. 
\newtheorem{definition}{Definition}
\newtheorem{prop}{Proposition}
\newtheorem{lemma}{Lemma}
\newtheorem{thm}{Theorem}
\newtheorem{corollary}{Corollary}

% Title and author
\title{MCMC with Gaussian Process Log-Likelihood Surrogates}
\author{Andrew Roberts}

\begin{document}

\maketitle

% Problem Setting
\section{Problem Setting}
We study methods of approximate inference in Bayesian settings which combine a Gaussian process (GP) approximation of the log-likelihood 
with Markov Chain Monte Carlo (MCMC) algorithms. Generically, the goal is to characterize a posterior distribution with (Lebesgue) density 
given by 
\begin{align}
\postDens(\bpar) &= \frac{p(\dataOut|\bpar)\priorDens(\bpar)}{\int p(\dataOut|\bpar)\priorDens(\bpar) d\bpar}, \label{generic_posterior}
\end{align}
where $\bpar \in \parSpace \subseteq \R^{\Npar}$. We consider the setting where evaluation of the likelihood $p(\dataOut|\bpar)$ is very 
computationally demanding, rendering standard MCMC algorithms intractable. This setting commonly occurs when \ref{generic_posterior} is the 
posterior distribution for a Bayesian inverse problem with an expensive forward model; e.g. where the likelihood arises via the data-generating 
process 
\begin{align*}
\dataOut &= \fwd(\bpar) + \boldsymbol{\epsilon},
\end{align*}
and evaluation of the forward model $\fwd(\bpar)$ requires a significant amount of time. A well-established solution to circumvent the intractability of standard MCMC is 
to leverage a cheaper surrogate model (also called an \textit{emulator} or \textit{meta-model}), which is trained from a small number of forward model 
runs, and then proceed to use the surrogate in place of the expensive model in subsequent inference algorithms. In the fields of computer modeling and 
Bayesian inverse problems, the typical approach is to replace the forward model $\fwd$ with an emulator. We instead focus on the alternative 
method of directly emulating the log-likelihood $\llik(\bpar) := \log p(\bpar|\dataOut)$. In particular, we assume uncertainty in the log-likelihood is represented by a Gaussian 
process (GP), 
\begin{align}
\llik_{\Ndesign} \sim \GP(\GPMeanPred[\Ndesign], \GPKerPred[\Ndesign]),
\end{align}
corresponding to a GP prior $\llik \sim \GP(\GPMean, \GPKer)$ which has been conditioned on a dataset 
$\designData := \{\designMat[], \designOutLik\}$ consisting of $\Ndesign$ \textit{design points} 
$\bpar_1, \dots, \bpar_{\Ndesign}$ stacked in the rows of $\designMat[]$, and their corresponding outputs $\designOutLik_{\designIdx} = \llik(\bpar_{\designIdx})$. The 
mean function $\GPMeanPred[\Ndesign]$ and kernel $\GPKerPred[\Ndesign]$ thus pertain to the GP predictive distribution and are given by 
\begin{align*}
\GPMeanPred[\Ndesign](\newInputMat) &= \GPMean(\newInputMat) + \GPKer(\newInputMat, \designMat[]) \KerMat[]^{-1} \left(\designOutLik - \GPMean(\designMat[]) \right) \\
\GPKerPred[\Ndesign](\newInputMat) &= \GPKer(\newInputMat) - \GPKer(\newInputMat, \designMat[]) \KerMat[]^{-1} \GPKer(\designMat[], \newInputMat),
\end{align*}
where $\KerMat[] \in \R^{\Ndesign \times \Ndesign}$ is the \textit{kernel matrix} with entries $\KerMat[n,m] = \GPKer(\bpar_n, \bpar_m)$. 

We investigate two questions: 
\begin{enumerate}
\item How should the GP emulator $\llik_{\Ndesign}$ be used to define a target posterior that approximates the true posterior, and which MCMC 
strategies are effective for sampling from this target?
\item What is an effective batch sequential design strategy for refining the emulator $\llik_{\Ndesign}$? 
\end{enumerate}

% Literature Review.
\section{Literature Review}

% Gaussian Process Posterior Approximations.
\section{Gaussian Process Posterior Approximations}
We now investigate different approaches for utilizing the emulator $\llik_{\Ndesign}$ to define an approximate inference scheme. Replacing the true log-likelihood 
with the emulator results in a random posterior approximation 
\begin{align}
\postDens_{\Ndesign}(\bpar)
&:= \frac{\priorDens(\bpar) \exp\left\{\llik_{\Ndesign}(\bpar) \right\}}{\int \priorDens(\bpar) \exp\left\{\llik_{\Ndesign}(\bpar) \right\} d\bpar}.
\end{align}
The randomness in $\postDens_{\Ndesign}(\bpar)$ results from the fact that the expression depends on the GP $\llik_{\Ndesign}(\bpar)$. We consider 
two general classes of methods for turning this random quantity into a well-defined sampling algorithm: (i.) methods which select a single deterministic 
posterior density from the distribution over all possible densities, and (ii.) methods which retain the randomness and result in noisy MCMC algorithms. 

% Deriving a Deterministic Posterior Approximation.
\subsection{Deriving a Deterministic Posterior Approximation}
We now consider producing a point estimate of $\postDens_{\Ndesign}(\bpar)$ which ideally propagates the emulator uncertainty. An argument could 
be made for the point estimator 
\begin{align}
\hat{\postDens}^*(\bpar) 
&:= \E_{\llik_{\Ndesign} \sim \GP(\GPMeanPred[\Ndesign], \GPKerPred[\Ndesign])} \left[ \postDens_{\Ndesign} \right] \nonumber \\
&= \E_{\llik_{\Ndesign} \sim \GP(\GPMeanPred[\Ndesign], \GPKerPred[\Ndesign])} \left[ \frac{\priorDens(\bpar) \exp\left\{\llik_{\Ndesign}(\bpar) \right\}}{\int \priorDens(\bpar) \exp\left\{\llik_{\Ndesign}(\bpar) \right\} d\bpar} \right] \label{ideal_point_estimate},
\end{align}
but the integral in the normalizing constant in \ref{ideal_point_estimate} renders this expectation intractable. Note that \ref{ideal_point_estimate} considers taking the 
expectation with respect to the entire GP distribution. An alternative, computationally tractable, approach instead considers taking the expectation of the 
likelihood $\exp\left\{\llik_{\Ndesign}(\bpar)\right\}$ separately for each $\bpar$, and then normalizing the result afterwords
\begin{align}
\hat{\postDens}(\bpar) 
&:= \frac{\priorDens(\bpar) \E\left[\exp\left\{\llik_{\Ndesign}(\bpar)\right\} \right]}{\int \priorDens(\bpar) \E\left[\exp\left\{\llik_{\Ndesign}(\bpar)\right\} \right] d\bpar} \\
&= \frac{\priorDens(\bpar) \exp\left\{\GPMeanPred[\Ndesign](\bpar) + \frac{1}{2}\GPKerPred[\Ndesign](\bpar) \right\}}{\int \priorDens(\bpar) \exp\left\{\GPMeanPred[\Ndesign](\bpar) + \frac{1}{2}\GPKerPred[\Ndesign](\bpar) \right\} d\bpar},
\end{align}
where the second line follows from the expectation formula for a log-normal random variable. 

% Noisy MCMC Approaches
\subsection{Noisy MCMC Approaches}
The previous approaches are attractive in that they target well-defined posterior distributions, but from an uncertainty quantification viewpoint it may
be unsatisfying to reduce to a point estimate. Intuitively, one approach to propagate the GP uncertainty might be to draw a sample from the 
distribution of $\postDens_{\Ndesign}(\bpar)$ for each iteration of an MCMC algorithm. Formally, we might consider the extended state space 
$(\bpar, \llik_{\Ndesign})$, in which case the random posterior density sample would manifest as a Gibbs sampling step for $\llik_{\Ndesign}$. 
Note that $\llik_{\Ndesign}$ is infinite-dimensional so that this notion is purely formal at this point. 


\end{document}






