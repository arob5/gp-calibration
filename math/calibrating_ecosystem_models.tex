\documentclass[12pt]{article}
\RequirePackage[l2tabu, orthodox]{nag}
\usepackage[main=english]{babel}
\usepackage[rm={lining,tabular},sf={lining,tabular},tt={lining,tabular,monowidth}]{cfr-lm}
\usepackage[T1]{fontenc}
\usepackage[utf8]{inputenc}
\usepackage[pdftex]{graphicx}
\usepackage{amsthm,amssymb,latexsym,gensymb,mathtools,mathrsfs}
\usepackage{epstopdf,enumitem,microtype,dcolumn,booktabs,hyperref,url,fancyhdr}
\usepackage{algorithm}
\usepackage{algpseudocode} % Note that this also loads algorithmicx
\usepackage{cleveref}
\usepackage{bbm}
\usepackage{caption, subcaption} % Captions and sub-figures. 
% \usepackage[demo]{graphicx}

% Plotting
\usepackage{pgfplots}
\usepackage{xinttools} % for the \xintFor***
\usepgfplotslibrary{fillbetween}
\pgfplotsset{compat=1.8}
\usepackage{tikz}

% Local custom commands. 
%!TEX root = SMA-book.tex

% Useful info on newcomand: https://tex.stackexchange.com/questions/117358/newcommand-argument-confusion


% --------------------------------------------------------------------------------------------------
% General commands.  
% --------------------------------------------------------------------------------------------------

\newcommand{\todo}{\textbf{TODO}}


% --------------------------------------------------------------------------------------------------
% General Math.  
% --------------------------------------------------------------------------------------------------

% Common math commands. 
\newcommand*{\abs}[1]{\left\lvert#1\right\rvert}
\newcommand{\R}{\mathbb{R}}
\newcommand*{\suchthat}{\,\mathrel{\big|}\,}
\newcommand{\Exp}[1]{\exp\mathopen{}\left\{#1\right\}\mathclose{}}
\newcommand{\Log}[1]{\log\mathopen{}\left\{#1\right\}\mathclose{}}
\newcommand{\BigO}{\mathcal{O}}
\newcommand{\Def}{\coloneqq} % coloneqq comes from the mathtools package. 

\DeclareMathOperator*{\argmax}{argmax}
\DeclareMathOperator*{\argmin}{argmin}
\DeclarePairedDelimiterX\innerp[2]{(}{)}{#1\delimsize\vert\mathopen{}#2}
\DeclarePairedDelimiter{\ceil}{\lceil}{\rceil}

% Linear algebra.
\newcommand*{\norm}[1]{\left\lVert#1\right\rVert}
\newcommand{\bx}{\mathbf{x}}
\newcommand{\bw}{\mathbf{w}}
\newcommand{\oneVec}[1][]{\boldsymbol{1}_{#1}}
\newcommand{\idMat}{\mathbf{I}}
\newcommand{\bS}{\mathbf{S}}
\newcommand{\bK}{\mathbf{K}}
\newcommand{\ba}{\mathbf{a}}
\newcommand{\bb}{\mathbf{b}}

% Probability. 
\newcommand{\E}{\mathbb{E}}
\newcommand{\Var}{\mathrm{Var}}
\newcommand{\Cov}{\mathrm{Cov}}
\newcommand{\Prob}{\mathbb{P}}
\newcommand{\Gaussian}{\mathcal{N}}
\newcommand{\LN}{\text{LN}} % Log-normal distribution 
\newcommand{\given}{\mid} 


%!TEX root = SMA-book.tex

% Useful info on newcomand: https://tex.stackexchange.com/questions/117358/newcommand-argument-confusion

% --------------------------------------------------------------------------------------------------
% General 
% --------------------------------------------------------------------------------------------------

\newcommand{\cst}{C}
\newcommand{\eqDist}{\overset{d}{=}} % Equality in distribution 
\newcommand{\diag}{\textrm{diag}}

% --------------------------------------------------------------------------------------------------
% Land Surface Models (LSMs) / Dynamic Models 
% --------------------------------------------------------------------------------------------------

\newcommand{\state}{x}
\newcommand{\Time}{t}
\newcommand{\timeInit}{\Time_0}
\newcommand{\stateInit}{\state_0}
\newcommand{\stateApprox}{\hat{\state}}
\newcommand{\timeEnd}{T}
\newcommand{\timeIdx}{k}
\newcommand{\lastTimeIdx}{K}
\newcommand{\timeStep}{h}
\newcommand{\funcODE}{f}
\newcommand{\forcing}{w}
\newcommand{\indexTime}[2][\timeIdx]{{#2}_{#1}}
\newcommand{\stateTime}[1][\timeIdx]{\indexTime[{#1}]{\state}}
\newcommand{\forcingTime}[1][\timeIdx]{\indexTime[{#1}]{\forcing}}
\newcommand{\fwdOne}{g} % One-step forward operator (flow map) 
\newcommand{\dimState}{P}
\newcommand{\stateIdx}{p}
\newcommand{\indexState}[2][\stateIdx]{{#2}^{(#1)}}



% --------------------------------------------------------------------------------------------------
% Bayesian inverse problem: general setup 
% --------------------------------------------------------------------------------------------------

% General inverse problems.
\newcommand{\Par}{\theta}
\newcommand{\parSpace}{\mathcal{U}} % Parameter space
\newcommand{\dimPar}{D} % Parameter dimension 
\newcommand{\idxParDim}{d} % Primary symbol used to index each parameter dimension. 
\newcommand{\obs}{y} % Data observation (response) vector
\newcommand{\noise}{\epsilon} % Random variable representing noise, typically in additive noise model. 
\newcommand{\obsSpace}{\mathcal{Y}} % Output space
\newcommand{\dimObs}{P} % Dimension of data observation vector
\newcommand{\idxObsDim}{p} % Primary symbol used to index each data dimension. 
\newcommand{\fwd}{G} % Forward model
\newcommand{\llik}{\mathcal{L}} % Log-likelihood 
\newcommand{\priorDens}{\pi_0} % Prior density 
\newcommand{\postDens}{\pi} % Unnormalized posterior density.
\newcommand{\postDensNorm}{\overline{\pi}} % Normalized posterior density.
\newcommand{\normCst}{Z} % Normalizing constant for posterior density. 
\newcommand{\likPar}{\Sigma} % Likelihood parameter


% --------------------------------------------------------------------------------------------------
% Gaussian Process Emulators
% --------------------------------------------------------------------------------------------------

% Generic GP commands. 
\newcommand{\Ndesign}{N}
\newcommand{\GP}{\mathcal{GP}} % GP distribution.
\newcommand{\LNP}{\mathcal{LNP}} % Log-normal process distribution. 
\newcommand{\func}{f} % Generic function used for results that hold for any generic GP. 
\newcommand{\funcPrior}{\func_{0}}
\newcommand{\funcVal}[1][\Ndesign]{\varphi_{#1}}
\newcommand{\funcValExact}[1][\Ndesign]{\varphi_{#1}^{*}}
\newcommand{\gpMeanBase}{\mu} % The base notation used for GP mean (no sub/superscripts). 
\newcommand{\gpKerBase}{k}
\newcommand{\gpMeanPrior}{\gpMeanBase_0}
\newcommand{\gpKerPrior}{\gpKerBase_0}
\newcommand{\gpMean}[1][\Ndesign]{\gpMeanBase_{#1}}
\newcommand{\gpKer}[1][\Ndesign]{\gpKerBase_{#1}}
\newcommand{\nuggetSD}{\eta}
\newcommand{\kerMat}[1][\Ndesign]{K_{#1}}
\newcommand{\emLawPrior}[1]{{\Pi}^{#1}} % The prior law/distribution of an emulator.
\newcommand{\design}[1][\Ndesign]{\mathcal{D}_{#1}}
\newcommand{\designRand}[2][\Nbatch]{\mathcal{D}_{#1}^{#2}} % When the response is unobserved/random. 
\newcommand{\emLaw}[2][\design]{{\Pi}^{#2}_{#1}} % The conditional law/distribution of an emulator (may or may not be Gaussian). 

% Exponentiated quadratic kernel.
\newcommand{\lengthscale}{\ell}
\newcommand{\margSD}{\sigma_{\gpKerBase}}

% Log-likelihood and forward model emulators. 
\newcommand{\fwdPrior}{\fwd_{0}}
\newcommand{\llikPrior}{\llik_{0}}
\newcommand{\funcEm}[1][\Ndesign]{\func_{#1}}
\newcommand{\fwdEm}[1][\Ndesign]{\fwd_{#1}}
\newcommand{\llikEm}[1][\Ndesign]{\llik_{#1}} 
\newcommand{\llikEmFwd}[1][\Ndesign]{\llik^{\fwd}_{#1}}
\newcommand{\Em}[2][\Ndesign]{{#2}_{#1}}

% Mean and kernel functions for log-likelihood and forward model emulators. 
\newcommand{\emMeanPrior}[1]{\gpMeanPrior^{#1}} 
\newcommand{\emKerPrior}[1]{\gpKerPrior^{#1}}
\newcommand{\emMean}[2][\Ndesign]{\gpMeanBase^{#2}_{#1}} % First argument is number of design points, second is the function being emulated. 
\newcommand{\emKer}[2][\Ndesign]{\gpKerBase^{#2}_{#1}}

% Random (sample) approximations. 
\newcommand{\llikEmRdm}[2][\Ndesign]{{#2}_{#1}}
\newcommand{\fwdEmRdm}[2][\Ndesign]{{#2}_{#1}^{\fwd}}
\newcommand{\llikEmRdmDens}[1][\Ndesign]{\llikEmRdm[#1]{\postDens}} % Convenience function 
\newcommand{\fwdEmRdmDens}[1][\Ndesign]{\fwdEmRdm[#1]{\postDens}} % Convenience function 

% Marginal approximations. 
\newcommand{\llikEmMarg}[2][\Ndesign]{{#2}_{#1}^{\text{marg}}}
\newcommand{\fwdEmMarg}[2][\Ndesign]{{#2}_{#1}^{\fwd \text{,marg}}}
\newcommand{\llikEmMargDens}[1][\Ndesign]{\llikEmMarg[#1]{\postDens}} 
\newcommand{\fwdEmMargDens}[1][\Ndesign]{\fwdEmMarg[#1]{\postDens}} 
\newcommand{\CovComb}[1][\Ndesign]{\Em[{#1}]{\likPar}^{\fwd}}

% Mean approximations. 
\newcommand{\llikEmMean}[2][\Ndesign]{{#2}_{#1}^{\text{mean}}}
\newcommand{\fwdEmMean}[2][\Ndesign]{{#2}_{#1}^{\fwd \text{,mean}}}
\newcommand{\llikEmMeanDens}[1][\Ndesign]{\llikEmMean[#1]{\postDens}} 
\newcommand{\fwdEmMeanDens}[1][\Ndesign]{\fwdEmMean[#1]{\postDens}} 

% Quantile approximations. 
\newcommand{\quantileProb}{\alpha}
\newcommand{\GaussianCDF}{\Phi}
\newcommand{\quantile}[1][\quantileProb]{q^{\quantileProb}}
\newcommand{\llikEmQ}[2][\Ndesign]{{#2}_{#1}^{\quantileProb}}
\newcommand{\fwdEmQ}[2][\Ndesign]{{#2}_{#1}^{\fwd, \quantileProb}}
\newcommand{\llikEmQDens}[1][\Ndesign]{\llikEmQ[#1]{\postDens}} 
\newcommand{\fwdEmQDens}[1][\Ndesign]{\fwdEmQ[#1]{\postDens}} 

% Joint marginal approximations. 
\newcommand{\llikEmJointMarg}[2][\Ndesign]{{#2}_{#1}^{\textrm{joint-marg}}}
\newcommand{\fwdEmJointMarg}[2][\Ndesign]{{#2}_{#1}^{\fwd \textrm{,joint-marg}}}

% Joint Monte Carlo within Metropolis-Hastings noisy MCMC approximations. 
\newcommand{\mcwmhJointLabel}{\textrm{mcwmh-joint}}
\newcommand{\mcwmhIndLabel}{\textrm{mcwmh-ind}}
\newcommand{\mhPseudoMargLabel}{\textrm{mh-pseudo-marg}}
\newcommand{\llikEmJointMCWMH}[2][\Ndesign]{{#2}_{#1}^{\mcwmhJointLabel}}
\newcommand{\fwdEmJointMCWMH}[2][\Ndesign]{{#2}_{#1}^{\fwd \textrm{,\mcwmhJointLabel}}}

% Design. 
\newcommand{\Nbatch}{B}
\newcommand{\Nrounds}{K_{\textrm{design}}}
\newcommand{\designIndex}{k}
\newcommand{\Naugment}{\Ndesign + \Nbatch}
\newcommand{\idxDesign}{n} % Primary symbol used to index each design point. 
\newcommand{\parMat}{U} % Generic matrix of input parameter points. 
\newcommand{\designIn}[1][\Ndesign]{U_{#1}} % Design inputs. 
\newcommand{\designOutLlik}[1][\Ndesign]{\ell_{#1}}
\newcommand{\designOutFwd}[1][\Ndesign]{g_{#1}}
\newcommand{\designOutLlikExact}[1][\Ndesign]{\ell^*_{#1}}
\newcommand{\designOutFwdExact}[1][\Ndesign]{g^*_{#1}}
\newcommand{\designBatchIn}{U_{\Nbatch}}
\newcommand{\designBatchLlik}{\ell_{\Nbatch}}
\newcommand{\designBatchFunc}{\varphi_{\Nbatch}}
\newcommand{\designBatchFwd}{g_{\Nbatch}}
\newcommand{\fwdEmCond}[2][\Ndesign]{\fwdEm[{#1}]^{#2}}
\newcommand{\acq}[1][\Ndesign]{\mathcal{A}_{#1}}

% Acquisition functions 
\newcommand{\labelAcq}[2][\Ndesign]{\acq[#1]^{\textrm{#2}}}
\newcommand{\maxvarLabel}{\textrm{max-var}}
\newcommand{\maxentLabel}{\textrm{max-ent}}
\newcommand{\intvarLabel}{\textrm{int-var}}
\newcommand{\intentLabel}{\textrm{int-ent}}
\newcommand{\maxexpvarLabel}{\textrm{max-exp-var}}
\newcommand{\maxexpentLabel}{\textrm{max-exp-ent}}
\newcommand{\intExpVarLabel}{\textrm{int-exp-var}}
\newcommand{\fwdintExpVarLabel}{\fwd \textrm{,}\intExpVarLabel}
\newcommand{\intExpEntLabel}{\textrm{int-exp-ent}}
\newcommand{\Ent}{H} % Entropy 
\newcommand{\weightDens}{\rho}
\newcommand{\varInflation}{v}

% Basis function emulation. 
\newcommand{\basisVec}{\phi}
\newcommand{\basisWeight}{w}
\newcommand{\dimBasis}{R}
\newcommand{\idxBasis}{r}

% --------------------------------------------------------------------------------------------------
% Markov Chain Monte Carlo. 
% --------------------------------------------------------------------------------------------------

% Generic MCMC commands. 
\newcommand{\propDens}{q} % Proposal density
\newcommand{\propPar}{\tilde{\Par}} % Proposal density
\newcommand{\llikSamp}{\ell} 
\newcommand{\llikSampProp}{\tilde{\ell}} 
\newcommand{\llikSampDist}{\nu}
\newcommand{\CovProp}{\mathbf{C}} % Proposal covariance matrix
\newcommand{\NMCMC}{K_{\text{MCMC}}} % Proposal covariance matrix
\newcommand{\accProbMH}{\alpha} % Metropolis-Hastings acceptance probability
\newcommand{\avgAccProbMH}{\overline{a}}
\newcommand{\accProbRatio}{r}
\newcommand{\likRatio}{\lambda}
\newcommand{\MarkovKernel}{P}
\newcommand{\mcmcIndex}{k}
\newcommand{\indexMCMC}[2][\mcmcIndex]{{#2}_{#1}}






\newcommand{\bphi}{\boldsymbol{\phi}}

\setlist{topsep=1ex,parsep=1ex,itemsep=0ex}
\setlist[1]{leftmargin=\parindent}
\setlist[enumerate,1]{label=\arabic*.,ref=\arabic*}
\setlist[enumerate,2]{label=(\alph*),ref=(\alph*)}

% For embedding images
\graphicspath{{../output/gp_post_approx_paper/}}

% Specifically for paper formatting 
\renewcommand{\baselinestretch}{1.2} % Spaces manuscript for easy reading

% Formatting definitions, propositions, etc. 
\newtheorem{definition}{Definition}
\newtheorem{prop}{Proposition}
\newtheorem{lemma}{Lemma}
\newtheorem{thm}{Theorem}
\newtheorem{corollary}{Corollary}
\newtheorem{example}{Example}

% Title and author
\title{Parameter Calibration for Land Surface Models}
\author{Andrew Roberts}

\begin{document}
\maketitle

% The Structure of Land Surface Models
\section{The Structure of Land Surface Models}
Unlike the typical PDE models used in atmospheric and oceanic modeling, the standard toolkit in land surface modeling 
consists primarily of ODEs describing time evolution at a single location. In this introductory section, we omit discussion 
of the application of these models across space, but return to this important topic in a later section. The notation 
used throughout this section can thus be thought of as suppressing a fixed spatial index on most of the quantities considered. 

While LSMs can incorporate a wide 
variety of terrestrial processes, we will focus on models of the carbon cycle when providing concrete examples. 
LSMs model the carbon cycle as consisting of a set of states (i.e., \textit{pools}; soil, roots, above-ground vegetation, atmosphere, etc.),
along with processes that transfer carbon between these states (i.e., \textit{fluxes}; photosynthesis, respiration, etc.). 
We denote the state vector at time $\Time$ by 
\begin{align}
\state(\Time) \Def [\indexState[1]{\state}(\Time), \dots, \indexState[\dimState]{\state}(\Time)]^\top \in \R^{\dimState}.
\end{align}
Carbon fluxes between these states are then described by an ODE model of the form 
\begin{align}
&\frac{d}{d\Time} \state_{\Par}(\Time) = \funcODE_{\Par}(\state_{\Par}(\Time), \forcing(\Time)), &&\state_{\Par}(\timeInit) = \stateInit, 
\end{align}
where $\funcODE_{\Par}(\cdot, \forcing): \R^{\dimState} \to \R^{\dimState}$ is typically a nonlinear function, and $\stateInit$ is the \textit{initial condition}, 
the value of the state at some initial time $\timeInit$. The forward dynamics encoded by $\funcODE_{\Par}$ are dependent both on a set of 
\textit{parameters} $\Par \in \parSpace \subset \R^{\dimPar}$, as well as a \textit{forcing} (or \textit{driving}) function $\forcing(\Time)$.
In general, the parameters $\Par$ are not physical constants; rather, they provide a means to empirically parameterize ecosystem 
processes, which may vary in space and time.   
We emphasize that the solution $\state_{\Par}(\Time)$ of the ODE is a function of the parameters $\Par$, driver $\forcing(\Time)$, and 
initial condition $\stateInit$, though only the first is made explicit in the notation as the parameters are the main quantity of interest in this document.

% The Forward Problem 
\subsection{The Forward Problem}
The \textit{forward problem} consists of recovering the solution $\state_{\Par}(\Time)$ over some time interval $[\timeInit, \timeEnd]$ for given 
values of $\Par$, $\stateInit$, and $\forcing(\cdot)$. Given the nonlinearity of $\funcODE_{\Par}$, this solution is typically approximated 
numerically at a finite set of times $\{\indexTime{\Time}\}_{\timeIdx=0}^{\Ntime-1} \subset [\timeInit, \timeEnd]$. We 
let $\stateApprox_{\Par}(\indexTime{\Time})$ denote the approximation of $\state_{\Par}(\indexTime{\Time})$ for each 
$\timeIdx = 0, \dots, \Ntime-1$. Also, let $\indexTime{\timeStep} \Def \indexTime[\timeIdx+1]{\Time} - \indexTime{\Time}$ denote the 
time step used by the numerical solver at the $\timeIdx^{\text{th}}$ iteration. Standard ODE solvers are time-stepping algorithms of the 
form 
\begin{align}
&\stateApprox_{\Par}(\indexTime[\timeIdx+1]{\Time}) \Def \fwdOne_{\Par}(\stateApprox_{\Par}(\indexTime{\Time}), \forcing(\indexTime{\Time}); \indexTime{\timeStep}), 
&&\stateApprox_{\Par}(\indexTime[0]{\Time}) = \stateInit, \label{ode_discrete}
\end{align}
for $\timeIdx = 1, \dots, \Ntime-1$. Setting a constant step size $\timeStep \equiv \indexTime{\timeStep}$ is a common choice, though 
algorithms that adapt the step size are also quite standard. In practice, the model drivers $\forcing(\indexTime{\Time})$
are typically given by a sequence of (noisy) observations of meteorological variables (e.g., photosynthetically-active radiation).
Going forward, we will treat the discrete dynamical system \ref{ode_discrete} as 
the true model and starting point for subsequent analysis, thus neglecting discretization error with respect to the true continuous-time solution. 

% The Inverse Problem
\subsection{The Inverse Problem}
Naturally, one must recognize that the predictions of the model \ref{ode_discrete} are subject to many sources of uncertainty. 
The simulated trajectory $\{\stateApprox_{\Par}(\indexTime{\Time})\}_{\timeIdx=1}^{\Ntime-1}$ is a function of  
the chosen value of the parameters $\Par$, the model drivers  $\{\indexTime{\obs}\}_{\timeIdx=1}^{\Ntime-1}$, the initial condition $\stateInit$, 
and the model processes encoded by $\fwdOne_{\Par}$ (irrespective of the parameter value). In the LSM context, 
all of these quantities contribute non-negligible uncertainty. 
In this document, we restrict ourselves to the consideration of parameter uncertainty. 

The values of LSM parameters are not known exactly, and moreover, cannot always be observed directly. In other words, offline estimation of $\Par$
using a separate dataset is typically not possible for all of the parameters of interest. Historically, the choice of parameter settings 
was commonly performed on an ad hoc basis using expert judgement. 
Given that $\Par$ consists of empirical parameters 
(not physical constants) the concept of ``true'' parameter values is largely ill-defined. It is perhaps most conceptually useful to draw an 
analogy with empirically-determined parameters of a statistical model, in which ``true values'' is interpreted as best-fit in a regression sense.
Therefore, a natural solution is to learn the parameter values from data. For example, suppose we have access to noisy observations 
$\{\indexTime{\obs}\}_{\timeIdx=0}^{\Ntime-1}$ of the true states $\{\state_{\Par}(\indexTime{\Time})\}_{\timeIdx=0}^{\Ntime-1}$. 
We can then consider tuning (i.e., \textit{calibrating}), the value of $\Par$ such that $\{\state_{\Par}(\indexTime{\Time})\}_{\timeIdx=0}^{\Ntime-1}$
is ``close'' to $\{\indexTime{\obs}\}_{\timeIdx=0}^{\Ntime-1}$  in some well-defined sense. We will refer to the 
general problem of learning parameters from data as \textit{parameter calibration} or \textit{parameter estimation}. We make this 
problem precise in the following section, which casts parameter calibration within the generic framework of \textit{inverse problems}. 

% Inverse problems 
\section{Inverse Problems}
In this section, we provide a brief introduction to inverse problems from a generic perspective. The following section 
interprets these ideas in the context of parameter calibration for LSMs. 

\subsection{Basic Setup}
We start by considering a \textit{forward model} $\fwd: \parSpace \subseteq \R^{\dimPar} \to \R^{\dimObs}$ describing some 
process of interest, parameterized by input parameters $\Par \in \parSpace$. In addition, suppose we have noisy 
observations $\obs \in \obsSpace \subset \R^{\dimObs}$ of the output signal that $\fwd(\Par)$ approximates. 
We seek to invert the process to infer the parameter values that produced the data; loosely speaking, the 
\textit{inverse problem} consists of identifying $\Par$ such that $\fwd(\Par) \approx \obs$. 

\begin{example}
There are many ways we might define the forward model in the dynamical setting \ref{ode_discrete}, which we recall 
conceptualizes an LSM at a single spatial location. Suppose we have noisy observations 
$\obs \Def \{\indexTime{\obs}\}_{\timeIdx=0}^{\Ntime-1}$ 
of the true states $\{\indexState[1]{\state}_{\Par}(\indexTime{\Time})\}_{\timeIdx=0}^{\Ntime-1}$; that is, we have observations of 
the first state variable (e.g., above-ground biomass). It is then natural to define the forward model as 
\begin{align}
&\fwd: \parSpace \to \R^{\Ntime}, && \fwd(\Par) 
\Def [\indexState[1]{\stateApprox}_{\Par}(\indexTime[0]{\Time}), \dots, \indexState[1]{\stateApprox}_{\Par}(\indexTime[\Ntime-1]{\Time})]^\top. \label{fwd_model}
\end{align}
In words, $\fwd(\Par)$ is the simulated trajectory of the first state variable given the parameter value $\Par$. In this context, achieving 
the approximation $\fwd(\Par) \approx \obs$ implies that this simulated trajectory agrees with the noisy state observations. We also 
see that $\dimObs = \Ntime$; i.e., the output dimension of the forward model equals the number of time steps. 
\end{example}

We emphasize that this setup is quite general; in particular, the forward model $\fwd$ can conceptualize a wide array 
of different problems. However, in the present setting, $\fwd$ typically has a variety of characteristics that make 
the inverse problem especially challenging. We will emphasize forward models that (1) are expensive to evaluate 
at inputs $\Par$; (2) have high-dimensional, structured output spaces (e.g., time series); and (3) are highly nonlinear. 
In addition, computing gradients of $\fwd$ may be difficult or impossible in many cases. Given these characteristics, 
it is desirable to consider algorithms that treat $\fwd$ as a ``black-box''; that is, algorithms that only require the 
ability to compute evaluations $\fwd(\Par)$, without assuming any additional structure. However, there are certainly 
cases where additional structure (e.g., derivative information, smoothness, periodicity) can be exploited. 

\subsection{Loss Minimization}
It is typically impossible (or undesirable) to seek an exact solution $\fwd(\Par) = \obs$. An alternative is to cast 
model inversion as an optimization problem of the form 
\begin{align}
\parEst \Def \text{argmin}_{\Par \in \parSpace} \loss(\Par), \label{opt}
\end{align}
where $\loss(\Par)$ is a loss function, providing some notion of discrepancy between $\fwd(\Par)$ and $\obs$.
A common choice is the Euclidean distance (i.e., mean squared error),
\begin{align}
\loss(\Par) = \norm{\obs - \fwd(\Par)}^2_2 \Def \sum_{\obsIdx=1}^{\dimObs} (\obs_{\obsIdx} - \fwd_{\obsIdx}(\Par))^2. \label{l2_loss}
\end{align}
This quadratic loss can be extended to weighted generalizations via 
\begin{align}
\loss(\Par) = \norm{\obs - \fwd(\Par)}^2_{\obsCov} \Def (\obs - \fwd(\Par))^\top \obsCov^{-1} (\obs - \fwd(\Par)), \label{l2_loss_weighted}
\end{align}
for some positive-definite matrix $\obsCov \in \R^{\dimObs \times \dimObs}$. Note that \ref{l2_loss_weighted} reduces to 
\ref{l2_loss} by setting $\obsCov$ to the identity matrix. 

In theory, any standard optimization algorithms may be employed to solve this optimization problem. However, 
the typical black-box structure of $\fwd$ poses a challenge. If gradients are available, then gradient descent or 
(quasi)-Newton methods are a possibility. However, evaluations of $\fwd(\Par)$ may be prohibitively expensive, 
requiring specialized optimization routines designed to minimize the required number of model evaluations. 
So-called \textit{black box}, or \textit{derivative free}, optimization algorithms are designed for the setting 
in which gradient evaluations are not available. 

Another issue to contend with is the potential for $\loss(\Par)$ to have many local minima. The phenomenon in which 
multiple values of $\Par$ explain the observed data equally well is generally referred to as \textit{non-identifiability}, 
or \textit{equifinality}. One approach to mitigate non-identifiability issues is to incorporate prior information regarding 
the value of $\Par$. This can be accomplished via the addition of a \textit{regularization} term $\regularizer(\Par)$ in 
the objective function, yielding 
\begin{align}
\parEst \Def \text{argmin}_{\Par \in \parSpace} \left[\loss(\Par) + \regularizer(\Par) \right]. \label{opt_reg}
\end{align}
A common choice of regularizer is $\regularizer(\Par) = \frac{1}{c^2} \norm{\Par - \priorMean}_2^2$, with $c > 0$ tuning 
the strength of the regularization. As before, this can be generalized to 
$\regularizer(\Par) = \norm{\Par - \priorMean}_{\priorCov}^2 = (\Par - \priorMean)^\top \priorCov^{-1} (\Par - \priorMean)$, 
with $\priorCov$ another positive definite matrix. 
Choosing both a quadratic loss and quadratic regularization yields the common formulation
\begin{align}
\parEst \Def \text{argmin}_{\Par \in \parSpace} \left\{\norm{\obs - \fwd(\Par)}^2_{\obsCov} + \norm{\Par - \priorMean}_{\priorCov}^2\right\}. \label{quadratic_opt}
\end{align}
The first term quantifies the model fit, while the second encourages agreement between $\Par$ and $\priorMean$.

\subsection{Maximum Likelihood}
We now consider a frequentist statistical approach to the solution of inverse problems. Instead of defining a loss function, the 
starting point is now to consider a \textit{likelihood function} $p(\obs | \Par)$ (viewed as a function of $\Par$, with $\obs$ fixed). 
For example, we might assume that the observations $\obs$ are given by Gaussian perturbations of the underlying signal 
$\fwd(\Par)$, for some ``true'' value of $\theta$: 
\begin{align}
\obs | \Par &\sim \Gaussian(\priorMean, \priorCov). \label{gaussian-lik}
\end{align}
In this case, the likelihood function is then given by $p(\obs | \Par) = \Gaussian(\obs | \fwd(\Par), \obsCov)$, with 
$\Gaussian(\obs | \fwd(\Par), \obsCov)$ denoting the density of $\Gaussian(\fwd(\Par), \obsCov)$ evaluated 
at $\obs$. As was the case with the loss function, evaluating the likelihood at $\fwd$ requires the forward 
model evaluation $\fwd(\Par)$. We can then define the solution to the inverse problem as the value 
of $\Par$ that maximizes the likelihood of the data; that is, 
\begin{align}
\parEst \Def \text{argmax}_{\Par \in \parSpace} \ p(\obs | \Par). \label{mle}
\end{align}
Maximizing $p(\obs | \Par)$ is equivalent to minimizing $\loss(\Par) \Def -\log p(\obs | \Par)$, so we can view the 
negative log-likelihood as defining a loss function, thus drawing a connection with the non-statistical optimization 
framework discussed above. Indeed, if we consider the Gaussian observation model \ref{gaussian-lik}, then we 
have 
\begin{align}
\loss(\Par) = -\log p(\obs | \Par) = \frac{1}{2} \norm{\obs - \fwd(\Par)}_{\obsCov}^2, 
\end{align}
which implies that the MLE solution in the Gaussian noise setting is equal to minimizing the minimizer of the loss function 
\ref{l2_loss_weighted}. 



\subsection{Bayesian Methods}

\subsection{Connections with State Estimation}

% Parameter calibration 
\section{Parameter Calibration for LSMs}
\subsection{Parameter Structure}
PFTs, identifiability/equifinality, non-linear dependence 

\subsection{Parameter Dimension Reduction}
Parameter fixing, potential for more sophisticated methods, scaling factors for PFTs

\subsection{Spatial and Temporal Variability}

\subsection{Multi-Objective Calibration}

% Surrogate Modeling 
\section{Surrogate Modeling}


\end{document} 




