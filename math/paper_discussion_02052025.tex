\documentclass[12pt]{article}
\RequirePackage[l2tabu, orthodox]{nag}
\usepackage[main=english]{babel}
\usepackage[rm={lining,tabular},sf={lining,tabular},tt={lining,tabular,monowidth}]{cfr-lm}
\usepackage[T1]{fontenc}
\usepackage[utf8]{inputenc}
\usepackage[pdftex]{graphicx}
\usepackage{amsthm,amssymb,latexsym,gensymb,mathtools,mathrsfs}
\usepackage{epstopdf,enumitem,microtype,dcolumn,booktabs,hyperref,url,fancyhdr}
\usepackage{algorithm}
\usepackage{algpseudocode} % Note that this also loads algorithmicx
\usepackage{cleveref}
\usepackage{natbib}
\usepackage{bbm}
\usepackage{caption, subcaption} % Captions and sub-figures. 
\usepackage{fancyvrb} % For writing using verbatim font inline. 
% \usepackage[demo]{graphicx}

% Bibliography
\bibliographystyle{plainnat}

% Plotting
\usepackage{pgfplots}
\usepackage{xinttools} % for the \xintFor***
\usepgfplotslibrary{fillbetween}
\pgfplotsset{compat=1.8}
\usepackage{tikz}

% Tables. 
\usepackage{multirow}



% Local custom commands. 
%!TEX root = SMA-book.tex

% Useful info on newcomand: https://tex.stackexchange.com/questions/117358/newcommand-argument-confusion

% --------------------------------------------------------------------------------------------------
% General commands.  
% --------------------------------------------------------------------------------------------------

\newcommand{\todo}{\textbf{TODO}}


% --------------------------------------------------------------------------------------------------
% General Math.  
% --------------------------------------------------------------------------------------------------

% Common math commands. 
\newcommand*{\abs}[1]{\left\lvert#1\right\rvert}
\newcommand{\R}{\mathbb{R}}
\newcommand*{\suchthat}{\,\mathrel{\big|}\,}
\newcommand{\Exp}[1]{\exp\mathopen{}\left\{#1\right\}\mathclose{}}
\newcommand{\Log}[1]{\log\mathopen{}\left\{#1\right\}\mathclose{}}
\newcommand{\BigO}{\mathcal{O}}

\DeclareMathOperator*{\argmax}{argmax}
\DeclareMathOperator*{\argmin}{argmin}
\DeclarePairedDelimiterX\innerp[2]{(}{)}{#1\delimsize\vert\mathopen{}#2}
\DeclarePairedDelimiter{\ceil}{\lceil}{\rceil}

% Linear algebra.
\newcommand*{\norm}[1]{\left\lVert#1\right\rVert}
\newcommand{\bx}{\mathbf{x}}
\newcommand{\bw}{\mathbf{w}}
\newcommand{\oneVec}[1][]{\boldsymbol{1}_{#1}}
\newcommand{\idMat}{\mathbf{I}}
\newcommand{\bS}{\mathbf{S}}
\newcommand{\bK}{\mathbf{K}}
\newcommand{\ba}{\mathbf{a}}
\newcommand{\bb}{\mathbf{b}}

% Probability. 
\newcommand{\E}{\mathbb{E}}
\newcommand{\Var}{\mathrm{Var}}
\newcommand{\Cov}{\mathrm{Cov}}
\newcommand{\Prob}{\mathbb{P}}
\newcommand{\Gaussian}{\mathcal{N}}
\newcommand{\LN}{\text{LN}} % Log-normal distribution 


%!TEX root = SMA-book.tex

% Useful info on newcomand: https://tex.stackexchange.com/questions/117358/newcommand-argument-confusion


% --------------------------------------------------------------------------------------------------
% Bayesian inverse problem: general setup 
% --------------------------------------------------------------------------------------------------

% General inverse problems.
\newcommand{\Par}{u}
\newcommand{\parSpace}{\mathcal{U}} % Parameter space
\newcommand{\dimPar}{D} % Parameter dimension 
\newcommand{\idxParDim}{d} % Primary symbol used to index each parameter dimension. 
\newcommand{\obs}{y} % Data observation (response) vector
\newcommand{\noise}{\epsilon} % Random variable representing noise, typically in additive noise model. 
\newcommand{\obsSpace}{\mathcal{Y}} % Output space
\newcommand{\dimObs}{P} % Dimension of data observation vector
\newcommand{\idxObsDim}{p} % Primary symbol used to index each data dimension. 
\newcommand{\fwd}{G} % Forward model
\newcommand{\llik}{\mathcal{L}} % Log-likelihood 
\newcommand{\priorDens}{\pi_0} % Prior density 
\newcommand{\postDens}{\pi} % Unnormalized posterior density.
\newcommand{\postDensNorm}{\pi_{\text{norm}}} % Normalized posterior density.
\newcommand{\normCst}{Z} % Normalizing constant for posterior density. 
\newcommand{\likPar}{\Sigma} % Likelihood parameter


% --------------------------------------------------------------------------------------------------
% Gaussian Process Emulators
% --------------------------------------------------------------------------------------------------

% Generic GP commands. 
\newcommand{\GP}{\mathcal{GP}} % GP distribution.
\newcommand{\LNP}{\mathcal{LNP}} % Log-normal process distribution. 
\newcommand{\func}{f} % Generic function used for results that hold for any generic GP. 
\newcommand{\gpMean}[1][]{\mu_{#1}}
\newcommand{\gpKer}[1][]{k_{#1}}
\newcommand{\nuggetSD}{\eta}

% Exponentiated quadratic kernel.
\newcommand{\lengthscale}{\ell}
\newcommand{\margSD}{\sigma_{\gpKer}}

% Log-likelihood and forward model emulators. 
\newcommand{\funcEm}[1][]{\func_{#1}}
\newcommand{\llikEm}[2][]{#1_{#2}}
\newcommand{\emMean}[2][]{\mu^{#1}_{#2}}
\newcommand{\emKer}[2][]{k^{#1}_{#2}}

% Design. 
\newcommand{\Ndesign}{N}
\newcommand{\NBatch}{B}
\newcommand{\idxDesign}{n} % Primary symbol used to index each design point. 
\newcommand{\parMat}{U} % Generic matrix of input parameter points. 
\newcommand{\designIn}[1][\Ndesign]{U_{#1}} % Design inputs. 
\newcommand{\designOutLlik}[1][\Ndesign]{\ell_{#1}}
\newcommand{\designOutFwd}[1][\Ndesign]{g_{#1}}
\newcommand{\designBatchIn}{U_{\NBatch}}
\newcommand{\designBatchLlik}{\ell_{\NBatch}}
\newcommand{\designBatchFwd}{g_{\NBatch}}

% --------------------------------------------------------------------------------------------------
% Markov Chain Monte Carlo. 
% --------------------------------------------------------------------------------------------------

% Generic MCMC commands. 
\newcommand{\propDens}{q} % Proposal density
\newcommand{\CovProp}{\mathbf{C}} % Proposal covariance matrix
\newcommand{\NMCMC}{T_{\text{MCMC}}} % Proposal covariance matrix
\newcommand{\accProbMH}{\alpha} % Metropolis-Hastings acceptance probability


\newcommand{\bphi}{\boldsymbol{\phi}}

\setlist{topsep=1ex,parsep=1ex,itemsep=0ex}
\setlist[1]{leftmargin=\parindent}
\setlist[enumerate,1]{label=\arabic*.,ref=\arabic*}
\setlist[enumerate,2]{label=(\alph*),ref=(\alph*)}

% For embedding images
\graphicspath{{../output/gp_post_approx_paper/}}

% Specifically for paper formatting 
\renewcommand{\baselinestretch}{1.2} % Spaces manuscript for easy reading

% Formatting definitions, propositions, etc. 
\newtheorem{definition}{Definition}
\newtheorem{prop}{Proposition}
\newtheorem{lemma}{Lemma}
\newtheorem{thm}{Theorem}
\newtheorem{corollary}{Corollary}

% Title and author
\title{Meeting 2/5/2025: some discussion related to paper}
\author{Andrew Roberts}

\begin{document}

\section{Understanding Implications of Emulator Uncertainty Propagation}
A key quantity to investigate is the likelihood ratio
\begin{align}
\likRatio(\Par, \propPar) &= \frac{\Exp{\llik(\propPar)}}{\Exp{\llik(\Par)}}.
\end{align}
Directly comparing two different approximations to the posterior density is difficult, as the normalizing 
constants will differ across different approximations. To avoid these complications, a reasonable quantity
to look at is the ratio of likelihood ratios.
\begin{align}
\likRatio^\prime(\Par, \propPar) &\Def \frac{\likRatio_1(\Par, \propPar)}{\likRatio_2(\Par, \propPar)}  
= \frac{\frac{\Exp{\llik_1(\propPar)}}{\Exp{\llik_1(\Par)}}}{\frac{\Exp{\llik_2(\propPar)}}{\Exp{\llik_2(\Par)}}}, \label{ratio_of_ratios}
\end{align}
where $\llik_1$ and $\llik_2$ are two different (deterministic) approximations to the log-likelihood. As the baseline approximation, I will 
consider $\llik_2$ as the plug-in mean approximation $\llik_2(\Par) \Def \llik^{\text{mean}}(\Par) := \gpMean(\Par)$ 
(assuming a log-likelihood emulator) so that
\begin{align}
\likRatio_2(\Par, \propPar) &= \Exp{\gpMean(\propPar) - \gpMean(\Par)}.
\end{align}
In this case, the ratio-of-ratios \ref{ratio_of_ratios} will assess how different approximations change the relative weighting between 
two parameter values relative to the plug-in mean method.

For the numerator, consider an approximation of the form
\begin{align}
\Exp{\llik_2(\Par)} = \Exp{\gpMean(\Par) + g(\sqrt{\gpKer(\Par)})},
\end{align}
where $g$ is some nondecreasing function of the predictive standard deviation. In this case, the ratio-of-ratios reduces to
\begin{align}
\likRatio^\prime(\Par, \propPar) &= \Exp{g(\sqrt{\gpKer(\propPar)}) - g(\sqrt{\gpKer(\Par)})}.
\end{align}
If the GP uncertainty is the same at the two points, this means the approximate likelihood ratio agrees with the plug-in mean 
approximate ratio. If the uncertainty is larger at $\propPar$ than $\Par$, then the approximation places more weight on 
$\propPar$, relative to the plug-in mean approximation. To compare the influence of the GP uncertainty, suppose the predictive
standard deviations at the two points are related as
\begin{align}
\sqrt{\gpKer(\propPar)} = \beta \sqrt{\gpKer(\Par)}.
\end{align}

\subsection{Pointwise Approximations}

\noindent
\textbf{Marginal Approximation.} The marginal approximation corresponds to the choice $g(t) = \frac{1}{\sqrt{2}} t^2$. Thus, we have
\begin{align}
\likRatio^\prime(\Par, \propPar) 
&= \Exp{\frac{1}{2}[\gpKer(\propPar) - \gpKer(\Par)]} = \Exp{\frac{1}{2}\gpKer(\Par)(\beta^2 - 1)} = \BigO(e^{\beta^2}).
\end{align}
This growth can be quite extreme. Supposing, $\gpKer(\Par) = 1$, then $\likRatio^\prime(\Par, \propPar) \approx 20$ if 
$\beta=2$; i.e., the uncertainty is two times larger at $\propPar$ relative to $\Par$. As $\beta$ grows, we quickly reach a point where
the GP predictive mean plays almost no role, and the likelihood approximation is completely dominated by the GP uncertainty.
This growth becomes more extreme when $\gpKer(\Par) > 1$. For example, if $\gpKer(\Par) = 20$ and $\beta = 1.05$ (uncertainty
is five percent larger at $\propPar$) then already $\likRatio^\prime(\Par, \propPar) \approx 10$. If $\beta = 2$ then 
$\likRatio^\prime(\Par, \propPar)$ is a massive number.

\vspace*{10px}

\noindent
\textbf{Quantile Approximation.} The $\alpha$-quantile approximation corresponds to the choice $g(t) = \Phi^{-1}(\alpha) t$, where 
$\Phi^{-1}$ is the standard normal quantile function. Thus,
\begin{align}
\likRatio^\prime(\Par, \propPar) 
&= \Exp{\Phi^{-1}(\alpha)[\sqrt{\gpKer(\propPar)} - \sqrt{\gpKer(\Par)}]} = \Exp{\Phi^{-1}(\alpha)\sqrt{\gpKer(\Par)}(\beta - 1)} = \BigO(e^{\beta}).
\end{align}

For example, let's take $\alpha=0.7$. If $\gpKer(\Par) = 1$, then $\beta \approx 2$ implies $\likRatio^\prime(\Par, \propPar)  \approx 2$. It 
still grows very quickly; for $\beta \approx 6.7$, $\likRatio^\prime(\Par, \propPar)  \approx 20$. For $\gpKer(\Par) > 1$ the growth quickly 
becomes more extreme.

\subsection{Ratio Approximations}
We now consider methods that directly approximate the likelihood ratio, rather than approximating the density and then taking the ratio 
of the density approximations. 

\vspace*{10px}

\noindent
\textbf{Joint Marginal.} We start by considering the approximation
\begin{align}
\likRatio_1(\Par, \propPar) &= \E\Exp{\llikEm(\propPar) - \llikEm(\Par)}
= \Exp{\gpMean(\propPar) - \gpMean(\Par)}\Exp{\frac{1}{2}[\gpKer(\propPar) + \gpKer(\Par) - 2\gpKer(\Par, \propPar)]},
\end{align} 
so that
\begin{align}
\likRatio^\prime(\Par, \propPar) &= \Exp{\frac{1}{2}[\gpKer(\propPar) + \gpKer(\Par) - 2\gpKer(\Par, \propPar)]}
= \Exp{\frac{1}{2}\gpKer(\Par)[\beta^2 + 1] - \gpKer(\Par, \propPar)}
\end{align}


\end{document}















