\documentclass[12pt]{article}
\RequirePackage[l2tabu, orthodox]{nag}
\usepackage[main=english]{babel}
\usepackage[rm={lining,tabular},sf={lining,tabular},tt={lining,tabular,monowidth}]{cfr-lm}
\usepackage[T1]{fontenc}
\usepackage[utf8]{inputenc}
\usepackage[pdftex]{graphicx}
\usepackage{amsthm,amssymb,latexsym,gensymb,mathtools,mathrsfs}
\usepackage{epstopdf,enumitem,microtype,dcolumn,booktabs,hyperref,url,fancyhdr}
\usepackage{algorithm}
\usepackage{algpseudocode} % Note that this also loads algorithmicx
\usepackage{cleveref}
\usepackage{natbib}
\usepackage{bbm}
\usepackage{caption, subcaption} % Captions and sub-figures. 
\usepackage{fancyvrb} % For writing using verbatim font inline. 
% \usepackage[demo]{graphicx}

% Bibliography
\bibliographystyle{plainnat}

% Plotting
\usepackage{pgfplots}
\usepackage{xinttools} % for the \xintFor***
\usepgfplotslibrary{fillbetween}
\pgfplotsset{compat=1.8}
\usepackage{tikz}

% Local custom commands. 
%!TEX root = SMA-book.tex

% Useful info on newcomand: https://tex.stackexchange.com/questions/117358/newcommand-argument-confusion


% --------------------------------------------------------------------------------------------------
% General commands.  
% --------------------------------------------------------------------------------------------------

\newcommand{\todo}{\textbf{TODO}}


% --------------------------------------------------------------------------------------------------
% General Math.  
% --------------------------------------------------------------------------------------------------

% Common math commands. 
\newcommand*{\abs}[1]{\left\lvert#1\right\rvert}
\newcommand{\R}{\mathbb{R}}
\newcommand*{\suchthat}{\,\mathrel{\big|}\,}
\newcommand{\Exp}[1]{\exp\mathopen{}\left\{#1\right\}\mathclose{}}
\newcommand{\Log}[1]{\log\mathopen{}\left\{#1\right\}\mathclose{}}
\newcommand{\BigO}{\mathcal{O}}
\newcommand{\Def}{\coloneqq} % coloneqq comes from the mathtools package. 

\DeclareMathOperator*{\argmax}{argmax}
\DeclareMathOperator*{\argmin}{argmin}
\DeclarePairedDelimiterX\innerp[2]{(}{)}{#1\delimsize\vert\mathopen{}#2}
\DeclarePairedDelimiter{\ceil}{\lceil}{\rceil}

% Linear algebra.
\newcommand*{\norm}[1]{\left\lVert#1\right\rVert}
\newcommand{\bx}{\mathbf{x}}
\newcommand{\bw}{\mathbf{w}}
\newcommand{\oneVec}[1][]{\boldsymbol{1}_{#1}}
\newcommand{\idMat}{\mathbf{I}}
\newcommand{\bS}{\mathbf{S}}
\newcommand{\bK}{\mathbf{K}}
\newcommand{\ba}{\mathbf{a}}
\newcommand{\bb}{\mathbf{b}}

% Probability. 
\newcommand{\E}{\mathbb{E}}
\newcommand{\Var}{\mathrm{Var}}
\newcommand{\Cov}{\mathrm{Cov}}
\newcommand{\Prob}{\mathbb{P}}
\newcommand{\Gaussian}{\mathcal{N}}
\newcommand{\LN}{\text{LN}} % Log-normal distribution 
\newcommand{\given}{\mid} 


%!TEX root = SMA-book.tex

% Useful info on newcomand: https://tex.stackexchange.com/questions/117358/newcommand-argument-confusion

% --------------------------------------------------------------------------------------------------
% General 
% --------------------------------------------------------------------------------------------------

\newcommand{\cst}{C}
\newcommand{\eqDist}{\overset{d}{=}} % Equality in distribution 
\newcommand{\diag}{\textrm{diag}}


% --------------------------------------------------------------------------------------------------
% Bayesian inverse problem: general setup 
% --------------------------------------------------------------------------------------------------

% General inverse problems.
\newcommand{\Par}{u}
\newcommand{\parSpace}{\mathcal{U}} % Parameter space
\newcommand{\dimPar}{D} % Parameter dimension 
\newcommand{\idxParDim}{d} % Primary symbol used to index each parameter dimension. 
\newcommand{\obs}{y} % Data observation (response) vector
\newcommand{\noise}{\epsilon} % Random variable representing noise, typically in additive noise model. 
\newcommand{\obsSpace}{\mathcal{Y}} % Output space
\newcommand{\dimObs}{P} % Dimension of data observation vector
\newcommand{\idxObsDim}{p} % Primary symbol used to index each data dimension. 
\newcommand{\fwd}{G} % Forward model
\newcommand{\llik}{\mathcal{L}} % Log-likelihood 
\newcommand{\priorDens}{\pi_0} % Prior density 
\newcommand{\postDens}{\pi} % Unnormalized posterior density.
\newcommand{\postDensNorm}{\overline{\pi}} % Normalized posterior density.
\newcommand{\normCst}{Z} % Normalizing constant for posterior density. 
\newcommand{\likPar}{\Sigma} % Likelihood parameter


% --------------------------------------------------------------------------------------------------
% Gaussian Process Emulators
% --------------------------------------------------------------------------------------------------

% Generic GP commands. 
\newcommand{\Ndesign}{N}
\newcommand{\GP}{\mathcal{GP}} % GP distribution.
\newcommand{\LNP}{\mathcal{LNP}} % Log-normal process distribution. 
\newcommand{\func}{f} % Generic function used for results that hold for any generic GP. 
\newcommand{\funcPrior}{\func_{0}}
\newcommand{\funcVal}[1][\Ndesign]{\varphi_{#1}}
\newcommand{\funcValExact}[1][\Ndesign]{\varphi_{#1}^{*}}
\newcommand{\gpMeanBase}{\mu} % The base notation used for GP mean (no sub/superscripts). 
\newcommand{\gpKerBase}{k}
\newcommand{\gpMeanPrior}{\gpMeanBase_0}
\newcommand{\gpKerPrior}{\gpKerBase_0}
\newcommand{\gpMean}[1][\Ndesign]{\gpMeanBase_{#1}}
\newcommand{\gpKer}[1][\Ndesign]{\gpKerBase_{#1}}
\newcommand{\nuggetSD}{\eta}
\newcommand{\kerMat}[1][\Ndesign]{K_{#1}}
\newcommand{\emLawPrior}[1]{{\Pi}^{#1}} % The prior law/distribution of an emulator.
\newcommand{\design}[1][\Ndesign]{\mathcal{D}_{#1}}
\newcommand{\designRand}[2][\Nbatch]{\mathcal{D}_{#1}^{#2}} % When the response is unobserved/random. 
\newcommand{\emLaw}[2][\design]{{\Pi}^{#2}_{#1}} % The conditional law/distribution of an emulator (may or may not be Gaussian). 

% Exponentiated quadratic kernel.
\newcommand{\lengthscale}{\ell}
\newcommand{\margSD}{\sigma_{\gpKerBase}}

% Log-likelihood and forward model emulators. 
\newcommand{\fwdPrior}{\fwd_{0}}
\newcommand{\llikPrior}{\llik_{0}}
\newcommand{\funcEm}[1][\Ndesign]{\func_{#1}}
\newcommand{\fwdEm}[1][\Ndesign]{\fwd_{#1}}
\newcommand{\llikEm}[1][\Ndesign]{\llik_{#1}} 
\newcommand{\llikEmFwd}[1][\Ndesign]{\llik^{\fwd}_{#1}}
\newcommand{\Em}[2][\Ndesign]{{#2}_{#1}}

% Mean and kernel functions for log-likelihood and forward model emulators. 
\newcommand{\emMeanPrior}[1]{\gpMeanPrior^{#1}} 
\newcommand{\emKerPrior}[1]{\gpKerPrior^{#1}}
\newcommand{\emMean}[2][\Ndesign]{\gpMeanBase^{#2}_{#1}} % First argument is number of design points, second is the function being emulated. 
\newcommand{\emKer}[2][\Ndesign]{\gpKerBase^{#2}_{#1}}

% Random (sample) approximations. 
\newcommand{\llikEmRdm}[2][\Ndesign]{{#2}_{#1}}
\newcommand{\fwdEmRdm}[2][\Ndesign]{{#2}_{#1}^{\fwd}}
\newcommand{\llikEmRdmDens}[1][\Ndesign]{\llikEmRdm[#1]{\postDens}} % Convenience function 
\newcommand{\fwdEmRdmDens}[1][\Ndesign]{\fwdEmRdm[#1]{\postDens}} % Convenience function 

% Marginal approximations. 
\newcommand{\llikEmMarg}[2][\Ndesign]{{#2}_{#1}^{\text{marg}}}
\newcommand{\fwdEmMarg}[2][\Ndesign]{{#2}_{#1}^{\fwd \text{,marg}}}
\newcommand{\llikEmMargDens}[1][\Ndesign]{\llikEmMarg[#1]{\postDens}} 
\newcommand{\fwdEmMargDens}[1][\Ndesign]{\fwdEmMarg[#1]{\postDens}} 
\newcommand{\CovComb}[1][\Ndesign]{\Em[{#1}]{\likPar}^{\fwd}}

% Mean approximations. 
\newcommand{\llikEmMean}[2][\Ndesign]{{#2}_{#1}^{\text{mean}}}
\newcommand{\fwdEmMean}[2][\Ndesign]{{#2}_{#1}^{\fwd \text{,mean}}}
\newcommand{\llikEmMeanDens}[1][\Ndesign]{\llikEmMean[#1]{\postDens}} 
\newcommand{\fwdEmMeanDens}[1][\Ndesign]{\fwdEmMean[#1]{\postDens}} 

% Quantile approximations. 
\newcommand{\quantileProb}{\alpha}
\newcommand{\GaussianCDF}{\Phi}
\newcommand{\quantile}[1][\quantileProb]{q^{\quantileProb}}
\newcommand{\llikEmQ}[2][\Ndesign]{{#2}_{#1}^{\quantileProb}}
\newcommand{\fwdEmQ}[2][\Ndesign]{{#2}_{#1}^{\fwd, \quantileProb}}
\newcommand{\llikEmQDens}[1][\Ndesign]{\llikEmQ[#1]{\postDens}} 
\newcommand{\fwdEmQDens}[1][\Ndesign]{\fwdEmQ[#1]{\postDens}} 

% Joint marginal approximations. 
\newcommand{\llikEmJointMarg}[2][\Ndesign]{{#2}_{#1}^{\textrm{joint-marg}}}
\newcommand{\fwdEmJointMarg}[2][\Ndesign]{{#2}_{#1}^{\fwd \textrm{,joint-marg}}}

% Joint Monte Carlo within Metropolis-Hastings noisy MCMC approximations. 
\newcommand{\mcwmhJointLabel}{\textrm{mcwmh-joint}}
\newcommand{\mcwmhIndLabel}{\textrm{mcwmh-ind}}
\newcommand{\mhPseudoMargLabel}{\textrm{mh-pseudo-marg}}
\newcommand{\llikEmJointMCWMH}[2][\Ndesign]{{#2}_{#1}^{\mcwmhJointLabel}}
\newcommand{\fwdEmJointMCWMH}[2][\Ndesign]{{#2}_{#1}^{\fwd \textrm{,\mcwmhJointLabel}}}

% Design. 
\newcommand{\Nbatch}{B}
\newcommand{\Nrounds}{K_{\textrm{design}}}
\newcommand{\designIndex}{k}
\newcommand{\Naugment}{\Ndesign + \Nbatch}
\newcommand{\idxDesign}{n} % Primary symbol used to index each design point. 
\newcommand{\parMat}{U} % Generic matrix of input parameter points. 
\newcommand{\designIn}[1][\Ndesign]{U_{#1}} % Design inputs. 
\newcommand{\designOutLlik}[1][\Ndesign]{\ell_{#1}}
\newcommand{\designOutFwd}[1][\Ndesign]{g_{#1}}
\newcommand{\designOutLlikExact}[1][\Ndesign]{\ell^*_{#1}}
\newcommand{\designOutFwdExact}[1][\Ndesign]{g^*_{#1}}
\newcommand{\designBatchIn}{U_{\Nbatch}}
\newcommand{\designBatchLlik}{\ell_{\Nbatch}}
\newcommand{\designBatchFunc}{\varphi_{\Nbatch}}
\newcommand{\designBatchFwd}{g_{\Nbatch}}
\newcommand{\fwdEmCond}[2][\Ndesign]{\fwdEm[{#1}]^{#2}}
\newcommand{\acq}[1][\Ndesign]{\mathcal{A}_{#1}}

% Acquisition functions 
\newcommand{\labelAcq}[2][\Ndesign]{\acq[#1]^{\textrm{#2}}}
\newcommand{\maxvarLabel}{\textrm{max-var}}
\newcommand{\maxentLabel}{\textrm{max-ent}}
\newcommand{\intvarLabel}{\textrm{int-var}}
\newcommand{\intentLabel}{\textrm{int-ent}}
\newcommand{\maxexpvarLabel}{\textrm{max-exp-var}}
\newcommand{\maxexpentLabel}{\textrm{max-exp-ent}}
\newcommand{\intExpVarLabel}{\textrm{int-exp-var}}
\newcommand{\fwdintExpVarLabel}{\fwd \textrm{,}\intExpVarLabel}
\newcommand{\intExpEntLabel}{\textrm{int-exp-ent}}
\newcommand{\Ent}{H} % Entropy 
\newcommand{\weightDens}{\rho}
\newcommand{\varInflation}{v}

% Basis function emulation. 
\newcommand{\basisVec}{\phi}
\newcommand{\basisWeight}{w}
\newcommand{\dimBasis}{R}
\newcommand{\idxBasis}{r}

% --------------------------------------------------------------------------------------------------
% Markov Chain Monte Carlo. 
% --------------------------------------------------------------------------------------------------

% Generic MCMC commands. 
\newcommand{\propDens}{q} % Proposal density
\newcommand{\propPar}{\tilde{\Par}} % Proposal density
\newcommand{\llikSamp}{\ell} 
\newcommand{\llikSampProp}{\tilde{\ell}} 
\newcommand{\llikSampDist}{\nu}
\newcommand{\CovProp}{\mathbf{C}} % Proposal covariance matrix
\newcommand{\NMCMC}{K_{\text{MCMC}}} % Proposal covariance matrix
\newcommand{\accProbMH}{\alpha} % Metropolis-Hastings acceptance probability
\newcommand{\avgAccProbMH}{\overline{a}}
\newcommand{\accProbRatio}{r}
\newcommand{\likRatio}{\lambda}
\newcommand{\MarkovKernel}{P}
\newcommand{\mcmcIndex}{k}
\newcommand{\indexMCMC}[2][\mcmcIndex]{{#2}_{#1}}






\newcommand{\bphi}{\boldsymbol{\phi}}

\setlist{topsep=1ex,parsep=1ex,itemsep=0ex}
\setlist[1]{leftmargin=\parindent}
\setlist[enumerate,1]{label=\arabic*.,ref=\arabic*}
\setlist[enumerate,2]{label=(\alph*),ref=(\alph*)}

% For embedding images
\graphicspath{{../output/gp_post_approx_paper/}}

% Specifically for paper formatting 
\renewcommand{\baselinestretch}{1.2} % Spaces manuscript for easy reading

% Formatting definitions, propositions, etc. 
\newtheorem{definition}{Definition}
\newtheorem{prop}{Proposition}
\newtheorem{lemma}{Lemma}
\newtheorem{thm}{Theorem}
\newtheorem{corollary}{Corollary}

% Title and author
\title{Gaussian Process Surrogates for Bayesian Inverse Problems: Posterior Approximation and Sequential Design}
\author{Andrew Roberts}

\begin{document}

\maketitle

% Introduction 
\section{Introduction}

% Setting and Background 
\section{Setting and Background}

\subsection{Bayesian Inverse Problems}
We consider a \textit{forward model} $\fwd: \R^{\dimPar} \to \R^{\dimObs}$ describing some system of interest, parameterized by 
input parameters $\Par \in \parSpace \subset \R^{\dimPar}$. In addition, we suppose that we have noisy observations 
$\obs \in \obsSpace \subset \R^{\dimObs}$ of the output signal that $\fwd(\Par)$ seeks to approximate. The 
\textit{inverse problem} concerns learning the parameter values $\Par$ such that $\fwd(\Par) \approx \obs$; i.e., \textit{calibrating}
the model so that it agrees with reality. The statistical approach to this problem assumes that the link between model outputs and 
observations is governed by a probability distribution on  $\obs \given \Par$. We assume that this distribution admits a density 
with corresponding log-likelihood 
\begin{align}
\llik: \parSpace \to \R, \label{log_likelihood}
\end{align}
such that $p(\obs \given \Par) = \Exp{\llik(\Par)}$ 
\footnote{For succinctness, the notation $\llik(\Par)$ suppresses dependence on the data $\obs$, 
as we assume the data is fixed throughout the analysis.}
\footnote{We are currently focusing on the forward model parameter $\Par$, treating other likelihood 
parameters (e.g., observation covariance) as fixed. We will relax this assumption 
in section \ref{section_lik_par}.}.

The Bayesian approach completes this specification with a prior distribution 
on $\Par$. Letting $\priorDens(\Par)$ denote the density of this distribution, the Bayesian solution of the inverse problem is given 
by the posterior distribution $\postDensNorm(\Par) \Def p(\Par | \obs)$. The posterior density is computed by Bayes' rule 
\begin{align}
\postDensNorm(\Par) = \frac{1}{\normCst}\postDens(\Par) \Def \frac{1}{\normCst} \priorDens(\Par) \Exp{\llik(\Par)}, \label{post_dens}
\end{align}
with normalizing constant given by 
\begin{align}
\normCst &= \int \priorDens(\Par) \Exp{\llik(\Par)} d\Par. \label{norm_cst}
\end{align}
We emphasize that throughout this paper $\postDens(\Par)$ denotes the \textit{unnormalized} 
posterior density. When relevant, we will make explicit the dependence on the forward 
model by writing $\llik(\Par; \fwd)$ and $\postDens(\Par; \fwd)$. In addition to considering 
generic likelihoods, we will give special attention to the additive Gaussian noise model
\begin{align}
\obs &= \fwd(\Par) + \noise \label{inv_prob_Gaussian} \\
\noise &\sim \Gaussian(0, \likPar) \nonumber \\
\Par &\sim \priorDens \nonumber, 
\end{align}
with corresponding log-likelihood 
\begin{align}
\llik(\Par; \fwd) &= \log\Gaussian(\obs| \fwd(\Par), \likPar) 
= -\frac{1}{2} \log\det (2\pi \likPar) - \frac{1}{2} (\obs - \fwd(\Par))^\top \likPar^{-1} (y - \fwd(\Par)). \label{llik_Gaussian}
\end{align}

In general, the nonlinearity of $\fwd$ precludes a closed-form characterization of the posterior. In this case, the 
standard approach is to instead draw samples from the distribution using a Markov chain Monte Carlo (MCMC) 
algorithm. Such algorithms are iterative in nature, often requiring $\BigO(10^6)$ iterations, with each 
iteration involving an evaluation of the unnormalized posterior density $\postDens(\Par)$. In the inverse problem 
context, this computational requirement is often prohibitive when the forward model evaluations $\fwd(\Par)$
incur significant computational cost. This becomes clear upon observing that each log-likelihood evaluation 
$\llik(\Par)$ (e.g., \ref{llik_Gaussian} in the additive Gaussian case) requires the corresponding forward 
model evaluation $\fwd(\Par)$. Motivated by this problem, a large body of work has focused on deriving 
cheap approximations to either $\fwd(\Par)$ or $\llik(\Par)$ (note that approximating the former induces 
an approximation of the latter). We refer to such approximations interchangeably 
as \textit{surrogates} or \textit{emulators}. The focus of the present work is on utilizing \textit{Gaussian processes} 
to derive emulators for the forward model or log-likelihood. 

% Motivating Example: Parameter Estimation for Dynamical Models
\subsection{Motivating Example: Parameter Estimation for Dynamical Models} \label{dynamical_models}
While many problems fall under the generic setting outlined above, our primary application of interest 
stems from the problem of parameter estimation for dynamical models that describe complex systems. 
This problem arises in various forms under the umbrella of Earth system modeling \citep{ESM_modeling_2pt0}, 
and we pull examples primarily from ecological models of the terrestrial biosphere (\todo: cite). We now 
establish notation for the dynamical model applications, and cast the problem of parameter estimation for 
differential equations as a specific instance of the generic inverse problem \ref{inv_prob_Gaussian}. 
Consider the initial value problem 
\begin{align}
\frac{d}{d\Time} \state(\Time, \Par) &= \odeRHS(\state(\Time, \Par), \Par), &&x(\timeStart) = \stateIC, \label{ode_ivp}
\end{align}
describing the time evolution of $\dimState$ state variables 
$\state(\Time) \Def \left[\indexState[1]{\state}(\Time), \dots, \indexState[\dimState]{\state}(\Time)\right]^\top \in \mathbb{R}^{\dimState}$,
with dynamics depending on parameters $\Par \in \parSpace$. As our focus will be on estimating these parameters 
from observations, we consider the solution operator 
\begin{align}
\Par &\mapsto \left\{\state(\Time, \Par) :  \Time \in [\timeStart, \timeEnd] \right\}, 
\end{align}
mapping parameters to state trajectories over some time horizon $[\timeStart, \timeEnd]$. In practice, the solution 
is often approximated via a numerical discretization of the form 
\begin{align}
\solutionOp: \Par &\mapsto \left[\indexTime[0]{\state}(\Par), \dots, \indexTime[\NTimeStep]{\state}(\Par) \right]^\top. 
\end{align}
Here, $\solutionOp: \parSpace \to \R^{\NTimeStep \times \dimState}$ represents the map implied by a numerical solver, 
and $\indexTime[0]{\state}(\Par), \dots, \indexTime[\NTimeStep]{\state}(\Par)$ are approximations of the state 
values $\state(\Time, \Par)$ at a finite set of time points in $[\timeStart, \timeEnd]$. 
Going forward, we focus on the approximate solution operator $\solutionOp$, neglecting discretization error 
for simplicity. Finally, suppose we have observed data $\obs \in \R^{\dimObs}$ that we model as a
noise-corrupted function of the state trajectory. This is formalized by the definition of an observation operator 
$\obsOp: \R^{\NTimeStep \times \dimState} \to \R^{\dimObs}$ mapping from the state trajectory to a 
$\dimObs$-dimensional observable quantity. We assume the observations are generated as 
\begin{align}
\obs &= (\obsOp \circ \solutionOp)(\ParTrue) + \noise \\
\noise &\sim \Gaussian(0, \likPar) \nonumber 
\end{align}
for some ``true'' parameter value $\ParTrue \in \parSpace$. Observe that this falls within the generic 
inverse problem framework with forward model $\fwd \Def \obsOp \circ \solutionOp$. 
Some of the methods discussed below readily 
generalize to more complicated settings that consider non-additive or non-Gaussian noise, or the presence of 
model discrepancy terms. While the observation map can assume many forms across different applications, 
common examples include 
\begin{align}
\obsOp(\indexState[0]{\state}, \dots, \indexState[\dimState]{\state}) 
&\Def \frac{1}{\timeEnd - \timeStart} \left(\sum_{\timeIndex=1}^{\NTimeStep} \left[\state^{(\stateIndex)}_{\timeIndex} \right]^{\idxObsDim} \right)_{\idxObsDim=1}^{\dimObs} \label{obs_op_moments} \\ 
\obsOp(\indexState[0]{\state}, \dots, \indexState[\dimState]{\state}) 
&\Def \left(\state^{(\stateIndex)}_{\timeIndex} \right)_{\timeIndex \in \mathcal{K}}, \mathcal{K} \subset \{1, 2, \dots, \NTimeStep\} \label{obs_op_subset}
\end{align} 
for some $\stateIndex \in \{1, 2, \dots, \dimState\}$. The map \ref{obs_op_moments} corresponds to the first $\dimObs$ moments of the 
$\stateIndex^{\text{th}}$ state trajectory, with $\dimObs=1$ implying a standard time average. The second example \ref{obs_op_subset}
indicates that noisy observations of the $\stateIndex^{\text{th}}$ state are available at a subset of the discrete time set. 
Of course, we can also generalize so that these observation operators depend on multiple state variables. 

% Gaussian Processes
\subsection{Gaussian Processes}
In this section, we provide a brief review of Gaussian processes; for in-depth treatments, we refer readers to 
\cite{gramacy2020surrogates, StuartTeck2, gpML}. A \textit{Gaussian process} (GP) is a probability 
distribution over a set of functions $\funcPrior: \parSpace \to \R$, defined by the property that 
the random vector $[\funcPrior(\Par_1), \dots, \funcPrior(\Par_\Ndesign)]^\top$ has
a joint Gaussian distribution for any set of inputs $\parMat \Def \{\Par_1, \dots, \Par_\Ndesign\} \subset \parSpace$. 
A GP is defined by its \textit{mean function} $\gpMeanPrior: \parSpace \to \R$ and positive-definite \textit{kernel} 
(i.e., \textit{covariance function}) $\gpKerPrior: \parSpace \times \parSpace \to \R$ and is thus denoted by 
$\GP(\gpMeanPrior, \gpKerPrior)$. Throughout this paper, we will vectorize notation as follows. Letting 
$\tilde{\parMat} \Def [\tilde{\Par}_1, \dots, \tilde{\Par}_M]  \subset \parSpace$ be a second set of inputs, we define 
$\funcPrior(\parMat) \Def [\funcPrior(\Par_1), \dots, \funcPrior(\Par_\Ndesign)]^\top \in \R^{\Ndesign}$, 
\footnote{We similarly use this vectorized notation for other functions; e.g., $\gpMeanPrior(\parMat)$, $\fwd(\parMat)$, and $\llik(\parMat)$}
$\gpKerPrior(\parMat, \tilde{\parMat}) \Def \{\gpKerPrior(\Par_{\idxDesign}, \tilde{\Par}_m)\}_{\substack{1 \leq \idxDesign \leq \Ndesign \\ 1 \leq m \leq M}} \in \R^{\Ndesign \times M}$, 
and $\gpKerPrior(\parMat) \Def \gpKerPrior(\parMat, \parMat) \in \R^{\Ndesign \times \Ndesign}$. Given this notation, 
the defining property of the GP can be written as 
\begin{align}
\funcPrior(\parMat) &\sim \Gaussian(\gpMeanPrior(\parMat), \gpKerPrior(\parMat)).
\end{align}
For our purposes, the GP $\funcPrior$ will be a prior distribution for a \textit{deterministic} function $\func: \parSpace \to \R$. 
However, the underlying ideas explored here can be extended to the setting where the evaluations of $\func$ 
are corrupted by noise. 
Now suppose that, at a set of \textit{design points} $\designIn$, we have computed the noiseless function values
$\func(\designIn)$. Viewing these values as a realization of the random variable $\funcPrior(\designIn)$, we can 
consider the conditional (i.e., \textit{predictive}) distribution $\funcPrior|[\funcPrior(\designIn)=\func(\designIn)]$. It is well-known
\cite{gpML} that this conditional distribution is also a GP
\begin{align}
\funcEm[\Ndesign] \Def \funcPrior|[\funcPrior(\designIn)=\func(\designIn)]\sim \GP(\gpMean[\Ndesign], \gpKer[\Ndesign]), \label{generic_gp_conditional}
\end{align}
with mean and kernel given by
\footnote{The GP conditional mean and covariance equations apply more generally when conditioning on 
$\{\designIn[\Ndesign], \funcVal[\Ndesign]\}$, regardless of whether $\funcVal[\Ndesign]$ is interpreted as 
a set of noiseless function evaluations or not. However, in our setting we reserve the notation $\gpMean[\Ndesign]$
for the case when $\funcVal[\Ndesign] = \func(\designIn[\Ndesign])$; i.e., conditioning on the exact function 
evaluations. Note that $\gpKer[\Ndesign]$ is independent of $\funcVal[\Ndesign]$ so this distinction is only 
relevant for the mean.} 
\begin{align}
\gpMean[\Ndesign](\parMat) &= \gpMeanPrior(\parMat) + \gpKerPrior(\parMat, \designIn) \gpKerPrior(\designIn)^{-1} [\func(\designIn) - \gpMeanPrior(\designIn)] \label{kriging_eqns} \\
\gpKer[\Ndesign](\parMat) &= \gpKerPrior(\parMat) - \gpKerPrior(\parMat, \designIn) \gpKerPrior(\designIn)^{-1} \gpKerPrior(\designIn, \parMat). \nonumber
\end{align}
The predictive mean can be seen to interpolate the design points (i.e., $\gpMean[\Ndesign](\designIn) = \func(\designIn[\Ndesign])$), with 
the predictive variance vanishing at these inputs (i.e., $\gpKer[\Ndesign](\designIn) = 0$). To break these interpolation properties,
a constant can be added to the diagonal of the kernel matrix so that $(\gpKerPrior(\designIn) + \nuggetSD^2 I)^{-1}$ replaces 
$\gpKerPrior(\designIn)^{-1}$ in \ref{kriging_eqns}. In the present context, the maps $\fwd$ and $\llik$ are assumed to be deterministic functions 
of $\Par$ and hence the interpolation property is desirable. Nevertheless, a small, positive fixed constant $\nuggetSD^2$ is still 
typically included for numerical stability. 

Our default choice of covariance function is the popular \textit{exponentiated quadratic} 
(also called \textit{squared exponential} or \textit{Gaussian}) kernel, which is given by 
\begin{align}
\gpKerPrior(\Par, \tilde{\Par}) &= 
\margSD^2 \Exp{- \sum_{\idxParDim=1}^{\dimPar} \left(\frac{\Par_{\idxParDim} - \tilde{\Par}_{\idxParDim}}{\lengthscale_{\idxParDim}} \right)^2}, 
\end{align}
defined by the choice of positive hyperparameters $\margSD, \lengthscale_{1}, \dots, \lengthscale_{\dimPar}$. This is an example
of a stationary, or shift-invariant, kernel since $\gpKerPrior(\Par, \tilde{\Par})$ only depends on its arguments through their difference 
$\Par - \tilde{\Par}$. To complete the GP prior specification, we typically take the mean 
function to be a constant or quadratic. We opt for the common empirical Bayes approach of fixing all mean and kernel hyperparameters 
at their maximum (marginal) likelihood estimates.

Finally, we consider the extension to multi-output functions $\funcPrior: \parSpace \to \R^{\dimObs}$. There is a large literature on multi-output 
kernels (\todo: cite), but we focus here on the simplest case in which each output is modeled separately as an independent GP. 
In this setting, the notation $\gpKerPrior(\Par)$ is now interpreted as the $\dimObs \times \dimObs$ diagonal matrix collecting the variance of each 
independent GP. Similarly, for a set of inputs $\parMat \in \R^{M \times \dimPar}$, $\gpKerPrior(\parMat)$ is now the 
$M\dimObs \times M\dimObs$ block diagonal matrix constructed from the predictive covariances of each independent GP. 

\subsection{Surrogate Construction}
We now consider modeling either $\fwd$ or $\llik$ with a GP, referring to these cases as \textit{forward model emulation} and 
\textit{log-likelihood} emulation, respectively. As mentioned previously, computationally costly forward models render 
standard inference algorithms intractable for solving Bayesian inverse problems due to the large number of model 
evaluations that must be conducted iteratively. The main idea of model emulation is to front-load the forward model 
runs, which can often be computed in parallel, to obtain a design $\{\designIn, \fwd(\designIn)\}$ or 
$\{\designIn, \llik(\designIn)\}$ which can then be used to train a model that 
approximates $\fwd$ or $\llik$. A basic illustration of forward model and log-likelihood GP emulators, along with the 
(log) likelihood approximations they induce, is provided in \Cref{fig:em_dist_1d}. 

\subsubsection{Forward Model Emulation}
If the output dimension $\dimObs$ is reasonably small, then a common approach is to specify an independent 
multi-output GP prior directly for the forward model, $\fwdPrior \sim \GP(\emMeanPrior{\fwd}, \emKerPrior{\fwd})$. 
Conditioning on this design yields the predictive distribution 
$\fwdEm[\Ndesign]  \Def \fwdPrior|[\fwdPrior(\designIn) = \fwd(\designIn)] \sim \GP(\emMean[\Ndesign]{\fwd}, \emKer[\Ndesign]{\fwd})$, 
which constitutes the emulator for $\fwd. $
We emphasize that $\fwdEm[\Ndesign]$ is a \textit{stochastic} surrogate; 
the model predictions at inputs $\parMat$ are given by the entire distribution 
$\fwdEm[\Ndesign](\parMat) \sim \Gaussian(\emMean[\Ndesign]{\fwd}(\parMat), \emKer[\Ndesign]{\fwd}(\parMat))$. 
Thus, the mean $\emMean[\Ndesign]{\fwd}(\parMat)$ provides a point estimate while the covariance 
$\emKer[\Ndesign]{\fwd}(\parMat)$ summarizes the uncertainty in the estimate. This is useful for the purpose
of uncertainty quantification in that the additional uncertainty introduced as a consequence of the surrogate 
approximation can be propagated to the approximate posterior distribution. We note that plugging the random 
function $\fwdEm[\Ndesign]$ in place of $\fwd$ in the log-likelihood induces an approximation of the log-likelihood,
which we denote by $\llikEmFwd$. 
In the case of the additive Gaussian likelihood \ref{llik_Gaussian} this approximation takes the form 
\begin{align}
\llikEmFwd(\Par) \Def \llik(\Par; \fwdEm) =  \log\Gaussian(\obs| \fwdEm[\Ndesign](\Par), \likPar). \label{induced_llik_emulator}
\end{align}
We emphasize two facts: (i.) the induced log-likelihood approximation $\llikEmFwd[\Ndesign](\Par)$ is also 
stochastic, but no longer Gaussian in general; and (ii.) no surrogate modeling is performed directly for $\llik$; instead, 
the approximation $\llikEmFwd[\Ndesign](\Par) = \llik(\Par; \fwdEm[\Ndesign])$ is derived by plugging the 
conditioned forward model emulator into the log-likelihood function, 
constituting a modular surrogate-modeling approach \cite{modularization}. While the induced stochastic approximation 
to the likelihood (and posterior density) cannot in general be described explicitly, the Gaussian likelihood 
example \ref{induced_llik_emulator} represents an important exception. The mean and variance of the induced
likelihood approximation are given in the following proposition. 

\begin{prop} \label{fwd_em_lik_emulator_moments_Gaussian}
Given a GP forward model emulator $\fwdEm[\Ndesign] \sim \GP(\emMean[\Ndesign]{\fwd}, \emKer[\Ndesign]{\fwd})$, 
the plug-in likelihood estimator $\Exp{\llik(\Par; \fwdEm)} = \Gaussian(\obs| \fwdEm[\Ndesign](\Par), \likPar)$ has 
mean and variance given by 
\begin{align}
\E_{\fwdEm}\left[\Exp{\llik(\Par; \fwdEm)} \right] 
&= \Gaussian(\obs | \emMean[\Ndesign]{\fwd}(\Par), \likPar + \emKer[\Ndesign]{\fwd}(\Par)) \\
\Var_{\fwdEm}\left[\Exp{\llik(\Par; \fwdEm)} \right]
&= \frac{\Gaussian\left(\obs | \emMean[\Ndesign]{\fwd}(\Par), \frac{1}{2}\likPar + \emKer[\Ndesign]{\fwd}(\Par)  \right)}{\det(2\pi \likPar)^{1/2}}
- \frac{\Gaussian\left(\obs | \emMean[\Ndesign]{\fwd}(\Par), \frac{1}{2}\left[\likPar + \emKer[\Ndesign]{\fwd}(\Par)\right]  \right)}{\det(2\pi [\likPar + \emKer[\Ndesign]{\fwd}(\Par)])^{1/2}}
\end{align}
\end{prop}
This follows as a special case of \Cref{prop:Gaussian_marginal_moments}, and its implications will be explored in subsequent sections. 

We note that the emulation of expensive models is one of the central questions studied in the 
computer experiments literature, with GPs arguably representing the most popular tool 
for the job; see \cite{gramacy2020surrogates} and 
\cite{design_analysis_computer_experiments} for detailed treatments.
The forward model emulation strategy detailed above becomes problematic when the output 
dimension $\dimObs$ is large or the outputs are highly interdependent, rendering the independent GP 
approach infeasible or ill-advised.
For example, the forward model output may correspond to the numerical solution of 
a differential equation and thus may take the form of a time series or spatiotemporal grid. 
A great deal of literature has addressed these challenges, including approaches tailored to forward models
with dynamical structure 
(e.g., \cite{GP_dynamic_emulation, Bayesian_emulation_dynamic, Liu_West_dynamic_emulation, dynamic_nonlinear_simulators_GP}).
A popular alternative is to first approximately express the high-dimensional output as a linear combination of basis vectors 
\begin{align}
\fwd(\Par) &= \sum_{\idxBasis=1}^{\dimBasis} \basisWeight_{\idxBasis}(\Par) \basisVec_{\idxBasis} + \noise_{\basisVec}, \label{fwd_model_basis_func}
\end{align}
and then model the coefficients $\basisWeight_{\idxBasis}(\Par)$ using independent GPs. The basis vectors 
are typically found via a singular value decomposition, though other bases have been explored 
\citep{HigdonBasis, emulate_functional_output, functionValuedModels, PODemulation}. The surrogate constructed in 
this fashion is simply a linear combination of independent GP emulators, and hence most of the results on forward model 
emulation in this paper are easily extended to this setting. See the appendix for details (\todo). 

\subsubsection{Log-Likelihood Emulation}
We observe in \ref{post_dens} that the forward model evaluations appear in the posterior density only through the 
likelihood. Thus, for the purposes of posterior inference, it is sufficient to directly approximate $\llik(\Par)$ without 
obtaining an approximation of the forward model. It is typically preferable to focus on approximating the logarithm 
of the likelihood, which usually results in a smoother response surface more amenable to GP emulation [\todo: cite]. 
The emulator construction proceeds as before; a GP prior $\llikPrior \sim \GP(\emMeanPrior{\llik}, \emKerPrior{\llik})$ 
is specified and the predictive distribution 
\begin{align*}
\llikEm[\Ndesign] \Def \llikPrior | [\llikPrior(\designIn) = \llik(\designIn)] \sim \GP(\emMean[\Ndesign]{\llik}, \emKer[\Ndesign]{\llik})
\end{align*}
is obtained by conditioning on the design $\{\designIn, \llik(\designIn)\}$. 
The distribution of $\llikEm[\Ndesign]$ is Gaussian, unlike
the log-likelihood surrogate $\llikEmFwd[\Ndesign]$ in the forward model emulation setting. The analog to 
\Cref{fwd_em_lik_emulator_moments_Gaussian} is stated below. In this case, the distribution of the induced likelihood 
surrogate is immediately seen to be log-normal
\footnote{If $f \sim \GP(\gpMean, \gpKer)$ then we refer to the exponential of $f$ as a \textit{log-normal process}, 
denoted by $\Exp{f} \sim \LNP(\gpMean, \gpKer)$.}.

\begin{prop} \label{llik_em_lik_emulator_moments}
Given a GP log-likelihood emulator $\llikEm[\Ndesign] \sim \GP(\emMean[\Ndesign]{\llik}, \emKer[\Ndesign]{\llik})$, 
the induced likelihood surrogate is a log-normal process
$\Exp{\llikEm[\Ndesign]} \sim \LNP(\emMean[\Ndesign]{\llik}, \emKer[\Ndesign]{\llik})$, with mean and variance given 
by 
\begin{align}
\E_{\llikEm}\left[\Exp{\llikEm(\Par)} \right] 
&= \Exp{\emMean[\Ndesign]{\llik}(\Par) + \frac{1}{2} \emKer[\Ndesign]{\llik}(\Par)} \\
\Var_{\llikEm}\left[\Exp{\llikEm(\Par)} \right] 
&= \left[\Exp{\emKer[\Ndesign]{\llik}(\Par) - 1} \right] \Exp{2\emMean[\Ndesign]{\llik}(\Par) + \emKer[\Ndesign]{\llik}(\Par)}.
\end{align}
\end{prop}

The use of GPs to approximate log-likelihoods has been widely used in approximate Bayesian inference 
\citep{VehtariParallelGP, Kandasamy_2017, llikRBF, trainDynamics, quantileApprox, wang2018adaptive, landslideCalibration}
and Bayesian quadrature \citep{BayesQuadrature, BayesQuadRatios}. In the context of inverse problems with 
high-dimensional output spaces, log-likelihood emulation has the clear benefit of reducing to a univariate GP setting. 
On the other hand, it has several downsides. Even on the log scale, log-likelihoods can be fast-varying and exhibit large 
dynamic range, proving troublesome for stationary GP priors \cite{wang2018adaptive}. A second potential weakness 
stems from the fact that log-likelihoods often have known bound constraints, which may not be respected by a GP 
prior or predictive distribution. Recent work has dealt with this problem by incorporating constraints into the 
GP hyperparameter optimization \cite{quantileApprox}. A third challenge follows from the fact that, in many applications, 
the likelihood parameters (e.g., $\likPar$) are typically not known and must also be learned from data. An obvious 
solution is to extend the input space of the emulator to include the likelihood parameters, though the resulting response surface 
may prove more challenging to emulate; this approach is explored in \cite{llikRBF}. Alternatively, specific likelihood 
choices may admit a sufficient statistic which can be emulated independently of the likelihood parameters. This method is employed 
in \cite{FerEmulation}, which focuses on the Gaussian likelihood \ref{llik_Gaussian} in the specific case $\likPar = \sigma^2 I$. In this 
case, a surrogate can be constructed to approximate the sufficient statistic $\norm{\obs - \fwd(\Par)}_2^2$, 
which does not depend on the likelihood parameter $\sigma^2$. 
Finally, we note that it may be difficult to incorporate prior domain knowledge 
into the GP prior for the log-likelihood, whereas in certain settings it is easier to do so when constructing a forward model 
surrogate \cite{GP_PDE_priors}. The paper \cite{VehtariParallelGP} finds that a GP with a quadratic mean function tends to 
work well for log-likelihood emulation, perhaps reflecting the fact that many log-likelihoods tend to be well-approximated by 
a quadratic at a coarse scale, with the GP kernel correcting for finer-scale variations. 
Despite all of these challenges, we emphasize that the reduction to a scalar-valued output space is a significant benefit which 
may outweigh the downsides of log-likelihood emulation in many applications. 

\begin{figure}
     \centering
     \begin{subfigure}[b]{0.3\textwidth} % GP dist [fwd]
         \centering
         \includegraphics[width=\textwidth]{{test_1d_visualization/gp_dist_fwdem.png}}
         \caption{$\fwdEm$}
         \label{fig:gp_dist_fwd}
     \end{subfigure}
     \hfill
     \begin{subfigure}[b]{0.3\textwidth} % llik dist [fwd]
         \centering
         \includegraphics[width=\textwidth]{{test_1d_visualization/llik_dist_fwdem.png}}
         \caption{$\llikEmFwd$}
         \label{fig:llik_dist_fwd}
     \end{subfigure}
     \hfill
     \begin{subfigure}[b]{0.3\textwidth} % lik dist [fwd]
         \centering
         \includegraphics[width=\textwidth]{{test_1d_visualization/lik_dist_fwdem.png}}
         \caption{$\Exp{\llikEmFwd}$}
         \label{fig:lik_dist_fwd}
     \end{subfigure}
     \begin{subfigure}[b]{0.3\textwidth} % GP dist [llik] (equal to llik dist so excluding)
     \includegraphics[width=\textwidth]{}
     \end{subfigure}
     \hfill
     \begin{subfigure}[b]{0.3\textwidth} % llik dist [llik] 
         \centering
         \includegraphics[width=\textwidth]{{test_1d_visualization/llik_dist_llikem.png}}
         \caption{$\llikEm$}
         \label{fig:llik_dist_llik}
     \end{subfigure}
     \hfill
     \begin{subfigure}[b]{0.3\textwidth} % lik dist [llik]
         \centering
         \includegraphics[width=\textwidth]{{test_1d_visualization/lik_dist_llikem.png}}
         \caption{$\Exp{\llikEm}$}
         \label{fig:lik_dist_llik}
     \end{subfigure}
        \caption{Comparison of forward model and log-likelihood emulation in the simple one-dimensional problem ($\dimPar = \dimObs = 1$)
        described in \Cref{vsem_1d}. The top and bottom rows correspond to the forward model emulation and log-likelihood 
        emulation settings, respectively. The columns correspond to (1) the forward model emulator; (2) the log-likelihood 
        emulator induced by the underlying GP; (3) the likelihood emulator induced by the underlying GP. When directly emulating the 
        log-likelihood, the underlying GP is itself the log-likelihood emulator, and there is no forward model emulator.
        The dashed black line is the ground truth forward model or (log) likelihood and the red points are the 
        design points used for emulator training. The blue line is the mean of the respective emulators, and the gray area 
        covers the region within $1.64$ standard deviations of the mean (capturing approximately 90\% of the predictive distribution in the 
        cases where the distribution is Gaussian).}
        \label{fig:em_dist_1d}
\end{figure}

% Posterior Approximation
\section{Posterior Approximation}
With a fit GP emulator in hand, we now turn to the question of how best to utilize the emulator in approximating the posterior 
distribution of the Bayesian inverse problem. We systematically review the various methods that have been considered in the 
literature, and propose a new approach based on noisy approximations of MCMC algorithms. For succinctness, we restrict 
to the log-likelihood emulation setting when introducing posterior approximation methods throughout this section. Natural 
analogs exist in the forward model emulation setting, with the caveat that the availability of closed-form computations will 
be dependent on the specific choice of likelihood. Such considerations for posterior approximation using forward model 
surrogates are discussed in \Cref{post_approx_fwd}. 

% Random Densities and the Sample-Based Approximation
\subsection{Random Densities and the Sample-Based Approximation}
We begin by recalling that the GP emulator $\llikEm$ results in the Gaussian predictor 
 $\llikEm[\Ndesign](\parMat) \sim \Gaussian(\emMean[\Ndesign]{\llik}(\parMat), \emKer[\Ndesign]{\llik}(\parMat))$
 at any parameter set $\parMat$. Plugging this surrogate into the posterior density expression \ref{post_dens} yields
\begin{align}
\llikEmRdm[\Ndesign]{\postDensNorm}(\Par) 
&\Def \frac{1}{\llikEmRdm[\Ndesign]{\normCst}} \llikEmRdmDens[\Ndesign](\Par) 
\Def \frac{1}{\llikEmRdm[\Ndesign]{\normCst}} \priorDens(\Par) \Exp{\llikEm[\Ndesign](\Par)} \label{post_dens_random_llik},
\end{align}
with normalizing constant given by 
\begin{align}
\llikEmRdm[\Ndesign]{\normCst}
&= \int \priorDens(\Par) \Exp{\llikEm[\Ndesign](\Par)} d\Par \label{norm_cst_random_llik}. 
\end{align}
The density $\llikEmRdm[\Ndesign]{\postDensNorm}$ is random; each sample trajectory from the 
underlying GP induces a different posterior. It is worth noting that $\llikEmRdmDens[\Ndesign](\Par)$,
the unnormalized random density evaluated at an input $\Par$, inherits its randomness only from the scalar random 
variable $\llikEm(\Par)$. However, its normalized counterpart $\llikEmRdm[\Ndesign]{\postDensNorm}(\Par)$ depends 
on the entire (infinite-dimensional) GP distribution through the normalizing constant $\llikEmRdm[\Ndesign]{\normCst}$.
Given that \ref{post_dens_random_llik} fully incorporates the GP predictive distribution, we consider 
$\llikEmRdm[\Ndesign]{\postDensNorm}$ to be the baseline posterior approximation to which other methods will
be compared. Following \citet{StuartTeck1}, we refer to $\llikEmRdm[\Ndesign]{\postDensNorm}$ as the 
\textit{sample-based} approximation. This approximation can be understood from a Monte Carlo perspective, 
as summarized in \Cref{alg:sample-approx} [\todo: subscript notation indexing samples in algorithm might 
be a bit confusing given subscript on prior density; maybe change this or just define the algorithm as returning 
a single sample]. 

\begin{algorithm}
    \caption{Monte Carlo Representation of the Sample-Based Approximation}
    \label{alg:sample-approx}
    \begin{algorithmic}[1] % The number tells where the line numbering should start
    \Function{sampleApprox}{$\llikEm, \NSample$}     
        \For{$\sampleIndex \gets1,\NSample$} 
        		\State $\indexSample{\ell} \sim \law(\llikEm)$ \Comment{Sample GP trajectory}
		\State $\indexSample{\postDens}(\Par) \gets \priorDens(\Par)\Exp{\ell(\Par)}$ \Comment{Induced unnormalized density}
		\State $\indexSample{\Par} \sim \postDensNorm$ \Comment{Draw from sampled density}
	\EndFor
	\State \Return $(\indexSample[1]{\Par}, \dots, \indexSample[\NSample]{\Par})$
	\EndFunction
    \end{algorithmic}
\end{algorithm}

Alternatively, we can think of the density of the $\Par$-samples generated via this procedure, which 
can be seen to be
\begin{align}
\llikEmSampDensNorm[\Ndesign](\Par)
&\Def \E_{\llikEm}\left[\frac{\priorDens(\Par) \Exp{\llikEm[\Ndesign](\Par)}}{\int \priorDens(\Par^\prime) \Exp{\llikEm[\Ndesign](\Par^\prime)} d\Par^\prime} \right]
\label{samp_dens_norm}
\end{align}

 % The Forward Model Emulation Setting
\subsection{The Forward Model Emulation Setting} \label{post_approx_fwd}
 In this section, we discuss notable differences in the above posterior approximation methods when working 
with a forward model emulator. In general, each of the previously-discussed methods naturally generalizes 
to this setting by replacing $\llikEm[\Ndesign]$ with $\llikEmFwd[\Ndesign]$ in the definitions. Continuing 
along the lines of previous notation, we will differentiate the approximated posterior density using forward 
model emulation by a superscript $\fwd$. In the log-likelihood emulation case, the marginal and quantile 
unnormalized density approximations admit closed-form characterizations 
owing to the log-normality of the likelihood surrogate. Under a forward model emulator, the distribution of 
$\llikEmFwd[\Ndesign]$ will be dependent on the particular likelihood under consideration, so closed-form 
marginal and quantile approximations may not be available. In such cases, one might still sample from 
$\fwdEmMargDens[\Ndesign]$ via pseudo-marginal MCMC \citep{seudoMarginalMCMC}.
Under a Gaussian likelihood (e.g., \ref{inv_prob_Gaussian}), a closed-form marginal approximation is
available, given by
\begin{align}
\fwdEmMargDens[\Ndesign](\Par) 
&\Def \priorDens(\Par) \E_{\fwdEm}\left[\Exp{\llikEmFwd[\Ndesign](\Par)} \right] \nonumber  \\
&= \priorDens(\Par) \ \E_{\fwdEm}\left[\Gaussian(\obs | \fwdEm[\Ndesign](\Par), \likPar) \right] \nonumber \\
&= \priorDens(\Par) \ \Gaussian(\obs | \emMean[\Ndesign]{\fwd}(\Par), \likPar + \emKer[\Ndesign]{\fwd}(\Par)), \label{post_dens_marg_fwd_Gaussian}
\end{align}
which follows immediately from the fact that the expectation is of the form of a convolution 
of two Gaussian densities (see \Cref{prop:Gaussian_marginal_moments} in the appendix).
The marginal approximation \ref{post_dens_marg_fwd_Gaussian}
can be interpreted as the posterior distribution of the modified Bayesian inverse problem 
\begin{align}
\obs &= \emMean[\Ndesign]{\fwd}(\Par) + v(\Par) + \noise \label{inv_prob_Gaussian_marginal_approx} \\
v(\Par) &\sim \Gaussian(0, \emKer[\Ndesign]{\fwd}(\Par)) \nonumber \\
\noise &\sim \Gaussian(0, \likPar) \nonumber \\
\Par &\sim \priorDens \nonumber, 
\end{align}
where $v(\Par)$ and $\noise$ are independent. It is interesting to note that the GP uncertainty in 
\ref{post_dens_marg_fwd_Gaussian} manifests in the form of additive variance inflation, distinct 
from the multiplicative uncertainty inflation in \ref{post_dens_marg_var_inflation} \cite{GP_PDE_priors}.
[\todo: see if I can derive a result that summarizes when the density is deflated vs. inflated a a function 
of distance from $\obs$ to the GP mean prediction].  
The Gaussian posterior approximation \ref{post_dens_marg_fwd_Gaussian} has been considered in 
\cite{StuartTeck2, hydrologicalModel, GP_PDE_priors, GP_PDE_priors, CES, idealizedGCM, weightedIVAR}.

% Posterior Approximation: OLD
\section{Posterior Approximation}
The forward model and log-likelihood emulation approaches result in the random log-likelihood approximations 
$\llikEmFwd[\Ndesign]$ and $\llikEm[\Ndesign]$, respectively. The latter results in a Gaussian predictor 
$\llikEm[\Ndesign](\parMat) \sim \Gaussian(\emMean[\Ndesign]{\llik}(\parMat), \emKer[\Ndesign]{\llik}(\parMat))$ 
at any parameter set $\parMat$. On the other hand, the distribution of $\llikEmFwd[\Ndesign](\parMat)$ is 
typically non-Gaussian and will depend on the specific likelihood under consideration.
Plugging these log-likelihood surrogates into the posterior density expression \ref{post_dens} yields

\begin{align}
\fwdEmRdm[\Ndesign]{\postDensNorm}(\Par) 
&\Def \frac{1}{\fwdEmRdm[\Ndesign]{\normCst}} \fwdEmRdmDens[\Ndesign](\Par) 
\Def  \frac{1}{\fwdEmRdm[\Ndesign]{\normCst}} \priorDens(\Par) \Exp{\llikEmFwd[\Ndesign](\Par)} \label{post_dens_random_fwd} \\
\llikEmRdm[\Ndesign]{\postDensNorm}(\Par) 
&\Def \frac{1}{\llikEmRdm[\Ndesign]{\normCst}} \llikEmRdmDens[\Ndesign](\Par) 
\Def \frac{1}{\llikEmRdm[\Ndesign]{\normCst}} \priorDens(\Par) \Exp{\llikEm[\Ndesign](\Par)} \label{post_dens_random_llik},
\end{align}
with normalizing constants given by 
\begin{align}
\fwdEmRdm[\Ndesign]{\normCst}
&= \int \priorDens(\Par) \Exp{\llikEmFwd[\Ndesign](\Par)} d\Par \label{norm_cst_random_fwd} \\
\llikEmRdm[\Ndesign]{\normCst}
&= \int \priorDens(\Par) \Exp{\llikEm[\Ndesign](\Par)} d\Par \label{norm_cst_random_llik}. 
\end{align}
The unnormalized posterior densities $\fwdEmRdmDens[\Ndesign]$ and $\llikEmRdmDens[\Ndesign]$ are stochastic; 
each sample trajectory from the underlying GP induces a different posterior. 
The papers \cite{StuartTeck1, StuartTeck2} refer to this as the \textit{sample-based}
approximation of the posterior. Given that it is not obvious how to perform inference using these random objects, one typically 
proceeds by choosing a single, deterministic posterior approximation that somehow summarizes the uncertainty in the 
random posterior measure. Popular choices for such deterministic approximations are described below. Going forward, 
for succinctness we will specify only the \textit{unnormalized} posterior densities, with the associated normalizing constants  
implicitly defined. Existence of the approximate posteriors is an important question that we discuss in \Cref{post_existence}. 
\Cref{fig:post_approx_1d} provides a one-dimensional illustration of the posterior approximations 
discussed below, utilizing the same emulators from \Cref{fig:em_dist_1d}. 

\begin{figure}
     \centering
     \begin{subfigure}[b]{0.48\textwidth} % Posterior approximations [fwd]
         \centering
         \includegraphics[width=\textwidth]{{test_1d_visualization/post_approx_fwdem.png}}
         \caption{$\postDensNorm(\Par)$ approximation using $\fwdEm$}
         \label{fig:y equals x}
     \end{subfigure}
     \hfill
     \begin{subfigure}[b]{0.48\textwidth} % Posterior approximations [fwd]
         \centering
         \includegraphics[width=\textwidth]{{test_1d_visualization/post_approx_llikem.png}}
         \caption{$\postDensNorm(\Par)$ approximation using $\llikEm$}
         \label{fig:three sin x}
     \end{subfigure}
        \caption{Posterior density approximations derived from the emulators in \Cref{fig:em_dist_1d}, a continuation of 
        the example detailed in \Cref{vsem_1d}. The left and right 
        figures plot the approximations derived from the forward model emulator and log-likelihood emulator, 
        respectively. The dashed black line is the true posterior density (no emulation). The exact posterior, as well 
        as the mean and marginal approximations, are normalized by approximating the normalizing constant using 
        numerical integration. The $\mcwmhJointLabel$ and $\mcwmhIndLabel$ densities are kernel density estimates 
        constructed from MCMC samples. Note that in general the normalized posterior density approximations do not 
        interpolate the design points due to differences in normalizing constants. \todo: change design points to vertical 
        dashed lines and add a second row to the legend so it doesn't get cut off.}
        \label{fig:post_approx_1d}
\end{figure}

\subsection{Approximating the Unnormalized Posterior Density}
Arguably the simplest approach to constructing a deterministic approximation is to reduce the random emulator 
to a deterministic one by considering only the GP predictive mean. 
The plug-in mean estimators yield the unnormalized posterior density approximations 
\begin{align}
\fwdEmMeanDens[\Ndesign](\Par) &\Def \priorDens(\Par) \Exp{\llik(\Par; \emMean[\Ndesign]{\fwd}(\Par))}
 \label{post_dens_mean_fwd} \\
\llikEmMeanDens[\Ndesign](\Par) &\Def \priorDens(\Par) \Exp{\emMean[\Ndesign]{\llik}(\Par)} 
\label{post_dens_mean_llik}.
\end{align}
These approximations fail to account for the uncertainty introduced by the surrogate model, which can lead 
to overconfidence in downstream posterior inference. These mean-based approximations are analyzed 
theoretically in \cite{StuartTeck1}, with error bounds provided in Hellinger distance. The paper 
\cite{gp_surrogates_random_exploration} considers fitting GP emulators directly to the unnormalized 
log posterior density, and also provides Hellinger-type error bounds for the plug-in mean posterior approximation.  
In \cite{VehtariParallelGP}, 
the approximation \ref{post_dens_mean_llik} is considered and is instead referred to as the 
\textit{median approximation}, owing to the fact that the median formula for a log-normal random variable 
gives $\text{median}(\Exp{\llikEm[\Ndesign](\Par)}) = \Exp{\emMean[\Ndesign]{\fwd}(\Par)}$. In other words, 
the median of the likelihood emulator yields the same estimator as plugging in the mean of the 
\textit{log}-likelihood emulator.
The approximation $\llikEmMeanDens[\Ndesign](\Par)$ can be shown to be optimal in a certain  
$L^1$ sense \cite{VehtariParallelGP, StuartTeck2}. 

One approach to propagate the 
emulator uncertainty is to marginalize the likelihood with respect to the emulator. Doing so yields the approximations 
\begin{align}
\fwdEmMargDens[\Ndesign](\Par) &\Def \priorDens(\Par) \E_{\fwdEm}\left[\Exp{\llikEmFwd[\Ndesign](\Par)} \right] \label{post_dens_marg_fwd} \\
\llikEmMargDens[\Ndesign](\Par) &\Def \priorDens(\Par) \E_{\llikEm}\left[\Exp{\llikEm[\Ndesign](\Par)} \right], \label{post_dens_marg_llik}
\end{align}
with the expectation subscripts indicating integration with respect to the GP predictive distributions 
$\Gaussian(\emMean[\Ndesign]{\fwd}(\Par), \emKer[\Ndesign]{\fwd}(\Par))$ and
$\Gaussian(\emMean[\Ndesign]{\llik}(\Par), \emKer[\Ndesign]{\llik}(\Par))$, respectively. Note that these expectations are 
computed independently for each $\Par$; the GP predictive covariance structure is not considered. 
We refer to these posteriors as \textit{marginal approximations} following the terminology used in
 \cite{StuartTeck1, StuartTeck2}, which analyze the error of these approximations in the Gaussian likelihood setting. 

Marginal approximations can be justified in various ways. \todo

In the log-likelihood emulation setting, the expectation in \ref{post_dens_marg_llik} is provided by 
\Cref{llik_em_lik_emulator_moments}, which yields 
\begin{align}
\llikEmMargDens[\Ndesign](\Par) 
&\Def \priorDens(\Par) \Exp{\emMean[\Ndesign]{\llik}(\Par) + \frac{1}{2} \emKer[\Ndesign]{\llik}(\Par)} 
\label{post_dens_marg_llik_simplified} \\
&= \llikEmMeanDens[\Ndesign](\Par) \Exp{\frac{1}{2} \emKer[\Ndesign]{\llik}(\Par)}. 
\label{post_dens_marg_var_inflation}
\end{align}
The final expression \ref{post_dens_marg_var_inflation} can be interpreted as a form of posterior inflation 
in regions where the emulator is uncertain. When $\emKer[\Ndesign]{\llik}(\Par) = 0$ the marginal approximation 
reduces to the mean approximation. Otherwise, the inflation factor is an increasing function of 
$\emKer[\Ndesign]{\llik}(\Par)$, irrespective of the mean $\emMean[\Ndesign]{\llik}(\Par)$. 
While the unnormalized posterior density \ref{post_dens_marg_var_inflation}
can be plugged into off-the-shelf sampling algorithms, the uncertainty inflation term can cause highly multi-modal 
distributions which prove difficult for standard MCMC schemes. (\todo: add plot; or remove this comment here
and include a more detailed discussion of the behavior of this approximation, its differences in the llik/fwd emulator 
setting, and its comparison to the sample-based approximation). This difficulty is noted in the 
supplement of \cite{VehtariParallelGP}. 

Finally we note that generalizations of $\text{median}(\Exp{\llikEm[\Ndesign](\Par)})$ have been considered, 
resulting in the $\quantileProb$-quantile, $\quantileProb \in (0,1)$ approximations 
\begin{align}
\fwdEmQDens[\Ndesign](\Par) &\Def \priorDens(\Par) \quantile \left[\Exp{\llikEmFwd[\Ndesign](\Par)} \right] \label{post_dens_quantile_fwd} \\
\llikEmQDens[\Ndesign](\Par) &\Def \priorDens(\Par) \quantile \left[\Exp{\llikEm[\Ndesign](\Par)} \right]. \label{post_dens_quantile_llik}
\end{align}
The log-likelihood emulation case \ref{post_dens_quantile_llik} simplifies to 
\begin{align}
\llikEmQDens[\Ndesign](\Par) 
&= \priorDens(\Par) \Exp{\emMean[\Ndesign]{\llik}(\Par) + \GaussianCDF^{-1}(\quantileProb) \sqrt{\emKer[\Ndesign]{\llik}(\Par)}} 
\label{post_dens_quantile_llik_simplified} \\
&= \llikEmMeanDens[\Ndesign](\Par) \Exp{\GaussianCDF^{-1}(\quantileProb) \sqrt{\emKer[\Ndesign]{\llik}(\Par)}},
\label{post_dens_quantile_llik_var_inflation}
\end{align}
with $\GaussianCDF$ denoting the standard Gaussian distribution function. The approximate posterior 
\ref{post_dens_quantile_llik_simplified} is considered in \cite{VehtariParallelGP, quantileApprox}. Larger choices 
of $\alpha$ correspond to more conservative approximations in the sense of inducing a larger degree of posterior 
inflation in regions where the GP is uncertain. As compared with $\llikEmMeanDens[\Ndesign](\Par)$
\ref{post_dens_marg_var_inflation}, we notice that the quantile approximation \ref{post_dens_quantile_llik_var_inflation}
induces a ``softer'' form of uncertainty inflation, with the inflation factor scaling with the exponentiated GP standard 
deviation instead of the variance. 

\subsection{Approximate MCMC}
We now interpret the approximate posteriors introduced in the previous section as inducing different 
approximations of the Metropolis-Hastings (MH) acceptance probability. This perspective motivates 
an alternative approach to GP-induced posterior approximation; instead of first deriving an 
approximation of the unnormalized posterior density, we consider the approximation to the MCMC 
scheme as the starting point. In addition to deterministic approximations,
we consider a class of MCMC algorithms which produce stochastic approximations to the MH acceptance 
probability, and thus fall into the noisy MCMC 
framework which has been considered in the literature 
\cite{noisyMCMC, stabilityNoisyMH, noisyMCSurvey, pseudoMarginalMCMC}. 

\subsubsection{Deterministic Metropolis-Hastings Approximations}
We recall that the MH algorithm is defined by a proposal kernel, with density that we denote by 
$\propDens(\Par, \cdot)$. If the Markov chain is in the current state $\Par \in \parSpace$ then 
the next state is defined by sampling a proposal $\propPar \sim \propDens(\Par, \cdot)$ which is 
accepted with probability $\accProbMH(\Par, \propPar)$, defined by 

\begin{align}
&\accProbMH(\Par, \propPar) = 
\min\left\{1, \frac{\priorDens(\propPar) \Exp{\llik(\propPar)} \propDens(\propPar, \Par)}{\priorDens(\Par) \Exp{\llik(\Par)} \propDens(\Par, \propPar)} \right\}.
\label{MH_acc_prob_exact}
\end{align}
If accepted, the next state is defined to be $\propPar$, else it is set to the current state $\Par$. 
This procedure is summarized in \Cref{alg:MH}. For notational convenience in deriving approximations 
of the acceptance probability, we define 
\begin{align}
&\accProbRatio(\Par, \propPar) 
= \frac{\priorDens(\propPar) \Exp{\llik(\propPar)} \propDens(\propPar, \Par)}{\priorDens(\Par) \Exp{\llik(\Par)} \propDens(\Par, \propPar)}, 
&&\likRatio(\Par, \propPar) = \frac{\Exp{\llik(\propPar)}}{\Exp{\llik(\Par)}},
\label{MH_acc_prob_exact}
\end{align}
and refer to $\accProbRatio(\Par, \propPar)$ and $\likRatio(\Par, \propPar)$ 
as the \textit{acceptance ratio} and \textit{likelihood ratio}, respectively.

\begin{algorithm}
    \caption{Metropolis-Hastings}
    \label{alg:MH}
    \begin{algorithmic}[1] % The number tells where the line numbering should start
    \Function{MH}{$\indexMCMC[0]{\Par}, \NMCMC$}     
        \For{$\mcmcIndex \gets1,\NMCMC$} 
        		\State $\propPar \sim \propDens(\indexMCMC{\Par}, \cdot)$ \Comment{Proposal}
		\State $\accProbMH(\indexMCMC{\Par}, \propPar) \gets \min\left\{1, \frac{\priorDens(\propPar)\Exp{\llik(\propPar)}
				\propDens(\propPar, \indexMCMC{\Par})}{\priorDens(\indexMCMC{\Par}) \Exp{\llik(\indexMCMC{\Par})}  \propDens(\indexMCMC{\Par}, \propPar)} \right\}$
		\State $b \sim \text{Bernoulli}(\accProbMH(\indexMCMC{\Par}, \propPar))$
		\If{$b = 1$}
			\State $\indexMCMC[\mcmcIndex+1]{\Par} \gets \propPar$ 
		\Else
			\State $\indexMCMC[\mcmcIndex+1]{\Par} \gets \indexMCMC{\Par}$
		\EndIf
	\EndFor
	\EndFunction
    \end{algorithmic}
\end{algorithm}

We can define approximate surrogate-based MCMC algorithms by utilizing $\llikEm[\Ndesign]$ 
or $\llikEmFwd[\Ndesign]$ to approximate $\accProbMH(\Par, \propPar)$, either directly or by plugging 
in approximations of $\accProbRatio(\Par, \propPar)$ or $\likRatio(\Par, \propPar)$.  
For brevity, we focus on the log-likelihood emulation setting here, as the definitions are similar in the forward 
model emulation case. 
We thus consider the random variables $\Em[\Ndesign]{\likRatio}(\Par, \propPar)$,
$\Em[\Ndesign]{\accProbRatio}(\Par, \propPar)$, and 
$\Em[\Ndesign]{\accProbMH}(\Par, \propPar)$ defined by inserting the emulated log-likelihoods 
$\llikEm[\Ndesign](\propPar)$ and $\llikEm[\Ndesign](\Par)$ in place of their exact counterparts
in \ref{MH_acc_prob_exact}. The mean, marginal, and quantile approximate posteriors introduced in the previous 
section correspond to the following deterministic, plug-in estimates of $\Em[\Ndesign]{\likRatio}(\Par, \propPar)$:
\begin{align}
\llikEmMean[\Ndesign]{\likRatio}(\Par, \propPar) 
\coloneqq \Exp{\emMean[\Ndesign]{\llik}(\propPar) - \emMean[\Ndesign]{\llik}(\Par)} \\
\llikEmMarg[\Ndesign]{\likRatio}(\Par, \propPar) 
\Def \llikEmMean[\Ndesign]{\likRatio}(\Par, \propPar) 
\Exp{\frac{1}{2}\left[\emKer[\Ndesign]{\llik}(\propPar) - \emKer[\Ndesign]{\llik}(\Par)\right]} \\
\llikEmQ[\Ndesign]{\likRatio}(\Par, \propPar) 
\Def \llikEmMean[\Ndesign]{\likRatio}(\Par, \propPar) 
\Exp{\GaussianCDF^{-1}(\quantileProb)  \left[\sqrt{\emKer[\Ndesign]{\llik}(\propPar)} - \sqrt{\emKer[\Ndesign]{\llik}(\Par)}\right]}.
\end{align}
Notice that the marginal and quantile approximations inflate the probability of acceptance when the 
uncertainty at the proposed state $\propPar$ is greater than that at the current state $\Par$.
All three of these approximations are derived by independently approximating the numerator and denominator 
in the ratio $\Exp{\llik(\propPar)} / \Exp{\llik(\Par)}$ and then dividing the two approximations.
However, the GP emulator $\llikEm[\Ndesign]$ is a random function, meaning that 
$\llikEm[\Ndesign](\propPar)$ and $\llikEm[\Ndesign](\Par)$ are typically correlated, the covariance
being given by $\emKer[\Ndesign]{\llik}(\propPar, \Par)$. It is therefore natural to consider incorporating 
this information into approximations of the likelihood ratio. We define a second marginal approximation 
by marginalizing $\Em[\Ndesign]{\likRatio}(\Par, \propPar)$ with respect to the \textit{joint} 
distribution of $[\llikEm[\Ndesign](\propPar), \llikEm[\Ndesign](\Par)]$, which gives
\begin{align}
\llikEmJointMarg[\Ndesign]{\likRatio}(\Par, \propPar) 
&\Def \E_{\llikEm[\Ndesign]} \left[\Em[\Ndesign]{\likRatio}(\Par, \propPar) \right] \\
&= \llikEmMean[\Ndesign]{\likRatio}(\Par, \propPar)  \Exp{\frac{1}{2} \Var\left[\llikEm[\Ndesign](\propPar) - \llikEm[\Ndesign](\Par) \right]} \\
&=  \llikEmMean[\Ndesign]{\likRatio}(\Par, \propPar)  
\Exp{\frac{1}{2}\left[\emKer[\Ndesign]{\llik}(\propPar) + \emKer[\Ndesign]{\llik}(\Par) - 2\emKer[\Ndesign]{\llik}(\propPar, \Par) \right]}. 
\label{lik_ratio_joint_marg}
\end{align} 
The joint marginal approximation inflates the acceptance probability when the GP is uncertain about the 
\textit{difference} in log-likelihood values at the locations $\propPar$ and $\Par$. For example, it is possible
that the marginal uncertainty at each location is large, but a large covariance between the locations 
might imply little uncertainty in the difference $\llikEm[\Ndesign](\propPar) - \llikEm[\Ndesign](\Par)$. 

The marginal approximation can be taken a step further by integrating $\llikEm$ out of 
$\Em[\Ndesign]{\accProbRatio}(\Par, \propPar)$ or $\Em[\Ndesign]{\accProbMH}(\Par, \propPar)$ instead of 
$\Em[\Ndesign]{\likRatio}(\Par, \propPar)$. The former case 
\begin{align}
\llikEmJointMarg[\Ndesign]{\accProbRatio}(\Par, \propPar) 
&= \frac{\priorDens(\propPar) \propDens(\propPar, \Par)}{\priorDens(\Par) \propDens(\Par, \propPar)} 
\llikEmJointMarg[\Ndesign]{\likRatio}(\Par, \propPar)
\label{acc_ratio_joint_marg}
\end{align}
follows immediately from \ref{lik_ratio_joint_marg}, 
given that $\accProbRatio(\Par, \propPar)$ differs from $\likRatio(\Par, \propPar)$ only by a multiplicative 
constant (with respect to the emulator). Thus, the approximations 
$\llikEmJointMarg[\Ndesign]{\accProbRatio}(\Par, \propPar)$ and 
$\llikEmJointMarg[\Ndesign]{\likRatio}(\Par, \propPar)$ induce identical inference algorithms. 
Marginalizing the acceptance probability directly instead yields 
\begin{align}
\llikEmJointMarg[\Ndesign]{\accProbMH}(\Par, \propPar) 
&\Def \E_{\llikEm[\Ndesign]} \left[\min\left\{1, \Em[\Ndesign]{\accProbRatio}(\Par, \propPar)\right\} \right] \\
&= w_1(\Par, \propPar) + w_2(\Par, \propPar) \llikEmJointMarg[\Ndesign]{\accProbRatio}(\Par, \propPar),
\label{acc_prob_joint_marg}
\end{align} 
where 
\begin{align*}
w_1(\Par, \propPar) &\Def \Prob(\Em[\Ndesign]{\accProbRatio}(\Par, \propPar) \geq 1) \\
w_2(\Par, \propPar) &\Def \Prob\left(\Em[\Ndesign]{\accProbRatio}(\Par, \propPar) \leq \Exp{-\Var\left[\llikEm[\Ndesign](\propPar) - \llikEm[\Ndesign](\Par) \right]}\right).
\end{align*}
This is derived in \Cref{prop:joint-marg-accept-prob} in the appendix. While the plug-in acceptance probability 
approximation based on $\llikEmJointMarg[\Ndesign]{\accProbRatio}(\Par, \propPar)$ yields 
$\min\left\{1, \llikEmJointMarg[\Ndesign]{\accProbRatio}(\Par, \propPar) \right\}$, 
$\llikEmJointMarg[\Ndesign]{\accProbMH}(\Par, \propPar)$ instead takes the form of a linear combination of 
$1$ and $\llikEmJointMarg[\Ndesign]{\accProbRatio}(\Par, \propPar)$, with weights given by the probabilities 
$w_1(\Par, \propPar)$, $w_2(\Par, \propPar)$.

\subsubsection{Noisy Metropolis-Hastings Approximations}
The previous section considered deterministic approximations to the MH acceptance probability. Here we turn 
to random approximations, thus entering the realm of so-called ``noisy'' MCMC algorithms 
\cite{noisyMCMC, stabilityNoisyMH, noisyMCSurvey, pseudoMarginalMCMC}. The general framework,
summarized in \Cref{alg:noisy-MH}, is characterized by the choice of distribution $\llikSampDist_{\Par, \propPar}$
from which to sample log-likelihood values $\llikSamp$ and $\llikSampProp$ which replace the exact 
log-likelihood evaluations $\llik(\Par)$ and $\llik(\propPar)$ when computing the MH acceptance probability. 
For brevity we will continue to specialize the definitions to the log-likelihood emulation setting, but the forward model 
emulation case is nearly identical, with $\llikEmFwd$ replacing $\llikEm$. 

\begin{algorithm}
    \caption{Noisy Metropolis-Hastings}
    \label{alg:noisy-MH}
    \begin{algorithmic}[1] % The number tells where the line numbering should start
    \Function{noisyMH}{$\indexMCMC[0]{\Par}, \NMCMC, \propDens, \llikSampDist$}     
        \For{$\mcmcIndex \gets1,\NMCMC$} 
        		\State $\propPar \sim \propDens(\indexMCMC{\Par}, \cdot)$ 
		\State $(\llikSamp, \llikSampProp) \sim \llikSampDist_{\indexMCMC{\Par}, \propPar}$  \Comment{Sample log-likelihood values}
		\State $\accProbMH(\indexMCMC{\Par}, \propPar) \gets \min\left\{1, \frac{\priorDens(\propPar)e^{\llikSampProp} 
				\propDens(\propPar, \indexMCMC{\Par})}{\priorDens(\indexMCMC{\Par}) e^{\llikSamp}  \propDens(\indexMCMC{\Par}, \propPar)} \right\}$
		\State $b \sim \text{Bernoulli}(\accProbMH(\indexMCMC{\Par}, \propPar))$
		\If{$b = 1$}
			\State $\indexMCMC[\mcmcIndex+1]{\Par} \gets \propPar$ 
		\Else
			\State $\indexMCMC[\mcmcIndex+1]{\Par} \gets \indexMCMC{\Par}$
		\EndIf
	\EndFor
	\EndFunction
    \end{algorithmic}
\end{algorithm}

We define the following noisy MCMC algorithms based on different choices for $\llikSampDist_{\Par, \propPar}$.
Below we consider various choices, and show that the transition kernel defining the Markov chain of the 
respective algorithm coincides with that of one of the deterministic approximations considered in the previous 
section. Thus, the noisy MCMC algorithms provide alternative inference algorithms for sampling from the previously 
considered approximate posteriors. Moreover, in the forward model emulation setting, they also generalize to 
situations in which the deterministic approximations yield intractable integrals. For example, these algorithms are 
not restricted to the setting of a Gaussian likelihood when using forward model emulators. 
We use the notation $\parMat \Def \{\Par, \propPar\}$ in the below definitions. 

\paragraph{Joint Monte Carlo Within Metropolis-Hastings.}
We first consider setting $\llikSampDist_{\Par, \propPar}$ to be the joint distribution of 
$[\llikEm[\Ndesign](\Par), \llikEm[\Ndesign](\propPar)]$, 
\begin{align}
\llikSampDist_{\Par, \propPar} \Def \Gaussian(\emMean[\Ndesign]{\llik}(\parMat), \emKer[\Ndesign]{\llik}(\parMat)).
\label{llik_dist_MCWMH-joint}
\end{align}
We call the resulting algorithm \textit{Joint Monte Carlo within Metropolis-Hastings} ($\mcwmhJointLabel$). 
Like the deterministic joint marginal approximations \ref{lik_ratio_joint_marg} and \ref{acc_prob_joint_marg}, 
the $\mcwmhJointLabel$ algorithm takes into account the GP covariance structure. Note that since the log-likelihood
values are sampled independently across iterations, the $\mcwmhJointLabel$ algorithm does indeed define a valid 
Markov chain. The transition kernel of this Markov chain is given by 
\footnote{We denote by $\delta_{x}$ the Dirac measure centered at a point $x$, which assigns a unit point mass to $x$.}
\begin{align}
\llikEmJointMCWMH[\Ndesign]{\MarkovKernel}(\Par, A) 
&= \int_{A} \propDens(\Par, \propPar) \llikEmJointMarg[\Ndesign]{\accProbMH}(\Par, \propPar) d\propPar 
+ \delta_{\Par}(A) \left[1 - \int_{\parSpace} \propDens(\Par, \propPar) \llikEmJointMarg[\Ndesign]{\accProbMH}(\Par, \propPar) d\propPar \right],
\label{MCWMH-joint-kernel}
\end{align}
which is derived in the appendix, \Cref{transition_kernel_derivations}. This is precisely the transition kernel of the Markov chain 
defined by the deterministic joint marginal approximation of the acceptance probability \ref{acc_prob_joint_marg}. We note that 
one might also consider setting 
$\llikSampDist_{\Par, \propPar} 
\Def \Gaussian\left(\emMean[\Ndesign]{\llik}(\parMat), \diag\left\{\emKer[\Ndesign]{\llik}(\Par), \emKer[\Ndesign]{\llik}(\propPar) \right\}\right)$; 
that is, ignore the GP covariance structure. We refer to the resulting algorithm as
\textit{Independent Monte Carlo Within Metropolis-Hastings} ($\mcwmhIndLabel$) and consider this choice as a basis for 
comparison to assess the influence of incorporating the GP covariance in the approximate posterior [\todo]. 
An algorithm similar to $\mcwmhIndLabel$ is utilized to calibrate expensive ecosystem models 
in \cite{FerEmulation}. 

\paragraph{Pseudo-Marginal Metropolis-Hastings.}
Note that the two previous algorithms produce new log-likelihood samples at both the current $\Par$ and proposed 
$\propPar$ locations at each iteration of the algorithm. An alternative is to draw a new sample only at the 
proposed location, while recycling the sample at the current location from the previous iteration. Formally, this procedure
implies the product measure 
\begin{align}
\llikSampDist_{\Par, \propPar} 
= \delta_{\llikSamp} \otimes  
\Gaussian(\emMean[\Ndesign]{\llik}(\propPar), \emKer[\Ndesign]{\llik}(\propPar)), \label{llik_dist_pseudo_marg}
\end{align}
which precisely falls into the pseudo-marginal MCMC framework \cite{pseudoMarginalMCMC}. 
We thus term 
this algorithm \textit{Pseudo-Marginal Metropolis-Hastings} ($\mhPseudoMargLabel$). Given a log-likelihood sample 
$\llikSamp$, then by definition the unnormalized posterior approximation $\priorDens(\Par) \Exp{\llikSamp}$ is 
an unbiased estimator of 
$\llikEmMargDens[\Ndesign](\Par) = \E_{\llikEm[\Ndesign]}\left[\priorDens(\Par) \Exp{\llikEm[\Ndesign](\Par)} \right]$. 
This implies that the $\mhPseudoMargLabel$ algorithm defines a Markov chain with target distribution 
$\llikEmMargDens[\Ndesign]$, which was already defined in \ref{post_dens_marg_llik}. 

% Comparing to the Sample-Based Approximation
\subsection{Comparing to the Sample-Based Approximation}
\todo: need to be clear on definitions here, clarifying the difference between the random measure, and 
$\Par$ samples that are generated from sample paths of the random measure. The random measure 
essentially defines a Markov kernel that can be used to generate $u$ samples. 
In this subsection, we take the viewpoint of treating the \textit{sample-based} approximation 
(\ref{post_dens_random_fwd} and \ref{post_dens_random_llik}) as a baseline 
that the other methods seek to approximate. In some sense, this is the method that fully propagates 
the full GP distribution. In theory, samples from this approximate posterior can be obtained generatively by 
the following procedure:
\begin{enumerate}
\item Sample a trajectory (i.e., sample path) from the GP $\llikEm[\Ndesign]$, which induces a sample 
path of the random density $\llikEmRdm[\Ndesign]{\postDensNorm}$.
\item Draw a sample $u$ from the sampled density.
\end{enumerate}
A similar procedure applies to the forward model emulation setting. This procedure is not realizable in 
practice due to the requirement of sampling a GP trajectory. An alternative view of this approach comes 
from considering the density of the sample $u$ generated by the above procedure:
\begin{align}
\E_{\llikEm}\left[\frac{\priorDens(\Par) \Exp{\llikEm[\Ndesign](\Par)}}{\int \priorDens(\Par) \Exp{\llik Em[\Ndesign](\Par)} d\Par} \right]
\end{align}


% Perspectives on Posteriors Induced by Gaussian Process Surrogates 
\subsection{Perspectives on Posteriors Induced by Gaussian Process Surrogates}
In the preceding subsections, we have summarized a wide variety of GP-induced posteriors that have been considered 
in the literature, as well as proposed a new approach based on a noisy approximation of a MH algorithm. This leads to 
the natural question: \textit{which is the correct posterior approximation to use?} We argue that the answer to this question 
depends largely on the goal at hand. The existing literature in this area tends to implicitly favor one of two potential 
viewpoints: (1) forward uncertainty propagation or; (2) posterior error minimization. The former perspective tends to 
accept that computation is limited and hence completely reducing surrogate-induced error is infeasible. The goal thus 
becomes to ``honestly'' propagate the surrogate uncertainty with the goal of achieving reasonable posterior coverage. 
The latter perspective emphasizes minimizing 
error with respect to the true posterior, and ensuring posterior convergence as the number of design points increases
(surrogate refinement via the selection of additional design points is the topic of  \Cref{seq_des}). We of course should 
hope for the ideal, whereby computation-limited approximations exhibit favorable 
uncertainty calibration in some sense, while also converging towards the true posterior in the event that more 
computational resources become available. Various papers have theoretically studied the latter property, in which results 
take the form of posterior consistency statements as $\Ndesign \to \infty$
 \citep{StuartTeck1,StuartTeck2,random_fwd_models,gp_surrogates_random_exploration}. To our knowledge, there is 
 currently no analogous theoretical framework for comparing the quality of various GP-induced posterior approximations
 from an uncertainty calibration perspective. We believe that such a framework should evaluate posterior calibration with 
 respect to the calibration of the GP emulators. If the underlying GP fit is poor, then one would expect the resulting posterior 
 calibration to be poor; the quality of the posterior approximation method should not be conflated with the quality of the 
 underlying GP in this setting. We leave the development of a theoretical framework addressing such questions to future 
 work. In the present article, we take care to assess the various posterior approximation methods using a variety of GP 
 models with different characteristics. 
 
\todo: Add comments on the effect of approximating the unnormalized posterior density. 


% Existence of approximate posteriors. 
\subsection{Existence of Approximate Posterior Distributions} \label{post_existence}
\todo

% Practical Gaussian Process Considerations
\section{Practical Gaussian Process Considerations}
\begin{itemize}
\item Examples from literature: quadratic trend, paper that uses constrained GP, Teckentrup GPs for PDEs
\item Issues with quadratic trend in higher dimensions. 
\item Issues with existence for priors with unbounded support: issue with ordinary kriging, marginalizing mean 
uncertainty, quadratic kernel. 
\item Potential for constrained GPs.
\end{itemize}

% Sequential Design
\section{Sequential Design} \label{seq_des}
In applications, the GP emulator is often constructed sequentially. In this section, we consider augmenting the 
initial design $\designIn[\Ndesign]$ with new input points that are chosen to yield maximal improvement in the 
GP-induced posterior approximation. This can be thought of as a question of experimental design for the 
solution of a Bayesian inverse problem. We begin by reviewing the relevant background on GP-based 
sequential design, and then proceed to discuss design methods that are tailored to the goal of 
posterior approximation.  

\subsection{Background: Sequential Design for Gaussian Processes}

\subsubsection{The Sequential Design Loop}
Sequential design algorithms for GPs have been studied extensively across several fields. We introduce this topic 
for a generic GP predictive distribution $\funcEm[\Ndesign] \Def \funcPrior|[\funcPrior(\designIn) = \func(\designIn)]$, as defined in 
\ref{generic_gp_conditional}. We now consider the question of identifying a new set of $\Nbatch$
inputs $\designBatchIn$ to augment the initial design. The augmented design is thus given by 
$\{\designIn[\Naugment], \func(\designIn[\Naugment])\}$, where 
$\designIn[\Naugment] \Def \designIn[\Ndesign] \cup \designBatchIn$. 
Conditioning on the augmented design results in the updated GP
\begin{align}
\funcEm[\Naugment] \Def \funcPrior|[\funcPrior(\designIn[\Naugment]) = \func(\designIn[\Naugment])] \sim \GP(\gpMean[\Naugment], \gpKer[\Naugment]),
\end{align}
where the predictive mean $\gpMean[\Naugment]$ and covariance $\gpKer[\Naugment]$ functions are given by \ref{kriging_eqns}
with $\{\designIn, \func(\designIn)\}$ replaced by $\{\designIn[\Naugment], \func(\designIn[\Naugment])\}$. 
We note that conditioning in stages is equivalent to conditioning on the entire design at once; i.e.,
$\funcPrior|[\funcPrior(\designIn[\Naugment]) = \func(\designIn[\Naugment])] \eqDist 
\funcEm[\Ndesign]|[\funcEm[\Ndesign](\designIn[\Nbatch]) = \func(\designBatchIn)]$, which is justified 
by \Cref{lemma:gp-condition-order} in the appendix. 
The update $\funcEm[\Ndesign] \to \funcEm[\Naugment]$ 
can be computed in $\BigO(\Nbatch^3 + \Nbatch^2 \Ndesign + \Nbatch \Ndesign^2)$ operations, 
a significant improvement over the naive $\BigO([\Ndesign + \Nbatch]^3)$
implementation (courtesy of the partitioned matrix inverse identity stated in \Cref{partitioned-matrix-inverse})
[\todo: probably will want an entire section on algorithm complexity, and will want to better explain where 
these Big O results come from]. 
Naturally, the new batch $\designBatchIn$ should be chosen to 
yield maximal improvement in the GP model with respect to the downstream task that is to be performed. 
This idea is formalized by an optimization problem of the form 
\begin{align}
\designBatchIn = \argmin_{\designBatchIn^\prime \subset \parSpace} \acq[\Ndesign](\designBatchIn^\prime), \label{acq_func_opt}
\end{align}
where $\acq[\Ndesign]: \parSpace^{\Nbatch} \to \R$ is an \textit{acquisition function} (i.e., \textit{design criterion}) that 
encodes the quality of a batch of design points in terms of the task at hand. Iterating this procedure $\Nrounds$ times yields the so-called 
\textit{sequential design} (a.k.a., \textit{active learning}) loop, summarized in \Cref{alg:seq_des_loop}.  

\begin{algorithm}
    \caption{Gaussian Process Sequential Design Loop}
    \label{alg:seq_des_loop}
    \begin{algorithmic}[1] % The number tells where the line numbering should start
    \Function{SeqDesign}{$\func, \funcPrior, \designIn[\Ndesign], \Nrounds$}     
	\State $\hat{\func} \gets \funcPrior|[\funcPrior(\designIn[\Ndesign]) = \func(\designIn[\Ndesign])] \sim \GP(\gpMean[\Ndesign], \gpKer[\Ndesign])$
	\Comment{Initial Design}
        \For{$\designIndex \gets 1,\Nrounds$} \Comment{Sequential Design Loop}
        		\State $\designBatchIn \gets \argmin_{\designBatchIn^\prime \subset \parSpace} \acq[\Ndesign](\designBatchIn^\prime)$ 
		\State $\hat{f} \gets \hat{f} | [\hat{f}(\designBatchIn) = \func(\designBatchIn)]$
	\EndFor
	\EndFunction
    \end{algorithmic}
\end{algorithm}

One could of course consider 
varying the batch size $\Nbatch$ across the $\Nrounds$ iterations, but to simplify notation we keep $\Nbatch$ constant. We note that the 
special case $\Nbatch = 1$ corresponds to the pure sequential design setting, in which the function $\func$ is evaluated at each acquired 
point before considering the subsequent acquisition. The \textit{batch} sequential design setting $\Nbatch > 1$ is more challenging,  
owing to the fact that $\func$ is only evaluated after an entire batch $\designBatchIn$ has been acquired, meaning that inputs must 
be selected without observing the outputs of the other points in the batch. 
We note that, while the design framework summarized in \Cref{alg:seq_des_loop} does allow for consideration 
of within-batch interactions between design points, 
it is still ``myopic'' in the sense that it disregards the potential for acquisitions in future rounds. 
Non-myopic strategies have been considered
through a dynamic programming lens, though they typically come at the cost of significant 
computational expense \cite{SURThesis, supermartingaleSUR}.


\subsubsection{Acquisition Functions}
We now summarize some particular choices of acquisition function $\acq[\Ndesign]$ which are popular in practice. These will form 
the basis for extensions that target the specific goal of GP-induced posterior approximation. We begin by emphasizing that the 
choice of acquisition function should reflect how the GP is being used in downstream tasks. Perhaps the most common goal is that 
of optimization. The field of \textit{Bayesian optimization} is dedicated to this question, which has produced a wide variety
acquisition functions that are tailored to the task of optimizing expensive black-box functions $\func$; 
see \cite{reviewBayesOpt, gramacy2020surrogates} for in-depth reviews.  
The related field of \textit{Bayesian quadrature} \cite{BayesQuadrature, BayesQuadrature, BayesQuadRatios}
instead focuses on the goal of approximating integrals with expensive integrands. The computer experiments community 
focuses on various tasks, including contour estimation and reliability analysis \cite{contourEstimation, cole2021entropybased}. 
Perhaps the most common goal considered in the sequential design of computer experiments is to fit a response surface
(see \cite{gramacy2020surrogates, SanterCompExp, design_analysis_computer_experiments} for reviews), 
which typically leads to design procedures 
that seek to ``evenly'' fill the input space $\parSpace$ in some sense.  
In such settings, a reasonable approach is to choose points in locations where the GP is most uncertain. 
Defining ``uncertainty'' in terms 
of variance or entropy yields the \textit{maximum variance} ($\maxvarLabel$) and \textit{maximum entropy}
($\maxentLabel$) acquisition functions, respectively, 
\begin{align}
&\labelAcq[\Ndesign]{\maxvarLabel}(\Par) \Def -\Var[\funcEm[\Ndesign](\Par)]  =  -\gpKer(\Par) \\ 
&\labelAcq[\Ndesign]{\maxentLabel}(\designBatchIn) \Def -\Ent[\funcEm[\Ndesign](\designBatchIn)] = -\frac{1}{2} \log\det(2\pi e \gpKer(\designBatchIn)), \label{acq_max_var_ent}
\end{align} 
where $\Ent[Z]$ denotes the entropy of a random variable $Z$. The variance and entropy are negated to align with our convention 
of minimizing acquisition functions. The $\maxvarLabel$ criterion does not have an obvious generalization to the batch setting, 
so is defined to only accept a single input $\Par$. A noted weakness of these criteria is their locality; 
they target the single point (sets) with maximal uncertainty, ignoring the effect of the acquisition on the remainder of the design space. 
This often favors points at the boundaries of $\parSpace$, behavior that is typically not desirable \cite{gramacy2020surrogates}. 
To address this weakness, an alternative approach is to minimize the \textit{conditional} uncertainty 
$\Var[\funcEm[\Ndesign](\Par) | \designBatchIn]$ or $\Ent[\funcEm[\Ndesign](\Par) | \designBatchIn]$), averaged over 
the input space $\parSpace$. The resulting \textit{integrated uncertainty criteria}
\begin{align}
\labelAcq[\Ndesign]{\intvarLabel}(\designBatchIn)
&\Def \int_{\parSpace} \Var[\funcEm[\Ndesign](\Par) | \funcEm(\designBatchIn)=\func(\designBatchIn)] \weightDens(\Par) d\Par
= \int_{\parSpace} \gpKer[\Naugment](\Par) \weightDens(\Par) d\Par, \label{acq_int_var} \\ 
\labelAcq[\Ndesign]{\intentLabel}(\designBatchIn)
&\Def \int_{\parSpace} \Ent[\funcEm[\Ndesign](\Par) | \funcEm(\designBatchIn)=\func(\designBatchIn)] \weightDens(\Par) d\Par
= \frac{1}{2} \int_{\parSpace} \log(2\pi e \gpKer[\Naugment](\Par)) \weightDens(\Par) d\Par, \label{acq_int_ent}
\end{align}
are naturally defined in the batch setting and account for the global effect of acquiring new design points. Here $\weightDens$ is 
a probability density on $\parSpace$, which can be chosen to favor certain portions of the design space. 
A nice summary and analysis of integrated variance criteria of the form \ref{acq_int_var} is given 
in \cite{Mercer_kernels_IVAR}, with an emphasis on one-shot, rather than sequential, design. 
The closed-form expressions for the integrands in \ref{acq_int_var} and \ref{acq_int_ent} follow directly from the Gaussian conditioning 
formulas \ref{kriging_eqns} and known expression for Gaussian entropy. 
However, outside of special choices for $\parSpace$ and 
$\weightDens$ (see, e.g., \cite{Binois_2018}), the integrals (with respect to $\Par$) are typically replaced with Monte Carlo 
approximations. It is worth noting that 
these acquisition functions must be computed without observing the function evaluations $\func(\designBatchIn)$, which is 
not an issue for any of the above acquisitions given that $\gpKer[\Naugment](\Par)$ is independent of $\func(\designBatchIn)$ 
(see \ref{kriging_eqns}). However, the predictive mean $\gpMean[\Naugment](\Par)$ does depend on $\func(\designBatchIn)$, 
a fact which will become relevant in the subsequent section. We therefore state a lemma below that summarizes the 
dependence of the predictive mean on $\func(\designBatchIn)$, under the assumption that the latter is distributed according 
to the current GP $\funcEm[\Ndesign]$. A similar result is provided in \cite{VehtariParallelGP}, and a proof is given in 
the appendix.
\begin{lemma} \label{lemma:pred-mean-dist}
Let $\{\designBatchIn, \funcVal[\Nbatch]\}$ be a new batch of $\Nbatch$ design points, with
$\designBatchFunc | \designBatchIn \sim \Gaussian(\gpMean[\Ndesign](\designBatchIn), \gpKer[\Ndesign](\designBatchIn))$. 
Let  
\begin{align}
\gpMean[\Ndesign](\Par | \designBatchFunc) 
\Def \E[\funcEm[\Ndesign](\Par) | \funcEm[\Ndesign](\designBatchIn) = \designBatchFunc] \label{conditional-mean-notation}
\end{align}
denote the predictive mean after conditioning $\funcEm[\Ndesign]$ on 
$\{\designBatchIn, \funcVal[\Nbatch]\}$. Then, viewed as a function of the random variable 
$\designBatchFunc$, the conditional mean $\gpMean[\Ndesign](\Par| \designBatchFunc)$ is 
distributed as 
\begin{align}
\gpMean[\Ndesign](\Par | \designBatchFunc)
\sim \Gaussian\left(\gpMean[\Ndesign](\Par), 
                               \gpKer[\Ndesign](\Par) - \gpKer[\Naugment](\Par) \right) \label{pred-mean-dist}.
\end{align}
\end{lemma}
We see that the expectation (with respect to $\designBatchFunc$) of $\gpMean[\Ndesign](\Par| \designBatchFunc)$ 
is equal to the current predictive mean $\gpMean[\Ndesign](\Par)$. Since 
$\gpMean[\Ndesign](\Par) = \gpMean[\Ndesign](\Par | \gpMean[\Ndesign](\designBatchIn))$, this can also be 
interpreted as the predictive mean resulting from the conditioning of $\funcEm[\Ndesign]$ on the ``pseudo-data''
$\{\designBatchIn, \gpMean[\Ndesign](\designBatchIn)\}$; i.e., treating the current GP predictive mean as if it were 
the true response. The variance of 
$\gpMean[\Ndesign](\Par | \designBatchFunc)$ is equal to the change in GP variance at the input $\Par$ due to the act 
of conditioning on the batch $\designBatchIn$. Therefore, the uncertainty in $\gpMean[\Ndesign](\Par | \designBatchFunc)$ 
is large when the design points $\designBatchIn$ are more ``influential.''

\subsection{Acquisition Functions for Posterior Approximation}
The previous section introduced several acquisition functions that seek to improve GP predictions by targeting
design point batches that cause large reductions of uncertainty in the GP predictive distribution. While such criteria may be appropriate 
when quality predictions are required across all of $\parSpace$, this may not align well with the goal of constructing a GP emulator 
that induces an accurate posterior approximation. In the Bayesian inverse problem setting, it has repeatedly been noted that the 
posterior is often highly concentrated relative to the prior [\todo: citations]. Intuitively, one would therefore expect that the GP emulator need only 
achieve a high predictive accuracy in the region with high posterior mass, and avoid wasteful acquisition of design points in 
regions with negligible mass. This intuition is placed on more rigorous ground in \cite{StuartTeck2}, in which bounds on the 
posterior error are given in terms of the likelihood approximation error in a $L^2_{\postDens}(\parSpace)$ sense. The authors 
stated motivation was to provide rigorous justification for the use of ensemble Kalman methods as a cheap experimental 
design method for the training of GP emulators \cite{CES, idealizedGCM}. 

We instead focus on design methods that fit within the sequential design framework summarized in \Cref{alg:seq_des_loop}, and 
are tailored to the goal of posterior approximation. One such approach is to define acquisition functions based on 
the unnormalized posterior density emulator ($\fwdEmRdm[\Ndesign]{\postDens}(\Par)$ or $\llikEmRdmDens[\Ndesign](\Par)$), 
rather than the underlying GPs themselves. The problem of GP-based sequential design for Bayesian inverse problems  
has gained recent attention, and is addressed in 
\cite{SinsbeckNowak, Surer2023sequential, VehtariParallelGP, briol2017sampling, ranjan2016inverse, 
	landslideCalibration, KandasamyActiveLearning2015, Kandasamy_2017, wang2018adaptive,   
	weightedIVAR, quantileApprox, hydrologicalModel, briol2017sampling}. 
[\todo: need to look at the method used in \cite{quantileApprox, hydrologicalModel} more closely.]
As an initial example, we consider tailoring the $\maxvarLabel$ and $\maxentLabel$ criteria to the task of posterior approximation 
using a log-likelihood emulator. The natural analogs of these acquisition functions in this setting are 
\begin{align}
&\labelAcq[\Ndesign]{\maxexpvarLabel}(\Par) \Def -\Var[\llikEmRdmDens[\Ndesign](\Par)] 
= \left[\Exp{\emKer[\Ndesign]{\llik}(\Par)} - 1 \right] \Exp{2\emMean[\Ndesign]{\llik}(\Par) + \emKer[\Ndesign]{\llik}(\Par)}, \\ 
&\labelAcq[\Ndesign]{\maxexpentLabel}(\designBatchIn) \Def -\Ent[\llikEmRdmDens[\Ndesign](\designBatchIn)]
= \todo. \label{acq_max_exp_var_ent}
\end{align} 

\begin{figure}
     \centering
     \begin{subfigure}[b]{0.48\textwidth} % Pointwise acquisitions [fwd]
         \centering
         \includegraphics[width=\textwidth]{{test_1d_visualization/acq_pw_fwd.png}}
         \caption{$\acq[\Ndesign](\Par)$, with respect to $\fwdEm$}
         \label{fig:y equals x}
     \end{subfigure}
     \hfill
     \begin{subfigure}[b]{0.48\textwidth} % Pointwise acquisitions [llik]
         \centering
         \includegraphics[width=\textwidth]{{test_1d_visualization/acq_pw_llik.png}}
         \caption{$\acq[\Ndesign](\Par)$, with respect to $\llikEm$}
         \label{fig:three sin x}
     \end{subfigure}
     \caption{Pointwise maximum uncertainty acquisition functions evaluated at a grid of input values computed
        using the emulators in \Cref{fig:em_dist_1d}, a continuation of the example detailed in \Cref{vsem_1d}. 
        The left and right figures plot the approximations derived from the forward model emulator and log-likelihood 
        emulator, respectively. Solid lines correspond to the acquisition functions in \Cref{acq_max_var_ent}; i.e., 
        those are defined with respect to the underlying GP. Dashed lines correspond to the acquisition functions in 
        \Cref{acq_max_exp_var_ent}; i.e., the analogous criteria tailored to the goal of posterior approximation.
        Vertical dashed red lines indicate design input locations, and square markers correspond to the minimum of 
        each acquisition function computed over the input grid.} 
        \label{fig:pw_max_uncertainty_1d}
\end{figure}

The closed-form expressions given follow from the Gaussian conditioning identities \ref{kriging_eqns}, the log-normal 
variance expression in \Cref{llik_em_lik_emulator_moments}, and the known expression for log-normal entropy. 
The ``exp'' in the naming convention highlights that these criteria are being applied to
$\llikEmRdmDens[\Ndesign](\Par) = \priorDens(\Par) \Exp{\llikEm(\Par)}$, which is an exponentiated GP. 
Thus, many of the acquisition functions 
considered here can be viewed as the analogs of existing GP acquisition functions instead applied to log-normal 
processes (LNP). The $\maxexpvarLabel$ acquisition is considered in 
\cite{KandasamyActiveLearning2015, Kandasamy_2017}, 
while $\maxexpentLabel$ is utilized in \cite{wang2018adaptive, landslideCalibration}. 

% Integrated Expected Variance
\subsubsection{Integrated Expected Variance}
We now focus on adapting the integrated variance criterion \ref{acq_int_var} to target the goal of posterior approximation.  
In the log-likelihood emulation setting, we will be able to derive a closed-form integrand, as was the case in \ref{acq_int_var}. 
In the special case of a Gaussian likelihood, the analogous criterion in the forward model emulation setting also admits a 
closed-form integrand.

\begin{figure}
     \centering
     \begin{subfigure}[b]{0.48\textwidth} % Integrated acquisitions [fwd]
         \centering
         \includegraphics[width=\textwidth]{{test_1d_visualization/acq_int_fwd.png}}
         \caption{$\acq[\Ndesign](\Par)$, with respect to $\fwdEm$}
         \label{fig:y equals x}
     \end{subfigure}
     \hfill
     \begin{subfigure}[b]{0.48\textwidth} % Integrated acquisitions [llik]
         \centering
         \includegraphics[width=\textwidth]{{test_1d_visualization/acq_int_llik.png}}
         \caption{$\acq[\Ndesign](\Par)$, with respect to $\llikEm$}
         \label{fig:three sin x}
     \end{subfigure}
     \caption{Integrated conditional uncertainty acquisition functions evaluated at a grid of input values computed
        using the emulators in \Cref{fig:em_dist_1d}, a continuation of the example detailed in \Cref{vsem_1d}. 
        The left and right figures plot the approximations derived from the forward model emulator and log-likelihood 
        emulator, respectively. Solid lines correspond to the acquisition functions \ref{acq_int_var} and \ref{acq_int_ent};
        i.e., those that are defined with respect to the underlying GP. Dashed lines correspond to the acquisition functions
        \ref{def:int_exp_var_llik} and \ref{fwd-int-exp-var}; i.e., the analogous criteria tailored to the goal of posterior approximation.
        Vertical dashed red lines indicate design input locations, and square markers correspond to the minimum of 
        each acquisition function computed over the input grid.
        \todo: need to define the integrated entropy acquisition targeting posterior approximation.} 
        \label{fig:pw_max_uncertainty_1d}
\end{figure}

\paragraph{Log-Likelihood Emulation}
We begin with the log-likelihood emulation case. Recall that the 
$\intvarLabel$ criterion relied on the GP property that the predictive variance $\emKer[\Naugment]{\llik}(\Par)$
does not depend on the response $\llik(\designBatchIn)$. This is no longer the case for 
$\Var[\llikEmRdmDens[\Naugment](\Par)]$, given that $\llikEmRdmDens[\Naugment](\Par)$ is a log-normal 
random variable and hence its variance depends on $\emMean[\Naugment]{\llik}(\Par)$,  
which in turn depends on $\llik(\designBatchIn)$ (see \ref{kriging_eqns}). 
We follow the approach of \cite{VehtariParallelGP} and assume that the unobserved response follows the 
current GP predictive distribution; that is, we consider conditioning on a new design batch 
$\{\designBatchIn, \designBatchLlik\}$ which is assumed 
to satisfy 
\begin{align}
\designBatchLlik | \designBatchIn &\sim \Gaussian(\emMean[\Ndesign]{\llik}(\designBatchIn), \emKer[\Ndesign]{\llik}(\designBatchIn)).
\end{align}
Following the notation in \Cref{conditional-mean-notation}, we will denote 
\begin{align}
\emMean[\Ndesign]{\llik}(\Par | \designBatchLlik)
&\Def \E\left[\llikEmRdmDens[\Ndesign](\Par) | \llikEm[\Ndesign](\designBatchIn) = \designBatchLlik \right]. 
\end{align}
The dependence of $\emMean[\Ndesign]{\llik}(\Par | \designBatchLlik)$ on $\designBatchLlik$ is given in 
\Cref{lemma:pred-mean-dist}. This lemma allows us to average over the uncertainty in  
$\Var[\llikEmRdmDens[\Ndesign](\Par) | \llikEm[\Ndesign](\designBatchIn) = \designBatchLlik]$ stemming from the
 random quantity $\designBatchLlik$. 
 
\begin{lemma} \label{lemma:evar}
Let $\{\designBatchIn, \designBatchLlik\}$ be a new batch of $\Nbatch$ design points, with 
$\designBatchLlik | \designBatchIn \sim \Gaussian(\emMean[\Ndesign]{\llik}(\designBatchIn), \emKer[\Ndesign]{\llik}(\designBatchIn))$. 
For any $\Par \in \parSpace$, it then follows that 
\begin{align}
\E_{\designBatchLlik} \Var[\llikEmRdmDens[\Ndesign](\Par) | \llikEm[\Ndesign](\designBatchIn) = \designBatchLlik]
&= \Var\left[\llikEmRdmDens[\Ndesign](\Par) | \llikEm[\Ndesign](\designBatchIn) = \emMean[\Ndesign]{\llik}(\designBatchIn)\right]
	\varInflation(\Par; \designBatchIn), \label{evar-expression}
\end{align}
where 
\begin{align}
\Var\left[\llikEmRdmDens[\Ndesign](\Par) | \llikEm[\Ndesign](\designBatchIn) = \emMean[\Ndesign]{\llik}(\designBatchIn)\right]
&= \priorDens(\Par)^2 \Exp{2\emMean[\Ndesign]{\llik}\left(\Par\right) + \emKer[\Naugment]{\llik}(\Par)} 
\left[\Exp{\emKer[\Naugment]{\llik}(\Par)} - 1 \right], \nonumber \\
\varInflation(\Par; \designBatchIn)
&= \Exp{2\left(\emKer[\Ndesign]{\llik}(\Par) - \emKer[\Naugment]{\llik}(\Par)\right)}. \label{var_inflation_factor}
\end{align}
\end{lemma}

A similar result is derived in \cite{VehtariParallelGP}.
The interpretation of \Cref{lemma:evar} is similar to that of \Cref{lemma:pred-mean-dist}. 
The first term in \ref{evar-expression} is the variance of the log-normal random variable
 $\llikEmRdmDens[\Ndesign](\Par)$ conditioned on the ``pseudo-data''
 $\{\designBatchIn, \emMean[\Ndesign]{\llik}(\designBatchIn)\}$. The second term 
 $\varInflation(\Par; \designBatchIn)$, which we refer to as the \textit{variance inflation factor}, 
 accounts for the uncertainty due to $\designBatchLlik$. We note that 
$\varInflation(\Par; \designBatchIn) \geq 1$ and the variance inflation is larger when the 
input batch $\designBatchIn$ is more influential. If $\designBatchIn$ is a subset of the 
current design $\designIn[\Ndesign]$, then $\varInflation(\Par; \designBatchIn) = 1$ (no 
variance inflation is applied). 

With these preliminary results established, we define the $\intExpVarLabel$ criterion
\begin{align}
\labelAcq[\Ndesign]{\intExpVarLabel}(\designBatchIn) \Def
\int_{\parSpace} \E_{\designBatchLlik} \Var[\llikEmRdmDens[\Ndesign](\Par) | \llikEm[\Ndesign](\designBatchIn) = \designBatchLlik] \weightDens(\Par) d\Par, \label{def:int_exp_var_llik}
\end{align}
emphasizing the the expectation is still with respect to 
$\designBatchLlik \sim \Gaussian(\emMean[\Ndesign]{\llik}(\designBatchIn), \emKer[\Ndesign]{\llik}(\designBatchIn))$. 
Applying \Cref{lemma:evar} immediately gives the closed-form expression for the integrand in \ref{def:int_exp_var_llik}. 
This criterion is considered in \cite{VehtariParallelGP} in training a log-likelihood emulator to be used for 
approximate Bayesian computation.

\paragraph{Forward Model Emulation}
To define a similar criterion in the forward model emulation setting, we first recall that we are considering ``plug-in'' forward model emulators. 
Thus, the forward model emulator $\fwdEm$ is conditioned on a new batch $\{\designBatchIn, \designBatchFwd\}$ and then 
the updated emulator is plugged into the density in place of the true forward model. This implies that the analogous expression to 
the expected variance in \ref{def:int_exp_var_llik}, 
$\E_{\designBatchFwd} \Var[\fwdEmRdm[\Ndesign]{\postDens}(\Par) | \fwdEm[\Ndesign](\designBatchIn) = \designBatchFwd]$
does \textit{not} correctly convey how the emulator is updated in this setting. To correctly specify the expected variance in this 
context, we define $\fwdEmCond{\designBatchFwd} \Def \fwdEm | [\fwdEm(\designBatchIn) = \designBatchFwd]$
\footnote{Note that $\fwdEmCond{\designBatchFwd}$ is conditioned on $\{\designBatchIn, \designBatchFwd\}$, though the 
dependence on the inputs $\designBatchIn$ is suppressed in the notation.}, so that 
$\postDens(\Par; \fwdEmCond{\designBatchFwd})$ represents the unnormalized posterior density 
with the updated emulator plugged in. Following the convention from previous sections, we let 
$\emMean[\Ndesign]{\fwd}(\Par | \designBatchFwd)$ denote the expectation of 
$\fwdEm(\Par) | [\fwdEm(\designBatchIn) = \designBatchFwd]$.
We can now define the integrated variance criterion as 
\begin{align}
\labelAcq[\Ndesign]{\fwdintExpVarLabel}(\designBatchIn) \Def
\int_{\parSpace} \E_{\designBatchFwd} \Var[\postDens(\Par; \fwdEmCond{\designBatchFwd}) | \designBatchFwd]\weightDens(\Par) d\Par, \label{fwd-int-exp-var}
\end{align}
under the assumption $\designBatchFwd|\designBatchIn \sim \Gaussian(\emMean{\fwd}(\designBatchIn), \emKer{\fwd}(\designBatchIn))$.
To our knowledge, 
the paper \cite{SinsbeckNowak} is the first to consider a design criterion of the form \ref{fwd-int-exp-var}. 
While in general the tractability of this criterion depends on the specific likelihood, \cite{Surer2023sequential} 
demonstrates that the integrand admits a closed-form expression when the likelihood is Gaussian. 
While their result is tailored to the basis function approach for forward model emulation (see \ref{fwd_model_basis_func}), 
we state the result generically for multi-output forward models emulated by independent GPs, with the basis 
function setting following as a corollary [\todo: add the corollary in the appendix].  
\begin{prop} \label{prop:evar-fwd-emulation}
Consider 
$\fwdEm(\Par) \sim \GP(\emMean[\Ndesign]{\fwd}, \emKer[\Ndesign]{\fwd})$, interpreted as a set of $\dimObs$ independent 
GPs, one for each model output. Let $\{\designBatchIn, \designBatchFwd\}$ be a new batch of $\Nbatch$ 
design points, with
$\designBatchFwd | \designBatchIn \sim \Gaussian(\emMean[\Ndesign]{\fwd}(\designBatchIn), \emKer[\Ndesign]{\fwd}(\designBatchIn))$.
Then for $\Par \in \parSpace$, 
\begin{align}
\E_{\designBatchFwd} &\Var[\postDens(\Par; \fwdEmCond{\designBatchFwd}) | \designBatchFwd] \nonumber \\
&= \priorDens^2(\Par) \left[\frac{\Gaussian\left(\obs | \emMean{\fwd}(\Par), \frac{1}{2}\likPar + \emKer{\fwd}(\Par) \right)}{2^{\dimObs/2} \det(2\pi \likPar)^{1/2}} - 
\frac{\Gaussian\left(\obs | \emMean{\fwd}(\Par), \left[\frac{1}{2}\likPar + \emKer{\fwd}(\Par)\right] - \frac{1}{2}\emKer[\Naugment]{\fwd}(\Par) \right)}{2^{\dimObs/2} \det(2\pi [\likPar + \emKer[\Naugment]{\fwd}(\Par)])^{1/2}} \right] \label{fwd_evar1} \\
&= \priorDens^2(\Par) \left[\frac{\Gaussian\left(\obs | \emMean{\fwd}(\Par), \CovComb(\Par) - \frac{1}{2}\likPar \right)}{2^{\dimObs/2} \det(2\pi \likPar)^{1/2}} - 
\frac{\Gaussian\left(\obs | \emMean{\fwd}(\Par), \CovComb(\Par) - \frac{1}{2}\CovComb[\Naugment](\Par) \right)}{2^{\dimObs/2} \det(2\pi \CovComb[\Naugment](\Par))^{1/2}} \right], \label{fwd_evar2}
\end{align}
where $\CovComb(\Par) \Def \likPar + \emKer{\fwd}(\Par)$ and $\CovComb[\Naugment](\Par) \Def \likPar + \emKer[\Naugment]{\fwd}(\Par)$. 
\end{prop}

% Learning Likelihood Parameters
\section{Learning Likelihood Parameters} \label{section_lik_par}

% Numerical Experiments
\section{Numerical Experiments}

% Very Simple Ecosystem Model 
\subsection{Very Simple Ecosystem Model}
Our first set of experiments utilizes the \textit{Very Simple Ecosystem Model} (VSEM), a toy model capturing the basic 
structure of more complex land surface models, thus ideally suited for algorithm evaluation \citep{vsem}. The model 
describes the evolution of the state vector
\begin{align*}
\state(\Time) \Def [\stateV(\Time), \stateR(\Time), \stateS(\Time)]^\top \in \R^{3}, 
\end{align*}
with the state variables representing the quantity of carbon (\textrm{kg C/$m^2$}) in above-ground vegetation, below-ground 
vegetation (roots), and soil reservoirs, respectively. The VSEM model is given by the system of coupled 
ordinary differential equations
\begin{align}
\dstateV(\Time) &= \alphaV \NPP(\stateV(\Time), \forcing(\Time)) - \frac{\stateV(\Time)}{\tauV} \\
\dstateR(\Time) &= (1.0 - \alphaV) \NPP(\stateV(\Time), \forcing(\Time)) - \frac{\stateR(\Time)}{\tauR} \nonumber \\ 
\dstateS(\Time) &= \frac{\stateR(\Time)}{\tauR} + \frac{\stateV(\Time)}{\tauV} - \frac{\stateS(\Time)}{\tauS}, \nonumber
\end{align}
where the model forcing $\forcing(\Time)$ is provided by photosynthetically active radiation 
(PAR; \textrm{MJ/$m^2$/day}), and the dynamics rely on the following parameterized model of 
Net Primary Productivity (NPP; \textrm{kg C/$m^2$/day}),
\begin{align}
\NPP(\stateV, \forcing) &= (1 - \fracRespiration) \GPP(\stateV, \forcing) \\
\GPP(\stateV, \forcing) &= \forcing \cdot \LUE \cdot \left[1 - \exp\left\{-\KEXT \cdot \LAI(\stateV) \right\} \right] \nonumber \\
\LAI(\stateV) &= \LAR \cdot \stateV. \nonumber
\end{align} 
We note that the ODE is of the form \ref{ode_ivp}, with the caveat that the dynamics $\odeRHS$ additionally depend on a 
time-dependent forcing $\forcing(\Time)$. Potential calibration parameters in this model include 
$\{\alphaV, \tauV, \tauR, \tauS, \LUE, \fracRespiration, \LAR\}$, as well as the initial conditions for the three state variables. 
For all experiments, the model is discretized via a standard forward Euler scheme operating at a daily time step using the 
software provided in the R \verb+BayesianTools+ package \citep{vsem}. The forcing PAR data is simulated using the basic
radiation model provided by the \verb+VSEMcreatePAR+ function in this package. \\
\todo: list default and true parameter values. \\
\todo: Introduce the notation $\ParTrue$ for the true parameter value used to simulate the data. \\
\todo: Introduce model misspecification by letting the ground truth fixed parameter differ from their defaults, but use the 
defaults in the calibration. \\
\todo: define LAI.

\subsubsection{One-Dimensional Example} \label{vsem_1d}
Here we describe a Bayesian inverse problem setup using the VSEM model with one-dimensional input and output 
spaces ($\dimPar = \dimObs = 1$). While not representative of the typical difficulties encountered in higher dimensions, 
this basic example is intended to build intuition by allowing for easy visualization. We consider calibrating the light extinction
coefficient $\KEXT$ with the remaining parameters fixed at their default values. The model is simulated over a three year period so 
that the number of time steps is $\NTimeStep \Def 365 \cdot 3$. The observable is defined to be the corresponding three 
year average of $\LAI$. Using the notation from \cref{dynamical_models}, we define the likelihood by 
\begin{align*}
\obs &= (\obsOp \circ \solutionOp)(\Par) + \noise \\
\noise &\sim \Gaussian(0, 0.277^2),
\end{align*}
with
\begin{align*}
\obsOp(\indexTime[1]{\state}, \dots, \indexTime[\NTimeStep]{\state}) 
\Def \frac{1}{\NTimeStep} \sum_{\timeIndex=1}^{\NTimeStep}  \LAR \cdot \indexTime{\left(\stateV\right)}.
\end{align*}
The ground truth observable is given by $\fwd(\ParTrue) = (\obsOp \circ \solutionOp)(\ParTrue) \approx 5.55$ with 
the single observation $\obs \approx 5.17$ simulated by adding a draw from the Gaussian noise, as specified above. 
[\todo: add model discrepancy and estimate the noise variance]. The single input parameter is equipped with the 
prior 
\begin{align*}
\Par \sim \mathcal{U}(0.2, 1.0), 
\end{align*}
thus completing the specification of the Bayesian inverse problem. 


% Appendix 
\section{Appendix}

\subsection{Marginal Approximation with Gaussian Likelihood: Forward Model Emulation}
The closed-form computations related to the marginal approximation with a Gaussian 
likelihood follow from standard results regarding the convolution of Gaussian densities.  

\begin{prop} \label{Gaussian_convolution}
Let $\Gaussian(A \mu, \likPar)$ and $\Gaussian(m, C)$ be Gaussian distributions on $\R^{\dimObs}$ and $\R^{\dimPar}$, 
respectively, with $A \in \R^{\dimObs \times \dimPar}$ and $\likPar, C$ symmetric, positive definite matrices. Then 
\begin{align*}
\int_{\R^{\dimPar}} \Gaussian(\obs | A \mu, \likPar) \Gaussian(\mu | m, C) d\mu
&= \Gaussian(\obs | Am, \likPar + ACA^\top). 
\end{align*}
\end{prop}

\begin{proof} 
\todo
\end{proof}

We next prove a lemma that will be used in the proofs of \Cref{prop:Gaussian_marginal_moments} 
and \Cref{lemma:fwd-Gaussian-density-dist} below. 
\begin{lemma} \label{lemma:squared_Gaussian_density}
Let $\Gaussian(m, C)$ be a Gaussian distribution on $\R^{\dimObs}$ with $C$ a symmetric, positive-definite 
matrix. Then, for $\obs \in \R^{\dimObs}$, 
\begin{align*}
\Gaussian(\obs | m, C)^2 
&= 2^{-\dimObs/2} \det(2\pi C)^{-1/2} \Gaussian\left(\obs | m, \frac{1}{2}C\right). 
\end{align*}
\end{lemma}

\begin{proof}
\begin{align*}
\Gaussian(\obs | m, C)^2 
&= \det(2\pi C)^{-1} \Exp{-\frac{1}{2} (\obs - m)^\top \left[\frac{1}{2}C \right]^{-1}(\obs - m)} \\
&= \det(2\pi C)^{-1} \det(2\pi (1/2)C)^{1/2} \Gaussian\left(\obs | m, \frac{1}{2}C\right) \\
&= 2^{-\dimObs/2} \det(2\pi C)^{-1/2} \Gaussian\left(\obs | m, \frac{1}{2}C\right). 
\end{align*}
\end{proof}

\begin{prop} \label{prop:Gaussian_marginal_moments}
Assume $\obs | \mu \sim \Gaussian(A \mu, \likPar)$ and $\mu \sim \Gaussian(m, C)$, where $\mu \in \R^{\dimPar}$, 
$A \in \R^{\dimObs \times \dimPar}$, and $\likPar$, $C$ are both symmetric, positive definite. Then 
\begin{align}
\E\left[\Gaussian(\obs | A \mu, \likPar) \right] &= \Gaussian(\obs | Am, \likPar + ACA^\top) \\
\Var\left[\Gaussian(\obs | A \mu, \likPar) \right] 
&= \frac{\Gaussian\left(\obs | Am, \frac{1}{2} \likPar + ACA^\top \right)}{2^{\dimObs/2} \det(2\pi \likPar)^{1/2}} - 
\frac{\Gaussian\left(\obs | Am, \frac{1}{2}[\likPar + ACA^\top] \right)}{2^{\dimObs/2} \det(2\pi[\likPar + ACA^\top])^{1/2}}
\end{align}
\end{prop}

\begin{proof} 
The first result follows immediately from \Cref{Gaussian_convolution}. For the variance, we have 
\begin{align}
\Var\left[\Gaussian(\obs | A \mu, \likPar) \right] 
&= \E\left[\Gaussian(\obs | A \mu, \likPar)^2 \right] - \E\left[\Gaussian(\obs | A \mu, \likPar) \right]^2 \nonumber \\
&= \E\left[\Gaussian(\obs | A \mu, \likPar)^2 \right] - \Gaussian(\obs | Am, \likPar + ACA^\top)^2. \label{two_terms_variance}
\end{align}
Starting with the first term, we apply \Cref{lemma:squared_Gaussian_density} and 
\Cref{Gaussian_convolution}, respectively, to obtain 
\begin{align*}
\E\left[\Gaussian(\obs | A \mu, \likPar)^2 \right]
&= 2^{-\dimObs/2} \det(2\pi \likPar)^{-1/2} \E\left[\Gaussian\left(\obs | A\mu, \frac{1}{2}\likPar \right)\right] \\
&= 2^{-\dimObs/2} \det(2\pi \likPar)^{-1/2} \Gaussian\left(\obs | Am, \frac{1}{2}\likPar + ACA^\top \right).
\end{align*}
For the second term in \ref{two_terms_variance}, another application of \Cref{lemma:squared_Gaussian_density} gives
\begin{align*}
\Gaussian(\obs | Am, \likPar + ACA^\top)^2
&= 2^{-\dimObs/2} \det(2\pi[\likPar + ACA^\top])^{-1/2} \Gaussian\left(\obs | Am, \frac{1}{2}[\likPar + ACA^\top]\right).
\end{align*}
Plugging these expressions back into \ref{two_terms_variance} completes the proof. 
\end{proof}


\subsection{Marginal Acceptance Probability}
In this section we derive an expression for 
\begin{align}
\llikEmJointMarg[\Ndesign]{\accProbMH}(\Par, \propPar) 
&= \E_{\llikEm[\Ndesign]} \left[\min\left\{1, \Em[\Ndesign]{\accProbRatio}(\Par, \propPar)\right\} \right], 
\end{align}
as considered in \ref{acc_prob_joint_marg}. We start by noting 
\begin{align}
\Em[\Ndesign]{\accProbRatio}(\Par, \propPar)
&\sim \LN\left(\log C + \emMean[\Ndesign]{\llik}(\propPar) - \emMean[\Ndesign]{\llik}(\Par), \Var\left[\llikEm[\Ndesign](\propPar) - \llikEm[\Ndesign](\Par) \right] \right),
\label{acc_ratio_LN}
\end{align}
where 
\begin{align*}
C \Def \frac{\priorDens(\propPar) \propDens(\propPar, \Par)}{\priorDens(\Par) \propDens(\Par, \propPar)}.
\end{align*}
Thus, the required computation reduces to computing the expectation of $\min\left\{1, Y \right\}$, where $Y$ is a log-normally distributed 
random variable. 

\begin{lemma} \label{lemma:exp_max_one_LN}
Let $Y \sim \LN(m, s^2)$. Then, 
\begin{align}
\E\left[\min\left\{1, Y \right\} \right] &= \GaussianCDF\left(\frac{m}{s} \right) + \GaussianCDF\left(-\frac{m + s^2}{s} \right) e^{m + \frac{1}{2}s^2}
\end{align}
\end{lemma}

\begin{proof}
We have 
\begin{align}
\E\left[\min\left\{1, Y \right\} \right]
&= \int_{0}^{\infty} \min\{1, y\} \LN(y | m, s^2) dy \nonumber \\
&= \int_{-\infty}^{\infty} \min\{1, e^x\} \Gaussian(x | m, s^2) dx \nonumber \\
&= \int_{0}^{\infty} \Gaussian(x | m, s^2) dx +  \int_{-\infty}^{0} e^x \Gaussian(x | m, s^2) dx \nonumber \\
&= \GaussianCDF(m/s) + \int_{-\infty}^{0} e^x \Gaussian(x | m, s^2) dx. \nonumber \\
&= \GaussianCDF(m/s) + \int_{-\infty}^{0} \frac{1}{\sqrt{2\pi s^2}} e^x \Exp{-\frac{1}{2s^2} (x - m)^2} dx.  
\label{second_integral} 
\end{align}
For the integral in \ref{second_integral} we combine the exponential terms and complete the square. 
This yields 
\begin{align*}
&\int_{-\infty}^{0} \frac{1}{\sqrt{2\pi s^2}} e^x \Exp{-\frac{1}{2s^2} (x - m)^2} dx \\
&= \Exp{-\frac{m^2}{2s^2}} \int_{-\infty}^{0} \frac{1}{\sqrt{2\pi s^2}} \Exp{-\frac{1}{2} \left[\frac{x^2}{2} - 2\left(\frac{m}{s^2} + 1\right)x  \right]} dx \\
&= \Exp{-\frac{m^2}{2s^2} + \frac{[m + s^2]^2}{2s^2}} \int_{-\infty}^{0} \frac{1}{\sqrt{2\pi s^2}} \Exp{-\frac{1}{2s^2}(x - [m+s^2])^2} dx \\
&= \Exp{m + \frac{1}{2}s^2} \GaussianCDF\left(-\frac{m+s^2}{s}\right).
\end{align*}
Plugging this back into \ref{second_integral} completes the proof. 
\end{proof}

We now apply \Cref{lemma:exp_max_one_LN} to obtain the expression for the marginal acceptance probability. 
\begin{prop} \label{prop:joint-marg-accept-prob}
\begin{align*}
\llikEmJointMarg[\Ndesign]{\accProbMH}(\Par, \propPar) 
&= \E_{\llikEm[\Ndesign]} \left[\min\left\{1, \Em[\Ndesign]{\accProbRatio}(\Par, \propPar)\right\} \right] 
= w_1 + w_2 \llikEmJointMarg[\Ndesign]{\accProbRatio}(\Par, \propPar), 
\end{align*}
where 
\begin{align*}
w_1 &= \Prob(\Em[\Ndesign]{\accProbRatio}(\Par, \propPar) \geq 1) \\
w_2 &= \Prob\left(\Em[\Ndesign]{\accProbRatio}(\Par, \propPar) \leq \Exp{-\Var\left[\llikEm[\Ndesign](\propPar) - \llikEm[\Ndesign](\Par) \right]}\right).
\end{align*}
\end{prop}

\begin{proof}
We recall from \ref{acc_ratio_LN} that $\Em[\Ndesign]{\accProbRatio}(\Par, \propPar) \sim \LN(m, s^2)$, with 
\begin{align*}
m &= \log C + \emMean[\Ndesign]{\llik}(\propPar) - \emMean[\Ndesign]{\llik}(\Par) \\
s^2 &= \Var\left[\llikEm[\Ndesign](\propPar) - \llikEm[\Ndesign](\Par) \right]. 
\end{align*}
We thus have 
\begin{align*}
e^{m + \frac{1}{2}s^2} = \E_{\llikEm}\left[\Em[\Ndesign]{\accProbRatio}(\Par, \propPar) \right] 
=  \llikEmJointMarg[\Ndesign]{\accProbRatio}(\Par, \propPar). 
\end{align*}
It remains to verify the expressions for the weights $w_1$ and $w_2$. Letting $Z \sim \Gaussian(0,1)$, 
we apply \Cref{lemma:exp_max_one_LN} to obtain
\begin{align*}
w_1 &= \GaussianCDF(m/s) = \Prob(Z \leq m/s) = \Prob(\Exp{m+sZ} \geq 1) = \Prob(\Em[\Ndesign]{\accProbRatio}(\Par, \propPar) \geq 1) \\
w_2 &= \GaussianCDF(-(m+s^2)/s) = \Prob(\Exp{m + sZ} \leq e^{-s}) = \Prob(\Em[\Ndesign]{\accProbRatio}(\Par, \propPar) \leq e^{-s}).
\end{align*}
\end{proof}

\subsection{Transition Kernels of Approximate MCMC Algorithms} \label{transition_kernel_derivations}
We derive the expression for the \textit{MCWMH-joint} transition 
kernel $\llikEmJointMCWMH[\Ndesign]{\MarkovKernel}$ given in \ref{MCWMH-joint-kernel}. 
Let $\Par \in \parSpace$ and $A \subset \parSpace$ a Borel set. 
We recall that the kernel for a standard MH algorithm is given by 
\begin{align}
\MarkovKernel(\Par, A)
&= \int_{A} \propDens(\Par, \propPar) \accProbMH(\Par, \propPar) d\propPar
+ [1 - \avgAccProbMH(\Par)] \delta_{\Par}(A) \label{MH-kernel}, 
\end{align}
where $\avgAccProbMH(\Par)$ is the overall acceptance probability, averaged over all proposals,
\begin{align}
\avgAccProbMH(\Par)
&= \int_{\parSpace} \propDens(\Par, \propPar) \accProbMH(\Par, \propPar) d\propPar. 
\end{align}
Denote $\parMat \Def \{\Par, \propPar\}$. 
The \textit{MCWMH-joint} algorithm replaces the exact log-likelihood evaluations 
$\llik(\parMat) \Def [\llik(\Par), \llik(\propPar)]^\top$ used to define $\propDens(\Par, \propPar)$ by 
sampled approximate values 
\begin{align}
\llikSamp_{\parMat} \Def [\llikSamp, \llikSampProp]^\top \sim \Gaussian(\emMean[\Ndesign]{\llik}(\parMat), \emKer[\Ndesign]{\llik}(\parMat)). 
\end{align}
We let $\Em[\Ndesign]{\accProbMH}(\Par, \propPar | \llikSamp_{\parMat})$ denote the approximate acceptance probability defined using 
the sampled values $\llikSamp_{\parMat}$. The probability of accepting a state in the set $A$, conditional on the sample $\llikSamp_{\parMat}$, 
is thus 
\begin{align}
\Prob(\indexMCMC[\mcmcIndex+1]{\Par} \in A | \Par, \llikSamp_{\parMat})
&=  \int_{A} \propDens(\Par, \propPar) \Em[\Ndesign]{\accProbMH}(\Par, \propPar | \llikSamp_{\parMat}) d\propPar. 
\end{align}
The unconditional probability follows from the law of total probability:
\begin{align}
\Prob(\indexMCMC[\mcmcIndex+1]{\Par} \in A | \Par)
&= \int_{\R^2} \Prob(\indexMCMC[\mcmcIndex+1]{\Par} \in A | \Par, \llikSamp_{\parMat}) \Gaussian(\llikSamp_{\parMat} | \emMean[\Ndesign]{\llik}(\parMat), \emKer[\Ndesign]{\llik}(\parMat)) d\llikSamp_{\parMat} \\
&= \int_{\R^2} \int_{A} \propDens(\Par, \propPar) \Em[\Ndesign]{\accProbMH}(\Par, \propPar | \llikSamp_{\parMat})  
\Gaussian(\llikSamp_{\parMat} | \emMean[\Ndesign]{\llik}(\parMat), \emKer[\Ndesign]{\llik}(\parMat)) d\propPar \ d\llikSamp_{\parMat} \\
&= \int_{A} \propDens(\Par, \propPar) \int_{\R^2}  \Em[\Ndesign]{\accProbMH}(\Par, \propPar | \llikSamp_{\parMat})  
\Gaussian(\llikSamp_{\parMat} | \emMean[\Ndesign]{\llik}(\parMat), \emKer[\Ndesign]{\llik}(\parMat)) d\llikSamp_{\parMat} \ d\propPar \label{flip_integral_order} \\
&= \int_{A} \propDens(\Par, \propPar) \llikEmJointMarg[\Ndesign]{\accProbMH}(\Par, \propPar)  d\propPar
\end{align}
where \ref{flip_integral_order} follows from Tonelli's theorem, and using the definition of $\llikEmJointMarg[\Ndesign]{\accProbMH}(\Par, \propPar)$
in \ref{acc_prob_joint_marg}. Setting $A \Def \parSpace$ in the above integral yields the overall acceptance probability 
\begin{align}
\llikEmJointMarg[\Ndesign]{\avgAccProbMH}(\Par) \Def \int_{\parSpace} \propDens(\Par, \propPar) \llikEmJointMarg[\Ndesign]{\accProbMH}(\Par, \propPar)  d\propPar. 
\end{align}
These quantities thus give 
\begin{align}
\llikEmJointMCWMH[\Ndesign]{\MarkovKernel}(\Par, A)
&= \int_{A} \propDens(\Par, \propPar) \llikEmJointMarg[\Ndesign]{\accProbMH}(\Par, \propPar) d\propPar
+ [1 - \llikEmJointMarg[\Ndesign]{\avgAccProbMH}(\Par)] \delta_{\Par}(A) \label{MH-kernel}, 
\end{align}
which follows from same derivations for the standard MH kernel \ref{MH-kernel} with the marginal acceptance probabilities substituted for
the original ones. The transition kernel for the \textit{MCWMH-ind} algorithm follows immediately by marginalizing with respect 
to $\Gaussian(\emMean[\Ndesign]{\llik}\left(\parMat), \diag\left\{\emKer[\Ndesign]{\llik}(\Par), \emKer[\Ndesign]{\llik}(\propPar) \right\}\right)$
in place of the full joint distribution. 

\subsection{Sequential Design Calculations}
We start by stating an identity for the inversion of partitioned matrices, which is very useful in updating GPs by 
conditioning on new design points. The generic result can be found in the lecture notes \cite{MinkaMatrixLectures}, 
but we specialize the statement to the GP setting.  

\subsubsection{Useful Lemmas}

\begin{prop} \label{partitioned-matrix-inverse}
Let $\funcPrior \sim \GP(\gpMeanPrior, \gpKerPrior)$ be a GP prior, with 
$\funcEm[\Ndesign] \Def \funcPrior | [\funcPrior(\designIn) = \func(\designIn)] \sim \GP(\gpMean[\Ndesign], \gpKer[\Ndesign])$  
the GP predictive distribution after conditioning on the design $(\designIn, \func(\designIn))$. Let 
$\designBatchIn$ be a set of $\Nbatch$ new design points. Define 
$\kerMat[\Ndesign] \Def \gpKerPrior(\designIn)$, $\kerMat[\Nbatch] \Def \gpKerPrior(\designBatchIn)$,
and $\kerMat[\Ndesign,\Nbatch] \Def \gpKerPrior(\designIn[\Ndesign], \designBatchIn)$.  
Then, letting 
$\designIn[\Naugment] \Def \designIn \cup \designBatchIn$, the inverse of the kernel matrix 
evaluated on the augmented design satisfies 
\begin{align}
\gpKerPrior(\designIn[\Naugment])^{-1}
&= \begin{pmatrix} \kerMat[\Ndesign] & \kerMat[\Ndesign,\Nbatch] \\
\kerMat[\Ndesign,\Nbatch]^\top & \kerMat[\Nbatch] \end{pmatrix}^{-1}
=  \begin{pmatrix} \tilde{K} & -\kerMat[\Ndesign]^{-1} \kerMat[\Ndesign,\Nbatch] \gpKer[\Ndesign](\designBatchIn)^{-1} \\
-\gpKer[\Ndesign](\designBatchIn)^{-1} \kerMat[\Ndesign,\Nbatch]^\top  \kerMat[\Ndesign]^{-1} & \gpKer[\Ndesign](\designBatchIn)^{-1} \end{pmatrix},
\end{align}
where 
\begin{align}
\tilde{K} = \kerMat[\Ndesign]^{-1} + \kerMat[\Ndesign]^{-1} \kerMat[\Ndesign,\Nbatch] \gpKer[\Ndesign](\designBatchIn)^{-1} \kerMat[\Nbatch,\Ndesign] \kerMat[\Ndesign]^{-1}.
\end{align}
Thus, assuming $\kerMat[\Ndesign]^{-1}$ has already been computed, $\gpKerPrior(\designIn[\Naugment])^{-1}$ can be constructed in an additional 
$\BigO(\Nbatch^3 + \Nbatch \Ndesign^2 + \Nbatch^2 \Ndesign)$ operations. 
\end{prop}

We repeatedly use the fact that $\funcPrior|[\funcPrior(\designIn[\Naugment]) = \func(\designIn[\Naugment])]$ and 
$\funcEm[\Ndesign] | [\funcEm[\Ndesign](\designBatchIn) = \func(\designBatchIn)]$ are equal in distribution ; 
i.e., conditioning the GP prior on the entire design 
is equivalent to sequentially conditioning on subsets of the design. For the sake of completeness, we provide the rigorous justification for this below.

\begin{lemma} \label{lemma:gp-condition-order}
Let $\funcPrior \sim \GP(\gpMeanPrior, \gpKerPrior)$ be a GP prior. Consider a set of $\Naugment$ design points $\{\designIn[\Naugment], \funcVal[\Naugment]\}$
partitioned as $\designIn[\Naugment] = \designIn[\Ndesign] \cup \designBatchIn$ and $\funcVal[\Naugment] = \funcVal[\Ndesign] \cup \funcVal[\Nbatch]$.
Let $\funcEm[\Ndesign] \Def \func | [\func(\designIn[\Ndesign]) = \funcVal[\Ndesign]] \sim \GP(\gpMean[\Ndesign], \gpKer[\Ndesign])$. 
Then the random process $\funcPrior | [\funcPrior(\designIn[\Naugment]) = \funcVal[\Naugment]]$ is equal in distribution to the random process
$\funcEm[\Ndesign] | [\funcEm[\Ndesign](\designIn[\Nbatch]) = \funcVal[\Nbatch]]$. 
\end{lemma} 

\begin{proof} 
Since both processes in question are Gaussian it suffices to check that 
\begin{align*}
\E[\funcPrior(\parMat) | \funcPrior(\designIn[\Naugment]) = \funcVal[\Naugment]] 
&= \E[\funcEm[\Ndesign](\parMat) | \funcEm[\Ndesign](\designIn[\Nbatch]) = \funcVal[\Nbatch]] \\
\Cov[\funcPrior(\parMat) | \funcPrior(\designIn[\Naugment]) = \funcVal[\Naugment]] 
&= \Cov[\funcEm[\Ndesign](\parMat) | \funcEm[\Ndesign](\designIn[\Nbatch]) = \funcVal[\Nbatch]],
\end{align*}
for an arbitrary finite set of inputs $\parMat \subset \parSpace$. The quantities on the lefthand side are 
$\gpMean[\Naugment](\parMat)$ and $\gpKer[\Naugment](\parMat)$, by definition. We begin by expanding 
the expression $\gpKerPrior(\parMat, \designIn[\Naugment]) \kerMat[\Naugment]^{-1}$, noting that we 
are borrowing the notation from \Cref{partitioned-matrix-inverse}. Applying the partitioned matrix 
inversion identity from \Cref{partitioned-matrix-inverse} yields 
\begin{align*}
\gpKerPrior(\parMat, \designIn[\Naugment]) \kerMat[\Naugment]^{-1}
&= \begin{pmatrix} \gpKerPrior(\parMat, \designIn[\Ndesign]) &  \gpKerPrior(\parMat, \designIn[\Nbatch]) \end{pmatrix}
\begin{pmatrix} \tilde{K} & -\kerMat[\Ndesign]^{-1} \kerMat[\Ndesign,\Nbatch] \gpKer[\Ndesign](\designBatchIn)^{-1} \\
-\gpKer[\Ndesign](\designBatchIn)^{-1} \kerMat[\Ndesign,\Nbatch]^\top  \kerMat[\Ndesign]^{-1} & \gpKer[\Ndesign](\designBatchIn)^{-1} \end{pmatrix}.
\end{align*}
Denoting $\funcVal[\Ndesign]^\prime \Def \funcVal[\Ndesign] - \gpMeanPrior(\designIn[\Ndesign])$ and 
$\funcVal[\Nbatch]^\prime \Def \funcVal[\Nbatch] - \gpMeanPrior(\designIn[\Nbatch])$, the predictive mean $\gpMean[\Naugment](\parMat)$ is thus given by 
\begin{align*}
\gpMean[\Naugment](\parMat)
&= \gpMeanPrior(\parMat) + \gpKerPrior(\parMat, \designIn[\Naugment]) \kerMat[\Naugment]^{-1}[\funcVal[\Naugment] - \gpMeanPrior(\designIn[\Naugment])] \\
&= \gpMeanPrior(\parMat) + \begin{pmatrix} \gpKerPrior(\designIn[\Ndesign], \parMat) \\  \gpKerPrior(\designIn[\Nbatch], \parMat) \end{pmatrix}^\top
\begin{pmatrix} \tilde{K} & -\kerMat[\Ndesign]^{-1} \kerMat[\Ndesign,\Nbatch] \gpKer[\Ndesign](\designBatchIn)^{-1} \\
-\gpKer[\Ndesign](\designBatchIn)^{-1} \kerMat[\Ndesign,\Nbatch]^\top  \kerMat[\Ndesign]^{-1} & \gpKer[\Ndesign](\designBatchIn)^{-1} \end{pmatrix} 
\begin{pmatrix}  \funcVal[\Ndesign]^\prime \\  \funcVal[\Nbatch]^\prime \end{pmatrix} \\
&= \gpMeanPrior(\parMat) + \gpKerPrior(\parMat, \designIn[\Ndesign])\kerMat[\Ndesign]^{-1}\funcVal[\Ndesign]^\prime + 
\gpKer[\Ndesign](\parMat, \designIn[\Nbatch]) \gpKer[\Ndesign](\designIn[\Nbatch])^{-1} \funcVal[\Nbatch]^\prime - 
\gpKer[\Ndesign](\parMat, \designIn[\Nbatch]) \gpKer[\Ndesign](\designIn[\Nbatch])^{-1} \kerMat[\Ndesign,\Nbatch]^\top \kerMat[\Ndesign]^{-1} \funcVal[\Ndesign]^\prime \\
&= \gpMean[\Ndesign](\parMat) + \gpKer[\Ndesign](\parMat, \designIn[\Nbatch]) \gpKer[\Ndesign](\designIn[\Nbatch])^{-1}[\funcVal[\Nbatch]^\prime - \kerMat[\Ndesign,\Nbatch]^\top \kerMat[\Ndesign]^{-1} \funcVal[\Ndesign]^\prime] \\
&= \gpMean[\Ndesign](\parMat) + \gpKer[\Ndesign](\parMat, \designIn[\Nbatch]) \gpKer[\Ndesign](\designIn[\Nbatch])^{-1}[\funcVal[\Nbatch]^\prime - \gpMean[\Ndesign](\designIn[\Nbatch]) + \gpMeanPrior(\designIn[\Nbatch])] \\
&= \gpMean[\Ndesign](\parMat) + \gpKer[\Ndesign](\parMat, \designIn[\Nbatch]) \gpKer[\Ndesign](\designIn[\Nbatch])^{-1}[\funcVal[\Nbatch] - \gpMean[\Ndesign](\designIn[\Nbatch])] \\
&= \E[\funcEm[\Ndesign](\parMat) | \funcEm[\Ndesign](\designIn[\Nbatch]) = \funcVal[\Nbatch]] 
\end{align*}
where we have used the fact that the predictive covariance of the GP $\funcEm[\Ndesign]$ gives 
\[
\gpKer[\Ndesign](\parMat, \designIn[\Nbatch]) = \gpKerPrior(\parMat, \designIn[\Nbatch]) - 
\gpKerPrior(\parMat, \designIn[\Ndesign])\kerMat[\Ndesign]^{-1} \gpKerPrior(\designIn[\Ndesign], \parMat).
\]
The covariance calculation proceeds similarly by replacing $\funcVal[\Ndesign]^\prime$ and $\funcVal[\Nbatch]^\prime$ with 
$\gpKerPrior(\designIn[\Ndesign], \parMat)$ and $\gpKerPrior(\designIn[\Nbatch], \parMat)$, respectively. We obtain 
\begin{align*}
\gpKer[\Naugment](\parMat)
&= \gpKerPrior(\parMat) - \gpKerPrior(\parMat, \designIn[\Naugment]) \kerMat[\Naugment]^{-1}[\funcVal[\Naugment] - \gpMeanPrior(\designIn[\Naugment])] \\
&= \gpKerPrior(\parMat) - \begin{pmatrix} \gpKerPrior(\designIn[\Ndesign], \parMat) \\  \gpKerPrior(\designIn[\Nbatch], \parMat) \end{pmatrix}^\top
\begin{pmatrix} \tilde{K} & -\kerMat[\Ndesign]^{-1} \kerMat[\Ndesign,\Nbatch] \gpKer[\Ndesign](\designBatchIn)^{-1} \\
-\gpKer[\Ndesign](\designBatchIn)^{-1} \kerMat[\Ndesign,\Nbatch]^\top  \kerMat[\Ndesign]^{-1} & \gpKer[\Ndesign](\designBatchIn)^{-1} \end{pmatrix} 
\begin{pmatrix}  \gpKerPrior(\designIn[\Ndesign], \parMat) \\  \gpKerPrior(\designIn[\Nbatch], \parMat) \end{pmatrix} \\
&= \gpKer[\Ndesign](\parMat) - 
\gpKer[\Ndesign](\parMat, \designIn[\Nbatch]) \gpKer[\Ndesign](\designIn[\Nbatch])^{-1} [\gpKerPrior(\designIn[\Nbatch], \parMat) - \kerMat[\Ndesign,\Nbatch]^\top \kerMat[\Ndesign]^{-1}\gpKerPrior(\designIn[\Ndesign], \parMat)] \\
&= \gpKer[\Ndesign](\parMat) -  \gpKer[\Ndesign](\parMat, \designIn[\Nbatch]) \gpKer[\Ndesign](\designIn[\Nbatch])^{-1} \gpKer[\Ndesign](\designIn[\Nbatch], \parMat) \\
&= \Cov[\funcEm[\Ndesign](\parMat) | \funcEm[\Ndesign](\designIn[\Nbatch]) = \funcVal[\Nbatch]].
\end{align*}

\end{proof}

\subsubsection{Uncertainty in GP Predictive Mean Due to Unobserved Response}

\begin{proof} [Proof of \Cref{lemma:pred-mean-dist}]
Though the result is only required for a single input $\Par$, it is no more difficult to establish for a set of inputs $\parMat$. 
We recall that 
\begin{align*}
\gpMean[\Ndesign](\parMat | \designBatchFunc) 
&= \E[\funcEm[\Ndesign](\parMat) | \funcEm(\designBatchIn) = \designBatchFunc] \\
&= \gpMean[\Ndesign](\parMat) + \gpKer[\Ndesign](\parMat, \designBatchIn) \gpKer[\Ndesign](\designBatchIn)^{-1} [\designBatchFunc - \gpMean[\Ndesign](\designBatchIn)],
\end{align*}
following from the GP predictive equations \ref{kriging_eqns}. Since, $\designBatchFunc|\parMat \sim \Gaussian(\gpMean[\Ndesign](\designBatchIn), \gpKer[\Ndesign](\designBatchIn))$, 
we see that $\gpMean[\Ndesign](\parMat | \designBatchFunc)$ is a linear function of a Gaussian random variable. It is thus Gaussian distributed, with mean and covariance
\begin{align*}
\E_{\designBatchFunc}[\gpMean[\Ndesign](\parMat | \designBatchFunc)]
&= \gpMean[\Ndesign](\parMat) + \gpKer[\Ndesign](\parMat, \designBatchIn) \gpKer[\Ndesign](\designBatchIn)^{-1} [\gpMean[\Ndesign](\designBatchIn) - \gpMean[\Ndesign](\designBatchIn)] = \gpMean[\Ndesign](\parMat) \\
\Cov_{\designBatchFunc}[\gpMean[\Ndesign](\parMat | \designBatchFunc)]
&= \gpKer[\Ndesign](\parMat, \designBatchIn) \gpKer[\Ndesign](\designBatchIn)^{-1}  \gpKer[\Ndesign](\designBatchIn) \gpKer[\Ndesign](\designBatchIn)^{-1} \gpKer[\Ndesign](\designBatchIn, \parMat) \\
&=  \gpKer[\Ndesign](\parMat, \designBatchIn) \gpKer[\Ndesign](\designBatchIn)^{-1} \gpKer[\Ndesign](\designBatchIn, \parMat) \\
&= \gpKer[\Ndesign](\parMat) - \gpKer[\Naugment](\parMat).
\end{align*}
\end{proof}

\subsubsection{Integrated Conditional Variance Calculations: Log-Likelihood Emulation}

\begin{proof} [Proof of \Cref{lemma:evar}]
We start by noting that 
\begin{align*}
\Var[\llikEmRdmDens[\Ndesign](\Par) | \llikEm[\Ndesign](\designBatchIn) = \designBatchLlik]
&= \Var[\priorDens(\Par) \Exp{\llikEm(\Par)} | \llikEm[\Ndesign](\designBatchIn) = \designBatchLlik] \\
&= \priorDens(\Par)^2 \Var[\Exp{\llikEm(\Par)} | \llikEm[\Ndesign](\designBatchIn) = \designBatchLlik],
\end{align*}
so we will focus on the likelihood emulator, ignoring the prior for now. Since 
\begin{align*}
\Exp{\llikEm(\Par)} | [\llikEm[\Ndesign](\designBatchIn) = \designBatchLlik] \sim 
\LN(\emMean[\Ndesign]{\llik}(\Par| \designBatchLlik), \emKer[\Naugment]{\llik}(\Par)),
\end{align*}
we apply the formula for a log-normal variance to obtain 
\begin{align}
\Var[\Exp{\llikEm(\Par)} | \llikEm[\Ndesign](\designBatchIn) = \designBatchLlik]
&= \cst \Exp{2\emMean[\Ndesign]{\llik}(\Par | \designBatchLlik)}, \label{formula_plug_in}
\end{align}
where 
\begin{align*}
\cst \Def \left[\Exp{\emKer[\Naugment]{\llik}(\Par)} -1 \right] \Exp{\emKer[\Naugment]{\llik}(\Par)}
\end{align*}
is not a function of the random variable $\designBatchLlik$. \Cref{lemma:pred-mean-dist} gives 
\begin{align*}
\emMean[\Ndesign]{\llik}(\Par | \designBatchLlik)
&\sim \LN(\emMean[\Ndesign]{\llik}(\Par), \emKer[\Ndesign]{\llik}(\Par) - \emKer[\Naugment]{\llik}(\Par)), 
\end{align*}
which implies 
\begin{align*}
\Exp{2\emMean[\Ndesign]{\llik}(\Par | \designBatchLlik)}
&\sim \Gaussian(2\emMean[\Ndesign]{\llik}(\Par), 4[\emKer[\Ndesign]{\llik}(\Par) - \emKer[\Naugment]{\llik}(\Par)]).
\end{align*}
Applying the formula for a log-normal mean thus yields 
\begin{align*}
\E_{\designBatchLlik} \left[\Exp{2\emMean[\Ndesign]{\llik}(\Par | \designBatchLlik)} \right]
&= \Exp{2\emMean[\Ndesign]{\llik}(\Par)} \Exp{2[\emKer[\Ndesign]{\llik}(\Par) - \emKer[\Naugment]{\llik}(\Par)]} \\
&=  \Exp{2\emMean[\Ndesign]{\llik}(\Par)} \varInflation(\Par; \designBatchIn), 
\end{align*}
where $\varInflation(\Par; \designBatchIn)$ is defined in \ref{var_inflation_factor}. Plugging this expression back 
into \ref{formula_plug_in} gives 
\begin{align*}
\E_{\designBatchLlik} \Var[\Exp{\llikEm(\Par)} | \llikEm[\Ndesign](\designBatchIn) = \designBatchLlik]
&= \cst \Exp{2\emMean[\Ndesign]{\llik}(\Par)} \varInflation(\Par; \designBatchIn) \\
&= \Var[\Exp{\llikEm(\Par)} | \llikEm[\Ndesign](\designBatchIn) = \emMean[\Ndesign]{\llik}(\Par)] \varInflation(\Par; \designBatchIn).
\end{align*}
Multiplying both sides by $\priorDens(\Par)$ completes the proof, with the closed-form expression for the first term following 
immediately from the formula for a log-normal variance. 
\end{proof}

\subsubsection{Integrated Conditional Variance Calculations: Forward Model Emulation}

\begin{proof} [Proof of \Cref{prop:evar-fwd-emulation}]
We begin by noting that the squared prior density can be pulled out of the expectation as
\begin{align*}
\E_{\designBatchFwd} \Var[\postDens(\Par; \fwdEmCond{\designBatchFwd}) | \designBatchFwd]
&= \priorDens^2(\Par) \ \E_{\designBatchFwd}  \Var[\Gaussian\left(\obs |  \fwdEmCond{\designBatchFwd}(\Par), \likPar \right) | \designBatchFwd]. 
\end{align*}
The variance on the righthand side can be expanded as 
\begin{align}
\Var[\Gaussian\left(\obs |  \fwdEmCond{\designBatchFwd}(\Par), \likPar \right) | \designBatchFwd]
&= \frac{\Gaussian\left(\obs | \emMean[\Ndesign]{\fwd}(\Par|\designBatchFwd), \frac{1}{2}\likPar + \emKer[\Naugment]{\fwd}(\Par) \right)}{\det(2\pi \likPar)^{1/2}}
- \frac{\Gaussian\left(\obs | \emMean[\Ndesign]{\fwd}(\Par|\designBatchFwd), \frac{1}{2}\left[\likPar + \emKer[\Naugment]{\fwd}(\Par)\right]  \right)}{\det(2\pi [\likPar + \emKer[\Naugment]{\fwd}(\Par)])^{1/2}}
\end{align}
following from an application of \Cref{fwd_em_lik_emulator_moments_Gaussian}. The denominators in the above expression can be 
pulled out of the outer expectation and are seen to equal the denominators in the desired expression. We thus complete the proof by 
computing the expectation of the numerators with respect to 
$\designBatchFwd | \designBatchIn \sim \Gaussian(\emMean[\Ndesign]{\fwd}(\designBatchIn), \emKer[\Ndesign]{\fwd}(\designBatchIn))$. 
These expectations can be computed by noting \Cref{lemma:pred-mean-dist} and then applying \Cref{prop:Gaussian_marginal_moments}.
This yields,
\begin{align*}
\E_{\designBatchFwd}\left[\Gaussian\left(\obs \bigg| \emMean[\Ndesign]{\fwd}(\Par|\designBatchFwd), \frac{1}{2}\likPar + \emKer[\Naugment]{\fwd}(\Par) \right)\right]
&= \Gaussian\left(\obs \big| \emMean[\Ndesign]{\fwd}(\Par), 
\left[\frac{1}{2}\likPar + \emKer[\Naugment]{\fwd}(\Par)\right] + \left[\emKer[\Ndesign]{\fwd}(\Par) - \emKer[\Naugment]{\fwd}(\Par)\right] \right) \\
&= \Gaussian\left(\obs \big| \emMean[\Ndesign]{\fwd}(\Par), \frac{1}{2}\likPar + \emKer[\Ndesign]{\fwd}(\Par) \right)
\end{align*}
and
\begin{align*}
\E_{\designBatchFwd}\left[\Gaussian\left(\obs \big| \emMean[\Ndesign]{\fwd}(\Par|\designBatchFwd), \frac{1}{2}\left[\likPar + \emKer[\Naugment]{\fwd}(\Par)\right]\right)\right] 
&= \Gaussian\left(\obs \big| \emMean[\Ndesign]{\fwd}(\Par), \frac{1}{2} \left[\likPar + \emKer[\Naugment]{\fwd}(\Par)\right] + \left[\emKer[\Ndesign]{\fwd}(\Par) - \emKer[\Naugment]{\fwd}(\Par)\right] \right) \\
&= \Gaussian\left(\obs \big| \emMean[\Ndesign]{\fwd}(\Par), \left[\frac{1}{2}\likPar + \emKer[\Ndesign]{\fwd}(\Par)\right] - \frac{1}{2} \emKer[\Naugment]{\fwd}(\Par)\right),
\end{align*}
which completes the derivation of \Cref{fwd_evar1}. To obtain \Cref{fwd_evar2} we rearrange the covariances of the above expressions to obtain 
\begin{align*}
\frac{1}{2}\likPar + \emKer[\Ndesign]{\fwd}(\Par) &= 
\left[\likPar +  \emKer[\Ndesign]{\fwd}(\Par) \right] - \frac{1}{2}\likPar 
= \CovComb(\Par) - \frac{1}{2}\likPar \\
\left[\frac{1}{2}\likPar + \emKer[\Ndesign]{\fwd}(\Par)\right] - \frac{1}{2} \emKer[\Naugment]{\fwd}(\Par)
&= \left[\likPar + \emKer[\Ndesign]{\fwd}(\Par) \right] - \frac{1}{2}\left[\likPar + \emKer[\Naugment]{\fwd}(\Par) \right] 
= \CovComb(\Par) - \frac{1}{2} \CovComb[\Naugment](\Par). 
\end{align*}

\end{proof}


% Questions and TODOs
\section{Questions and TODOs}
\subsection{Questions}
\begin{enumerate}
\item How to reliably use a log-normal emulator for Bayesian inference? 
	\begin{enumerate}
	\item Improve GP calibration (e.g., quadratic mean) 
	\item Use more robust statistics (e.g., interquartile range)
	\item Truncate proposal or prior. 
	\end{enumerate}
\item How to deal with highly concentrated/correlated posteriors? 
	\begin{enumerate}
	\item MALA or other samplers. 
	\end{enumerate}
\end{enumerate}

\subsection{Need to add}
\begin{enumerate}
\item Summary of results from noisy MCMC literature.
\item Try working out results that compare the ratio of the approx density at two points across different approximations; 
alternatively could consider deriving these results for the normalized densities. 
\item Include result that sample and marginal approx agree at the design points.
\item Include existence results (check existence result from that new paper)
\item Viewing noisy MCMC approaches as approximations to the sample-based posterior.
\item Different views of noisy MCMC approaches, including extending the state space. How does this alg compare to the 
marginal and mcwmh-ind algs?
\item Add some sort of theoretical result that demonstrates that the marginal approximation is extremely sensitive to the GP 
variance. Based on numerical experiments, seems like this result should be given with respect to the dynamic range of the 
log-likelihood. Also of course depends on how fast the GP variance grows away from the design points, so perhaps should 
consider fill distance or something like this as well.
\item Numerical experiment that considers the different ways to weight the integrated uncertainty criteria (i.e., targeting 
the unnormalized posterior density vs. using the approx posterior samples as weights).
\item Posterior consistency results for the noisy MCMC emulators; combine the noisy MCMC results with GP approximation 
results.
\item Evaluating calibration of GP-approximated posteriors relative to calibration of the underlying GP emulator.
\item Constrained GPs
\item Pathwise sampling approach to approximate the sample-based approximation.
\item Compare noisy MCMC vs. deterministic version that considers integrating over the acceptance prob. 
\end{enumerate}

\subsection{Emulator Ideas}
\begin{enumerate}
\item Sum of quadratic kernel, Gaussian kernel, and some sort of flat/linear kernel (i.e. something with very long lengthscales) to capture the 
part of the response surface that "flattens" out. 
\end{enumerate}

\subsection{Notation}
\begin{enumerate}
\item How to clean up notation for all of the different approximations being considered here? 
\end{enumerate}

\subsection{Numerical experiments:}
My plan is to have emulation in dynamical settings (ODEs) as the unifying theme here. VSEM can provide the core example but could also consider 
others, such as Lorenz-63 (see Hwanwoo Kim, Daniel Sanz-Alonso paper for the ODEs they consider). When introducing the dynamical setting,
cite Stuart/Schneider Earth System modeling 2.0 paper. Provide various examples of observation operators: time-averages (moments) of state 
variables, identity operator, many shorter time-averages (e.g., weekly/monthly averages), multi-objective settings of calibrating to multiple state 
variables. I should probably include Lorenz-63 to have a more familiar example to many audiences. 

\begin{enumerate}
\item 1D example with 1D output for basic illustration. 
	\begin{enumerate}
	\item VSEM with single varied parameter. 
	\item For 1D output, consider long time average of of a single state variable. This will allow us to compare forward model and log-likelihood emulation directly.
	\item Gaussian likelihood. 
	\item Compare emulator distributions, log-likelihood emulator distributions, and likelihood emulator distributions, and various posterior approximations. 
	\item Validation metrics: RMSE, MAE, CRPS, Log-Score, Coverage. 
	\end{enumerate}
\item 2D examples for basic illustration and consideration of different posterior characteristics: Gaussian, banana, unidentifiable, concentration level, bimodal. 
\item Extension of 2D densities to 6-10 dims (see Vehtari paper numerical experiments) 
\item Sequential design performance at different levels of posterior concentration. 
\item Integrated uncertainty criterion: incorporating the current posterior via the integrand or the measure?
\item Large batch, few iteration sequential design experiment. 
(4 different combinations we could consider) 
\end{enumerate}

\begin{enumerate}
\item 1D example for basic illustration. 
\item 2D examples for basic illustration and consideration of different posterior characteristics: Gaussian, banana, unidentifiable, concentration level, bimodal. 
\item Extension of 2D densities to 6-10 dims (see Vehtari paper numerical experiments) 
\item Sequential design performance at different levels of posterior concentration. 
\item Integrated uncertainty criterion: incorporating the current posterior via the integrand or the measure?
\item Large batch, few iteration sequential design experiment. 
(4 different combinations we could consider) 
\end{enumerate}

\subsection{Potential examples:}
\begin{enumerate}
\item Banana, unimodal, bimodal, unidentifiable
\item Heat equation (see Sinsbeck and Nowak) 
\item VSEM
\end{enumerate}

\subsection{Things to consider trying}
\begin{enumerate}
\item Perhaps give some context by discussing connections to Bayesian optimization and to log-likelihood approximation used in 
simulation-based inference. 
\item Hyperparameter marginalization: need to look into opportunities for closed-form hyperparameter marginalization during 
sequential design phase. 
\item Developing a design criterion that better aligns with the MHWMC procedure; e.g., something that targets the likelihood ratio. 
\item Implementing Higdon basis function approach for comparison. 
\item Nonnegative constrained GPs (may be able to do this by modifying the kergp optimization code and using nloptr's option to add constraints) 
\item GP-accelerated MALA 
\end{enumerate}

\subsection{Limitations of existing literature}
\begin{enumerate}
\item Very little discussion of case where likelihood parameters are unknown. 
\item Lack of emphasis on batch design (with some exceptions).
\item Little guidance on which approximation/design criterion to choose.
\item Vehtari fixes the marginal approx, and focuses instead on varying the design criterion, but notes that sampling from the marginal approx is problematic. 
\end{enumerate}

\subsection{Conjectures}
\begin{enumerate}
\item The MCWMH algorithms will perform better than sampling from the marginal approx in the log-likelihood emulation setting, especially when the GP is very uncertain. 
\end{enumerate}

\subsection{Consideration}
\begin{enumerate}
\item Literature typically focuses on convergence of the approx posterior. But in cases with very expensive computer models, one might have to stop pre-convergence. 
In these settings the comparison between the approximate posteriors becomes even more important. 
\end{enumerate}



\bibliography{post_approx_with_GP_emulators} 
% \bibliographystyle{ieeetr}

\end{document}







