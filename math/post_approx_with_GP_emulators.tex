\documentclass[12pt]{article}
\RequirePackage[l2tabu, orthodox]{nag}
\usepackage[main=english]{babel}
\usepackage[rm={lining,tabular},sf={lining,tabular},tt={lining,tabular,monowidth}]{cfr-lm}
\usepackage{amsthm,amssymb,latexsym,gensymb,mathtools,mathrsfs}
\usepackage[T1]{fontenc}
\usepackage[utf8]{inputenc}
\usepackage[pdftex]{graphicx}
\usepackage{epstopdf,enumitem,microtype,dcolumn,booktabs,hyperref,url,fancyhdr}
\usepackage{algorithmic}
\usepackage[ruled,vlined,commentsnumbered,titlenotnumbered]{algorithm2e}
\usepackage{bbm}

% Plotting
\usepackage{pgfplots}
\usepackage{xinttools} % for the \xintFor***
\usepgfplotslibrary{fillbetween}
\pgfplotsset{compat=1.8}
\usepackage{tikz}

% Local custom commands. 
%!TEX root = SMA-book.tex

% Useful info on newcomand: https://tex.stackexchange.com/questions/117358/newcommand-argument-confusion

% --------------------------------------------------------------------------------------------------
% General commands.  
% --------------------------------------------------------------------------------------------------

\newcommand{\todo}{\textbf{TODO}}


% --------------------------------------------------------------------------------------------------
% General Math.  
% --------------------------------------------------------------------------------------------------

% Common math commands. 
\newcommand*{\abs}[1]{\left\lvert#1\right\rvert}
\newcommand{\R}{\mathbb{R}}
\newcommand*{\suchthat}{\,\mathrel{\big|}\,}
\newcommand{\Exp}[1]{\exp\mathopen{}\left\{#1\right\}\mathclose{}}
\newcommand{\Log}[1]{\log\mathopen{}\left\{#1\right\}\mathclose{}}
\newcommand{\BigO}{\mathcal{O}}

\DeclareMathOperator*{\argmax}{argmax}
\DeclareMathOperator*{\argmin}{argmin}
\DeclarePairedDelimiterX\innerp[2]{(}{)}{#1\delimsize\vert\mathopen{}#2}
\DeclarePairedDelimiter{\ceil}{\lceil}{\rceil}

% Linear algebra.
\newcommand*{\norm}[1]{\left\lVert#1\right\rVert}
\newcommand{\bx}{\mathbf{x}}
\newcommand{\bw}{\mathbf{w}}
\newcommand{\oneVec}[1][]{\boldsymbol{1}_{#1}}
\newcommand{\idMat}{\mathbf{I}}
\newcommand{\bS}{\mathbf{S}}
\newcommand{\bK}{\mathbf{K}}
\newcommand{\ba}{\mathbf{a}}
\newcommand{\bb}{\mathbf{b}}

% Probability. 
\newcommand{\E}{\mathbb{E}}
\newcommand{\Var}{\mathrm{Var}}
\newcommand{\Cov}{\mathrm{Cov}}
\newcommand{\Prob}{\mathbb{P}}
\newcommand{\Gaussian}{\mathcal{N}}
\newcommand{\LN}{\text{LN}} % Log-normal distribution 


%!TEX root = SMA-book.tex

% Useful info on newcomand: https://tex.stackexchange.com/questions/117358/newcommand-argument-confusion


% --------------------------------------------------------------------------------------------------
% Bayesian inverse problem: general setup 
% --------------------------------------------------------------------------------------------------

% General inverse problems.
\newcommand{\Par}{u}
\newcommand{\parSpace}{\mathcal{U}} % Parameter space
\newcommand{\dimPar}{D} % Parameter dimension 
\newcommand{\idxParDim}{d} % Primary symbol used to index each parameter dimension. 
\newcommand{\obs}{y} % Data observation (response) vector
\newcommand{\noise}{\epsilon} % Random variable representing noise, typically in additive noise model. 
\newcommand{\obsSpace}{\mathcal{Y}} % Output space
\newcommand{\dimObs}{P} % Dimension of data observation vector
\newcommand{\idxObsDim}{p} % Primary symbol used to index each data dimension. 
\newcommand{\fwd}{G} % Forward model
\newcommand{\llik}{\mathcal{L}} % Log-likelihood 
\newcommand{\priorDens}{\pi_0} % Prior density 
\newcommand{\postDens}{\pi} % Unnormalized posterior density.
\newcommand{\postDensNorm}{\pi_{\text{norm}}} % Normalized posterior density.
\newcommand{\normCst}{Z} % Normalizing constant for posterior density. 
\newcommand{\likPar}{\Sigma} % Likelihood parameter


% --------------------------------------------------------------------------------------------------
% Gaussian Process Emulators
% --------------------------------------------------------------------------------------------------

% Generic GP commands. 
\newcommand{\GP}{\mathcal{GP}} % GP distribution.
\newcommand{\LNP}{\mathcal{LNP}} % Log-normal process distribution. 
\newcommand{\func}{f} % Generic function used for results that hold for any generic GP. 
\newcommand{\gpMean}[1][]{\mu_{#1}}
\newcommand{\gpKer}[1][]{k_{#1}}
\newcommand{\nuggetSD}{\eta}

% Exponentiated quadratic kernel.
\newcommand{\lengthscale}{\ell}
\newcommand{\margSD}{\sigma_{\gpKer}}

% Log-likelihood and forward model emulators. 
\newcommand{\funcEm}[1][]{\func_{#1}}
\newcommand{\llikEm}[2][]{#1_{#2}}
\newcommand{\emMean}[2][]{\mu^{#1}_{#2}}
\newcommand{\emKer}[2][]{k^{#1}_{#2}}

% Design. 
\newcommand{\Ndesign}{N}
\newcommand{\NBatch}{B}
\newcommand{\idxDesign}{n} % Primary symbol used to index each design point. 
\newcommand{\parMat}{U} % Generic matrix of input parameter points. 
\newcommand{\designIn}[1][\Ndesign]{U_{#1}} % Design inputs. 
\newcommand{\designOutLlik}[1][\Ndesign]{\ell_{#1}}
\newcommand{\designOutFwd}[1][\Ndesign]{g_{#1}}
\newcommand{\designBatchIn}{U_{\NBatch}}
\newcommand{\designBatchLlik}{\ell_{\NBatch}}
\newcommand{\designBatchFwd}{g_{\NBatch}}

% --------------------------------------------------------------------------------------------------
% Markov Chain Monte Carlo. 
% --------------------------------------------------------------------------------------------------

% Generic MCMC commands. 
\newcommand{\propDens}{q} % Proposal density
\newcommand{\CovProp}{\mathbf{C}} % Proposal covariance matrix
\newcommand{\NMCMC}{T_{\text{MCMC}}} % Proposal covariance matrix
\newcommand{\accProbMH}{\alpha} % Metropolis-Hastings acceptance probability


\newcommand{\bphi}{\boldsymbol{\phi}}

\setlist{topsep=1ex,parsep=1ex,itemsep=0ex}
\setlist[1]{leftmargin=\parindent}
\setlist[enumerate,1]{label=\arabic*.,ref=\arabic*}
\setlist[enumerate,2]{label=(\alph*),ref=(\alph*)}

% For embedding images
\graphicspath{ {./images/} }

% Specifically for paper formatting 
\renewcommand{\baselinestretch}{1.2} % Spaces manuscript for easy reading

% Formatting definitions, propositions, etc. 
\newtheorem{definition}{Definition}
\newtheorem{prop}{Proposition}
\newtheorem{lemma}{Lemma}
\newtheorem{thm}{Theorem}
\newtheorem{corollary}{Corollary}

% Title and author
\title{Gaussian Process Surrogates for Bayesian Inverse Problems: Posterior Approximation and Sequential Design}
\author{Andrew Roberts}

\begin{document}

\maketitle


% Introduction 
\section{Introduction}

% Setting and Background 
\section{Setting and Background}

\subsection{Bayesian Inverse Problems}
We consider a \textit{forward model} $\fwd: \R^{\dimPar} \to \R^{\dimObs}$ describing some system of interest, parameterized by 
input parameters $\Par \in \parSpace \subset \R^{\dimPar}$. In addition, we suppose that we have noisy observations 
$\obs \in \obsSpace \subset \R^{\dimObs}$ of the output signal which $\fwd(\Par)$ seeks to approximate. The 
\textit{inverse problem} concerns learning the parameter values $\Par$ such that $\fwd(\Par) \approx \obs$; i.e., \textit{calibrating}
the model so that it agrees with reality. The statistical approach to this problem assumes that the link between model outputs and 
observations is governed by a probability distribution on  $\obs | \Par$. We assume that this distribution admits a density 
with corresponding log-likelihood 
\begin{align}
\llik: \parSpace \to \R, \label{log_likelihood}
\end{align}
such that $p(\obs | \Par) = \Exp{\llik(\Par)}$ 
\footnote{For succinctness, the notation $\llik(\Par)$ suppresses dependence on the data $\obs$, 
as we assume the data is fixed throughout the analysis.}
\footnote{We are currently focusing on the forward model parameter $\Par$, treating other likelihood 
parameters (e.g., observation covariance) as fixed. We will relax this assumption 
in section \ref{section_lik_par}.}.

The Bayesian approach completes this specification with a prior distribution 
on $\Par$. Letting $\priorDens(\Par)$ denote the density of this distribution, the Bayesian solution of the inverse problem is given 
by the posterior distribution $\postDensNorm(\Par) := p(\Par | \obs)$. The posterior density is computed by Bayes' rule 
\begin{align}
\postDensNorm(\Par) = \frac{1}{\normCst}\postDens(\Par) := \frac{1}{\normCst} \priorDens(\Par) \Exp{\llik(\Par)}, \label{post_dens}
\end{align}
with normalizing constant given by 
\begin{align}
\normCst &= \int \priorDens(\Par) \Exp{\llik(\Par)} d\Par. \label{norm_cst}
\end{align}
We emphasize that throughout this paper $\postDens(\Par)$ denotes the \textit{unnormalized} 
posterior density. In addition to considering 
generic likelihoods, we will give special attention to the additive Gaussian noise model
\begin{align}
\obs &= \fwd(\Par) + \noise \label{inv_prob_Gaussian} \\
\noise &\sim \Gaussian(0, \likPar) \nonumber \\
\Par &\sim \priorDens \nonumber, 
\end{align}
with corresponding log-likelihood 
\begin{align}
\llik(\Par) &= \log\Gaussian(\obs| \fwd(\Par), \likPar) 
= -\frac{1}{2} \log\det (2\pi \likPar) - \frac{1}{2} (\obs - \fwd(\Par))^\top \likPar^{-1} (y - \fwd(\Par)). \label{llik_Gaussian}
\end{align}

In general, the nonlinearity of $\fwd$ precludes a closed-form characterization of the posterior. In this case, the 
standard approach is to instead draw samples from the distribution using a Markov chain Monte Carlo (MCMC) 
algorithm. Such algorithms are iterative in nature, often requiring $\BigO(10^6)$ iterations, with each 
iteration involving an evaluation of the unnormalized posterior density $\postDens$. In the inverse problem 
context, this computational requirement is often prohibitive when the forward model evaluations $\fwd(\Par)$
incur significant computational cost. This becomes clear upon observing that each log-likelihood evaluation 
$\llik(\Par)$ (e.g., \ref{llik_Gaussian} in the additive Gaussian case) requires the corresponding forward 
model evaluation $\fwd(\Par)$. Motivated by this problem, a large body of work has focused on deriving 
cheap approximations to either $\fwd(\Par)$ or $\llik(\Par)$ (note that approximating the former induces 
an approximation of the latter). We refer to such approximations interchangeably 
as \textit{surrogates} or \textit{emulators}. The focus of the present work is on utilizing \textit{Gaussian processes} 
to derive emulators for the forward model or log-likelihood. 

\subsection{Gaussian Processes}
In this section, we provide a brief review of Gaussian processes (GPs); for in depth treatments, we refer readers to 
(TODO: cite Gramacy, Stuart, Rasmussen and Williams). 

% Posterior Approximation
\section{Posterior Approximation}
\subsection{Deterministic Approximations}
\subsection{Noisy MCMC}

% Sequential Design
\section{Sequential Design}
\subsection{Background}
\subsection{Integrated Expected Variance}

% Learning Likelihood Parameters
\section{Learning Likelihood Parameters} \label{section_lik_par}

% Appendix 
\section{Appendix}

% Questions and TODOs
\section{Questions and TODOs}
\subsection{Notation}
\begin{enumerate}
\item Cleanest way to handle notation for llik emulation vs. forward model emulation? 
\item Should I define $\postDens$ to be unnormalized, or introduce separate notation for the unnormalized density. 
\item Should I define everything in terms of the potential (i.e. negative llik) instead of the llik? 
\item Think about how to define the domain/co-domain of the forward model; should this differ from the par/observation space? 
\end{enumerate}



\bibliography{framework_calibrating_ecosystem_models} 
\bibliographystyle{ieeetr}

\end{document}







