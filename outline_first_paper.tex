\documentclass[12pt]{article}
\RequirePackage[l2tabu, orthodox]{nag}
\usepackage[main=english]{babel}
\usepackage[rm={lining,tabular},sf={lining,tabular},tt={lining,tabular,monowidth}]{cfr-lm}
\usepackage{amsthm,amssymb,latexsym,gensymb,mathtools,mathrsfs}
\usepackage[T1]{fontenc}
\usepackage[utf8]{inputenc}
\usepackage[pdftex]{graphicx}
\usepackage{epstopdf,enumitem,microtype,dcolumn,booktabs,hyperref,url,fancyhdr}
\usepackage{algorithmic}
\usepackage[ruled,vlined,commentsnumbered,titlenotnumbered]{algorithm2e}
\usepackage{bbm}

% Plotting
\usepackage{pgfplots}
\usepackage{xinttools} % for the \xintFor***
\usepgfplotslibrary{fillbetween}
\pgfplotsset{compat=1.8}
\usepackage{tikz}

% Custom Commands
\newcommand*{\norm}[1]{\left\lVert#1\right\rVert}
\newcommand*{\abs}[1]{\left\lvert#1\right\rvert}
\newcommand*{\suchthat}{\,\mathrel{\big|}\,}
\newcommand{\E}{\mathbb{E}}
\newcommand{\Var}{\mathrm{Var}}
\newcommand{\R}{\mathbb{R}}
\newcommand{\N}{\mathcal{N}}
\newcommand{\Ker}{\mathrm{Ker}}
\newcommand{\Cov}{\mathrm{Cov}}
\newcommand{\Prob}{\mathbb{P}}
\DeclarePairedDelimiterX\innerp[2]{(}{)}{#1\delimsize\vert\mathopen{}#2}
\DeclareMathOperator*{\argmax}{argmax}
\DeclareMathOperator*{\argmin}{argmin}
\DeclarePairedDelimiter{\ceil}{\lceil}{\rceil}

\setlist{topsep=1ex,parsep=1ex,itemsep=0ex}
\setlist[1]{leftmargin=\parindent}
\setlist[enumerate,1]{label=\arabic*.,ref=\arabic*}
\setlist[enumerate,2]{label=(\alph*),ref=(\alph*)}

% For embedding images
\graphicspath{ {./images/} }

% Specifically for paper formatting 
\renewcommand{\baselinestretch}{1.2} % Spaces manuscript for easy reading

% Formatting definitions, propositions, etc. 
\newtheorem{definition}{Definition}
\newtheorem{prop}{Proposition}
\newtheorem{lemma}{Lemma}
\newtheorem{thm}{Theorem}
\newtheorem{corollary}{Corollary}

% Title and author
\title{Loss Emulation for Scalable Ecosystem Model Calibration}
\author{Andrew Roberts}

\begin{document}

\maketitle
\tableofcontents
\newpage

\section{Abstract}
\begin{enumerate}
\item \textbf{Problem}: Improve prediction and understanding of ecosystem dynamics. \\
	\textbf{Solution}: Model-data fusion; combining strengths of mechanistic models and data-driven approaches. 
\item \textbf{Problem}: Effective model-data fusion requires careful and robust UQ. In particular, uncertainty must be acknowledged in empirically-determined estimates of process model parameters. 
\end{enumerate}


\section{Introduction}


\section{Methods}

\subsection{Statistical Model}
\subsubsection{Likelihood: Multi-objective calibration}
\subsubsection{Priors}
\subsubsection{Exact MCMC-based inference}

\subsection{Loss Emulation}
\subsubsection{Emulator Details}
\subsubsection{Emulator-based MCMC inference}

\subsection{Sequential Design}
\begin{enumerate}
\item \textbf{Problem}: While space-filling designs (e.g. via LHS or maximin) are common, they can be very inefficient in the Bayesian inverse problem setting. If the posterior distribution is concentrated in a small 
subset of the parameter space, then space-filling designs will yield many design points in regions that are not of interest, while simultaneously under-sampling the region of significance.  \\
	\textbf{Solution}: Sequential design approach that takes into account knowledge of the posterior as it proceeds. 
\end{enumerate}


\end{document}




