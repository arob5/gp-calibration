\documentclass[12pt]{article}
\RequirePackage[l2tabu, orthodox]{nag}
\usepackage[main=english]{babel}
\usepackage[rm={lining,tabular},sf={lining,tabular},tt={lining,tabular,monowidth}]{cfr-lm}
\usepackage{amsthm,amssymb,latexsym,gensymb,mathtools,mathrsfs}
\usepackage[T1]{fontenc}
\usepackage[utf8]{inputenc}
\usepackage[pdftex]{graphicx}
\usepackage{epstopdf,enumitem,microtype,dcolumn,booktabs,hyperref,url,fancyhdr}
\usepackage{algorithmic}
\usepackage[ruled,vlined,commentsnumbered,titlenotnumbered]{algorithm2e}

% Plotting
\usepackage{pgfplots}
\usepackage{xinttools} % for the \xintFor***
\usepgfplotslibrary{fillbetween}
\pgfplotsset{compat=1.8}
\usepackage{tikz}

% Custom Commands
\newcommand*{\norm}[1]{\left\lVert#1\right\rVert}
\newcommand*{\abs}[1]{\left\lvert#1\right\rvert}
\newcommand*{\suchthat}{\,\mathrel{\big|}\,}
\newcommand{\E}{\mathbb{E}}
\newcommand{\Var}{\mathrm{Var}}
\newcommand{\R}{\mathbb{R}}
\newcommand{\N}{\mathcal{N}}
\newcommand{\Ker}{\mathrm{Ker}}
\newcommand{\Cov}{\mathrm{Cov}}
\newcommand{\Prob}{\mathbb{P}}
\DeclarePairedDelimiterX\innerp[2]{(}{)}{#1\delimsize\vert\mathopen{}#2}
\DeclareMathOperator*{\argmax}{argmax}
\DeclareMathOperator*{\argmin}{argmin}
\DeclarePairedDelimiter{\ceil}{\lceil}{\rceil}

\setlist{topsep=1ex,parsep=1ex,itemsep=0ex}
\setlist[1]{leftmargin=\parindent}
\setlist[enumerate,1]{label=\arabic*.,ref=\arabic*}
\setlist[enumerate,2]{label=(\alph*),ref=(\alph*)}

% For embedding images
\graphicspath{ {./images/} }

% Specifically for paper formatting 
\renewcommand{\baselinestretch}{1.2} % Spaces manuscript for easy reading

% Formatting definitions, propositions, etc. 
\newtheorem{definition}{Definition}
\newtheorem{prop}{Proposition}
\newtheorem{lemma}{Lemma}
\newtheorem{thm}{Theorem}
\newtheorem{corollary}{Corollary}

% Title and author
\title{Research Notes: Ecological Modeling}
\author{Andrew Roberts}

\begin{document}

\maketitle
\tableofcontents
\newpage

\section{11/11/2022}

\subsection{Summary of Current Model}
\begin{itemize}
\item Notation: Calibration inputs $\theta$, computer model $f$, field data $y$.
\item Computer model: 
	\begin{itemize}
	\item Inputs: $\theta \in \R^{d_\theta}$, $n$ time steps defining the time range. 
	\item Outputs: $p$ time series, each of length $n$. Each time series corresponds to a constraint/output variable (e.g. NEE, LEE). 
	\item Option 1: think of $f(\theta)$ as mapping from $\R^{d_\theta}$ to $\R^{n \times p}$ (highly multi-output simulator).
	\item Option 2: think of $f(t, \theta)$ as mapping from $\{1, \dots, n\} \times \R^{d_\theta}$ to $\R^p$ (time-input, multi-output simulator).
	\item Option 3: think of $f(t, j, \theta)$ as mapping from $\{1, \dots, n\} \times \{1, \dots, p\} \times \R^{d_\theta}$ to $\R$ (time-variable input, univariate output simulator). 
	\item Let $Y \in \R^{n \times p}$ denote the field data corresponding to the computer outputs $f(\theta)$.
	\end{itemize}
\item Statistical model:
	\begin{itemize}
	\item Currently assuming independence between output variables, and across time within each output variable. For Gaussian likelihoods this means,
		\begin{align}
		p(Y|\Sigma_\epsilon, \theta) &= \prod_{t = 1}^{n} \prod_{j = 1}^{p} N(y_{tj}|f(t, j, \theta), \sigma^2_{\epsilon_j}) \label{ind_lik}
		\end{align}
	\end{itemize}
\item Emulator:
	\begin{itemize}
	\item The likelihood [\ref{ind_lik}] implies a log-likelihood
	\begin{align}
	\log p(Y|\Sigma_\epsilon, \theta) &= -\frac{np}{2} \log(2\pi \sigma^2_{\epsilon_j}) - \frac{1}{2} \sum_{j = 1}^{p} \frac{1}{\sigma^2_{\epsilon_j}} \sum_{t = 1}^{n} (y_{tj} - f(t, j, \theta))^2
	\end{align}
	\item Current model fits $p$ independent GP emulators to the functions
	\begin{align}
	T_j(\theta) &= \sum_{t = 1}^{n} (y_{tj} - f(t, j, \theta))^2, \ j = 1, \dots, p
	\end{align} 
	\item Replaces problem of fitting one $np$-output emulator with that of fitting $p$ univariate emulators. 
	\end{itemize}
\end{itemize}


\end{document}








