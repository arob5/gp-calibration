\documentclass[12pt]{article}
\RequirePackage[l2tabu, orthodox]{nag}
\usepackage[main=english]{babel}
\usepackage[rm={lining,tabular},sf={lining,tabular},tt={lining,tabular,monowidth}]{cfr-lm}
\usepackage{amsthm,amssymb,latexsym,gensymb,mathtools,mathrsfs}
\usepackage[T1]{fontenc}
\usepackage[utf8]{inputenc}
\usepackage[pdftex]{graphicx}
\usepackage{caption}
\usepackage{subcaption}
\usepackage{epstopdf,enumitem,microtype,dcolumn,booktabs,hyperref,url,fancyhdr}
\usepackage[margin=0.5in]{geometry}

% Plotting
\usepackage{pgfplots}
\usepackage{xinttools} % for the \xintFor***
\usepgfplotslibrary{fillbetween}
\pgfplotsset{compat=1.8}
\usepackage{tikz}

% Custom Commands
\newcommand*{\norm}[1]{\left\lVert#1\right\rVert}
\newcommand*{\abs}[1]{\left\lvert#1\right\rvert}
\newcommand*{\suchthat}{\,\mathrel{\big|}\,}
\newcommand{\E}{\mathbb{E}}
\newcommand{\Var}{\mathrm{Var}}
\newcommand{\R}{\mathcal{R}}
\newcommand{\N}{\mathcal{N}}
\newcommand{\Ker}{\mathrm{Ker}}
\newcommand{\Cov}{\mathrm{Cov}}
\newcommand{\Cor}{\mathrm{Corr}}
\newcommand{\Prob}{\mathbb{P}}
\DeclarePairedDelimiterX\innerp[2]{(}{)}{#1\delimsize\vert\mathopen{}#2}
\DeclareMathOperator*{\argmax}{argmax}
\DeclareMathOperator*{\argmin}{argmin}
\def\R{\mathbb{R}}
\DeclarePairedDelimiter\ceil{\lceil}{\rceil}
\DeclarePairedDelimiter\floor{\lfloor}{\rfloor}

\setlist{topsep=1ex,parsep=1ex,itemsep=0ex}
\setlist[1]{leftmargin=\parindent}
\setlist[enumerate,1]{label=\arabic*.,ref=\arabic*}
\setlist[enumerate,2]{label=(\alph*),ref=(\alph*)}

% Specifically for paper formatting 
\renewcommand{\baselinestretch}{1.2} % Spaces manuscript for easy reading

% Formatting definitions, propositions, etc. 
\newtheorem{definition}{Definition}
\newtheorem{condition}{Condition}
\newtheorem{prop}{Proposition}
\newtheorem{lemma}{Lemma}
\newtheorem{thm}{Theorem}
\newtheorem{corollary}{Corollary}
\newtheorem{notation}{Notation}

\begin{document}

\begin{center}
Scalable Bayesian Methods for Scientific Machine Learning
\end{center}

\begin{flushright}
Andrew Roberts
\end{flushright} 

The effectiveness of combining physics and simulation-based models with statistical and machine learning methods is increasingly being recognized \cite{Willcox}. Physical models
encapsulate prior scientific knowledge, while a Bayesian statistical framework allows for principled inference, prediction, and uncertainty quantification. This interplay between domain 
knowledge and statistical inference is essential for advancing scientific theory and informing high-consequence decision-making. My research goal is to develop novel methodological 
and computational approaches for model calibration, prediction, and uncertainty quantification for scientifically-informed modeling of complex systems. 



\begin{thebibliography}{20}
\bibitem{Willcox} Willcox, K.E., Ghattas, O. \& Heimbach, P. The imperative of physics-based modeling and inverse theory in computational science. Nat Comput Sci 1, 166–168 (2021). https://doi.org/10.1038/s43588-021-00040-z
\bibitem{Livingstone} Livingstone, S., Zanella, G. (2022), The Barker Proposal: Combining robustness and efficiency in gradient-based MCMC, \textit{Journal of the Royal Statistical Society} 1-28.
\bibitem{Qin} Qin, Q., Hobert, J., and Khare, K. (2019), Estimating the spectral gap of a trace-class Markov operator. \textit{Electronic Journal of Statistics} \textbf{13} 1790–1822.
\bibitem{Roberts} Roberts, G. O. and Rosenthal, G. O. (2001), Optimal scaling for various Metropolis Hastings algorithms. Statistical Science, 16(4):351 – 367, 129.
\end{thebibliography}



\end{document}



