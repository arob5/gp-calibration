\documentclass[12pt]{article}
\RequirePackage[l2tabu, orthodox]{nag}
\usepackage[main=english]{babel}
\usepackage[rm={lining,tabular},sf={lining,tabular},tt={lining,tabular,monowidth}]{cfr-lm}
\usepackage{amsthm,amssymb,latexsym,gensymb,mathtools,mathrsfs}
\usepackage[T1]{fontenc}
\usepackage[utf8]{inputenc}
\usepackage[pdftex]{graphicx}
\usepackage{caption}
\usepackage{subcaption}
\usepackage{epstopdf,enumitem,microtype,dcolumn,booktabs,hyperref,url,fancyhdr}
\usepackage[margin=0.5in]{geometry}

% Plotting
\usepackage{pgfplots}
\usepackage{xinttools} % for the \xintFor***
\usepgfplotslibrary{fillbetween}
\pgfplotsset{compat=1.8}
\usepackage{tikz}

% Custom Commands
\newcommand*{\norm}[1]{\left\lVert#1\right\rVert}
\newcommand*{\abs}[1]{\left\lvert#1\right\rvert}
\newcommand*{\suchthat}{\,\mathrel{\big|}\,}
\newcommand{\E}{\mathbb{E}}
\newcommand{\Var}{\mathrm{Var}}
\newcommand{\R}{\mathcal{R}}
\newcommand{\N}{\mathcal{N}}
\newcommand{\Ker}{\mathrm{Ker}}
\newcommand{\Cov}{\mathrm{Cov}}
\newcommand{\Cor}{\mathrm{Corr}}
\newcommand{\Prob}{\mathbb{P}}
\DeclarePairedDelimiterX\innerp[2]{(}{)}{#1\delimsize\vert\mathopen{}#2}
\DeclareMathOperator*{\argmax}{argmax}
\DeclareMathOperator*{\argmin}{argmin}
\def\R{\mathbb{R}}
\DeclarePairedDelimiter\ceil{\lceil}{\rceil}
\DeclarePairedDelimiter\floor{\lfloor}{\rfloor}

\setlist{topsep=1ex,parsep=1ex,itemsep=0ex}
\setlist[1]{leftmargin=\parindent}
\setlist[enumerate,1]{label=\arabic*.,ref=\arabic*}
\setlist[enumerate,2]{label=(\alph*),ref=(\alph*)}

% Specifically for paper formatting 
\renewcommand{\baselinestretch}{1.2} % Spaces manuscript for easy reading

% Formatting definitions, propositions, etc. 
\newtheorem{definition}{Definition}
\newtheorem{condition}{Condition}
\newtheorem{prop}{Proposition}
\newtheorem{lemma}{Lemma}
\newtheorem{thm}{Theorem}
\newtheorem{corollary}{Corollary}
\newtheorem{notation}{Notation}

\begin{document}

\begin{center}
Scalable Bayesian Methods for Scientific Machine Learning
\end{center}

\begin{flushright}
Andrew Roberts
\end{flushright} 

The effectiveness of formally combining scientific theory with statistical and machine learning methods is increasingly being recognized \cite{Willcox, Laubmeier, Wikle}. 
Physical models encapsulate prior scientific knowledge, while a rigorous statistical framework allows for principled inference, prediction, and uncertainty quantification \cite{Clark}. Computer models have become essential tools at this scientific-statistical interface, tackling problems from 
climate projections \cite{Canadell} to pandemic response \cite{Swallow}. The task of combining these models with real-world data continues to present
significant challenges, requiring tradeoffs between realism and computational and statistical feasibility. Given the ubiquity of computer models in informing
 evidence-based policy, it is essential to develop methods to handle increased complexity while simultaneously providing a full accounting of uncertainties.
Recent calls have highlighted the need to bring together state-of-the-art computational methods with modern statistical 
and machine learning models in order to address these problems \cite{Wikle, Baker}. My proposed research aims to address this need by developing novel methodological and computational approaches for combining large-scale datasets with [adjective?] computer models, with the goal of improving scientifically-informed modeling of complex systems. 

Complex processes are commonly studied using a computer model $f$, where the output is a function of some unknown, unobserved parameter $\theta$.
For example, to study the terrestrial carbon cycle, ecologists use a process model $f$ 
that predicts the net exchange of carbon between an ecosystem and the atmosphere \cite{Friedlingstein, Waring}. Running the model requires specifying values for parameters that 
may not be known, such as the soil respiration rate or seasonal leaf growth \cite{Fer}. In order to estimate (i.e. calibrate) these unknown parameters, it is necessary to obtain observations $y$ 
of the underlying physical process in order to tune $\theta$ so that $f(\theta) \approx y$. This problem is typically formulated in a Bayesian statistical setting due to its flexibility in accommodating
various sources of uncertainty \cite{Kennedy, Clark}. For example, the simple yet common data model\begin{align*}
&y \sim N(f(\theta), \sigma^2 I) \\
&(\theta, \sigma^2) \sim \pi_0,
\end{align*}
assumes the simulator output gives the mean of the true process, subject to white noise. The prior distribution $\pi_0$ encodes domain knowledge about the 
process parameter $\theta$ and statistical parameter $\sigma^2$. The problem of inferring $\theta$ given the noisy observations $y$ and prior beliefs $\pi_0$ is referred to as a \textit{Bayesian inverse problem} \cite{Stuart}. Solving such a problem is defined 
 as obtaining the posterior distribution $p(\theta|y) \propto N(y|f(\theta), \sigma^2 I)\pi_0(\theta)$, which typically involves leveraging an iterative inference procedure like Markov Chain Monte Carlo (MCMC). [maybe mention other things people do with computer models here: UQ, prediction]
 
 The use of computer models is complicated by several features commonly present in real-world applications. First, the runtime of the simulator $f$ may be prohibitively long, precluding the tens of thousands of evaluations typically required by MCMC. A standard approach to address this problem is to fit a Gaussian process 
emulator $\hat{f}(\cdot) \sim \mathcal{GP}(m(\cdot), k(\cdot, \cdot))$ to approximate $f$, which is then used in place of $f$ in the inference algorithm \cite{Kennedy, Fer, Cleary}. While this approximation is often necessary, the emulator introduces an additional source of uncertainty that is often ignored 
or accounted for in an ad hoc manner \cite{Fer}. A second complication is the difficulty in extending the statistical model to account for complex dynamics
in the underlying physical process. Simulation outputs $y$, process parameters $\theta$, and statistical parameters 
(e.g. $\sigma^2$) may exhibit variation across space and time. 

Attempts to explicitly model these dynamics lead to even more computationally and statistically challenging models. For example, terrestrial carbon models often face both of these challenges; sophisticated simulations can have a runtime of many hours \cite{Fer} and ecological parameters have been found to exhibit spatiotemporal variation \cite{Fer2}.


% Old paragraph
My proposed research program will contribute methodological 
advances for calibrating, predicting, and quantifying uncertainty using such models. Moreover, this research will support two 
overarching objectives: (i.) improving data-informed decision-making under uncertainty and (ii.) advancing scientific discovery and research. To define the 
setting of interest, let $f: \theta \mapsto \R$ denote the computer model (i.e. a simulator or process model) as a function of some unknown and unobserved
parameters $\theta \in \R^d$. As a motivating example, consider the problem of modeling the terrestrial component of the carbon cycle \cite{Friedlingstein}. 
In this case, $f$ is a process model that predicts the net exchange of carbon between an ecosystem and the atmosphere \cite{Waring} and $\theta$ may encompass 
unknown ecological parameters such as the soil respiration rate or seasonal leaf growth \cite{Fer}. The task is then to confront these models with observational data $y \in \R^n$ in 
order to calibrate (i.e. estimate) the unknown parameters, utilize the calibrated model for prediction, and carefully track sources of uncertainty throughout the analysis. These problems are typically formulated in a Bayesian setting due to its flexibility in accommodating various distinct sources of uncertainty \cite{Kennedy, Clark}. One might consider the very simple data model in which the simulator output is assumed to give the mean of the true underlying process, subject
to some white noise.   
\begin{align*}
&y \sim N(f(\theta), \sigma^2 I_n) \\
&(\theta, \sigma^2) \sim \pi_0
\end{align*}
Here, $\pi_0$ is a prior distribution that encodes previous domain knowledge about the process parameters $\theta$ and statistical parameter $\sigma^2$.
The problem of inferring the parameters $\theta$ given the noisy observations $y$ and prior beliefs $\pi_0$ is referred to as a \textit{Bayesian inverse problem} \cite{Stuart}. ``Solving'' such a problem is defined as obtaining the posterior distribution $p(\theta|y) \propto N(y|f(\theta), \sigma^2)\pi_0(\theta)$, which typically involves utilizing an iterative inference procedure like Markov Chain Monte Carlo (MCMC). 

The use of computer models is complicated by several features commonly present in real-world applications. First, the runtime of the simulator $f$ may be prohibitively long, precluding the tens of thousands of evaluations typically required by MCMC. A standard approach to address this problem is to fit a Gaussian process 
emulator $\hat{f}(\cdot) \sim \mathcal{GP}(m(\cdot), k(\cdot, \cdot))$ to approximate $f$, which is then used in place of $f$ in the inference algorithm \cite{Kennedy, Fer, Cleary}. While this approximation is often necessary, the emulator introduces an additional source of uncertainty that is often ignored 
or accounted for in an ad hoc manner \cite{Fer}. A second complication common in this setting arises due to the complexity of the underlying process being modeled by $f$. Simulation outputs $y$, process parameters $\theta$, and statistical parameters 
(e.g. $\sigma^2$) may exhibit variation across space and time. Attempts to explicitly model these dynamics lead to even more computationally and statistically challenging models. Terrestrial carbon models often face both of these challenges; sophisticated simulations can have a runtime of many hours \cite{Fer} and ecological parameters have been found to exhibit spatiotemporal variation \cite{Fer2} [another citation?]. All of these problems share a common theme: the difficulty in striking a balance between realism and computational and statistical feasibility in the analysis of complex systems. Often, one of these characteristics must be sacrificed in favor of the other [cite].
Given the ubiquity of computer models in informing evidence-based policy, it is essential to develop methods to handle more complex models while simultaneously providing a full accounting of uncertainties. 
There have been recent calls highlighting the need for research that brings together state-of-the-art computational methods with modern statistical and machine learning models in order to address these problems \cite{Wikle, Baker}. My proposed research aims to address this need. 
[somewhere should state why it is important to generalize Gaussian error model] [Include stochastic simulators as third challenge?]

\noindent
\textbf{Research Plan.} Uncertainty quantification is a major theme underlying each stage of this proposal, and is made more difficult
as increasingly complex statistical models require additional approximation techniques. Therefore, I will begin by addressing fundamental issues of uncertainty in the relatively simpler static, deterministic setting. Specifically, I will develop methods for incorporating emulator uncertainty into the statistical inference algorithm in order to achieve more ``honest'' uncertainty quantification. With this foundation established, I will then be in a position to integrate spatiotemporal dynamics into the statistical model, as well as consider the issue of stochastic simulators. These modeling steps will pose increasingly difficult computational challenges, which I intend to address via an interdisciplinary approach, drawing from spatiotemporal statistics, Bayesian inverse problem theory, and machine learning to develop novel dimensionality reduction and approximation methods. Throughout this process, I will test these methods on terrestrial carbon models commonly used in ecological forecasting \cite{Dietze, Canadell}. 
 
 \textbf{1. Propagating Emulator Uncertainty.} I will begin by exploring the ramifications of approximating the computationally expensive model $f$ with the  cheaper Gaussian process emulator $\hat{f}$. Naive emulation would simply use the Gaussian process mean $m(\theta)$ in place of the process model evaluation $f(\theta)$. However, this fails to account
 for the uncertainty introduced by the emulator approximation, resulting in overly confident posterior estimates. This begs the question: what is the ``correct'' way to propagate the emulator uncertainty through the inference procedure? In the current literature, this question is either largely ignored or addressed in an ad hoc manner \cite{Cleary, Fer}. Gaining a rigorous understanding of methods for propagating emulator uncertainty is essential to ensure that model predictions are not inaccurately presented as overly confident. Various papers have offered bespoke solutions to this problem \cite{Cleary, Fer}, but no comprehensive
methodological study has been conducted. Moreover, these existing methods consider only the Gaussian error model and treat emulator uncertainty in an independent fashion, failing to take advantage of the correlation in Gaussian process predictions. I will address these gaps in the literature by (i.) conducting a rigorous comparison of methods for propagating emulator uncertainty under the standard Gaussian noise model, (ii.) developing novel techniques which take advantage of the Gaussian process predictive correlation, and (iii.) generalize these methods to auto-regressive and Laplace error models, both of which 
are common in ecological applications \cite{Fer}. 

 \textbf{2. Spatio-temporal Dynamics}
With methodologically sound and scalable algorithms in hand for the baseline static framework, I will then be in a position to begin investigating spatiotemporal generalizations of the statistical model. Given the spatiotemporal nature of most physical processes, this is an important extension that is often ignored, presumably due to computational challenges (\cite{Fer2}, [another citation]). Parameters $\theta$ may naturally be thought to vary over space
and time; even if they are believed to be static, a dynamic model allows for spatially and temporally varying uncertainty estimates, which can help identify 
missing processes in physical models, as well as provide a greater understanding of how models generalize beyond observed data \cite{Fer2, Dietze}. To address these issues I will extend the data model as 
  \[y(s, t) = f(\theta(s, t)) + \epsilon(s, t)\]
 where observed data $y$, process parameters $\theta$, and error $\epsilon$ are all modeled as spatiotemporal processes. Ad hoc solutions have been developed to deal with computer models with spatially or temporally indexed outputs \cite{Sun}, but these approaches rarely incorporate a
 explicit spatiotemporal model for the process parameters $\theta$ [cite]. Recent studies in the the terrestrial carbon modeling literature have 
 taken a step in this direction by allowing process parameters $\theta$ to exhibit Gaussian variability across different ecological data collection
sites \cite{Fer2}. This hierarchical structure was informative but fails to incorporate explicit spatial information such as a concept of ``nearness'' between the sites. Researchers in both the spatiotemporal statistics and computer modeling communities
have noted the potential for integrating methods in both fields using modern advances in machine learning and computation \cite{Wikle, Baker}. My proposed research will bridge this divide by (i.) developing novel techniques to integrate modern spatiotemporal dynamical models \cite{Wikle, Hefley}
within the computer modeling emulation framework, (ii.) develop dimensionality reduction and approximation methods that draw from Bayesian inverse problem 
theory \cite{Kugler} and Bayesian filtering \cite{Sarkka}, and (iii.) extend my previous uncertainty quantification methodology to this more complex setting. 
 

\begin{thebibliography}{20}
\bibitem{Willcox} Willcox, K.E., Ghattas, O. \& Heimbach, P. The imperative of physics-based modeling and inverse theory in computational science. Nat Comput Sci 1, 166–168 (2021). https://doi.org/10.1038/s43588-021-00040-z
\bibitem{Cayelan} Cayelan C. Carey, Whitney M. Woelmer, Mary E. Lofton, Renato J. Figueiredo, Bethany J. Bookout, Rachel S. Corrigan, Vahid Daneshmand, Alexandria G. Hounshell, Dexter W. Howard, Abigail S. L. Lewis et al (2022) Advancing lake and reservoir water quality management with near-term, iterative ecological forecasting, Inland Waters, 12:1, 107-120, DOI: 10.1080/20442041.2020.1816421
\bibitem{Cleary} Emmet Cleary, Alfredo Garbuno-Inigo, Shiwei Lan, Tapio Schneider, Andrew M. Stuart, “Calibrate, emulate, sample”, Journal of Computational Physics, Volume 424, 2021, 109716, ISSN 0021-9991, https://doi.org/10.1016/j.jcp.2020.109716.
\bibitem{Dietze} Dietze et al, “Iterative near-term ecological forecasting: Needs, opportunities, and challenges”, Proceedings of the National Academy of Sciences, 115, 7, 1424-1432, 2018.
\bibitem{Fer} Fer, I., Kelly, R., Moorcroft, P. R., Richardson, A. D., Cowdery, E. M., and Dietze, M. C.: Linking big models to big data: efficient ecosystem model calibration through Bayesian model emulation, Biogeosciences, 15, 5801–5830, https://doi.org/10.5194/bg-15-5801-2018, 2018.
\bibitem{Fer2} Istem Fer, Alexey Shiklomanov, Kimberly A. Novick, Christopher M. Gough, M. Altaf Arain, Jiquan Chen, Bailey Murphy, Ankur R. Desai, Michael C. Dietze: Capturing site-to-site variability through Hierarchical Bayesian calibration of a process-based dynamic vegetation model, bioRxiv 2021.04.28.441243; doi: https://doi.org/10.1101/2021.04.28.441243. 
\bibitem{Arab} Arab, Ali \& Hooten, Mevin \& Wikle, Christopher. (2007). Hierarchical Spatial Models. Encyclopedia of geographical information science.
\bibitem{Friedlingstein} Friedlingstein, et al: Global Carbon Budget 2021, Earth Syst. Sci. Data, 14, 1917–2005, https://doi.org/10.5194/essd-14-1917-2022, 2022.
\bibitem{Hefley} Trevor J. Hefley, Mevin B. Hooten, Ephraim M. Hanks, Robin E. Russell, Daniel P. Walsh, Dynamic spatio-temporal models for spatial data, Spatial Statistics.
\bibitem{Kugler} Benoit Kugler, Florence Forbes, Sylvain Douté. Fast Bayesian Inversion for high dimensional inverse problems. Statistics and Computing, Springer Verlag (Germany), In press. ffhal-02908364v3f
\bibitem{Sarkka} Särkkä, S. (2013). Bayesian Filtering and Smoothing (Institute of Mathematical Statistics Textbooks). Cambridge: Cambridge University Press. doi:10.1017/CBO9781139344203
\bibitem{Waring} Richard H. Waring, Steven W. Running, CHAPTER 10 - Advances in Eddy-Flux Analyses, Remote Sensing, and Evidence of Climate Change, Forest Ecosystems (Third Edition), Academic Press, 2007, Pages 317-344, ISBN 9780123706058, https://doi.org/10.1016/B978-012370605-8.50017-7.
\bibitem{Clark} Clark, James S. 2005. Why environmental scientists are becoming Bayesians. Ecology Letters, Vol. 8: 2-14
\bibitem{Laubmeier} Laubmeier et al. Ecological Dynamics: Integrating Empirical, Statistical, and Analytical Methods. Trends Ecol Evol. 2020 Dec;35(12):1090-1099. doi: 10.1016/j.tree.2020.08.006. Epub 2020 Sep 12. PMID: 32933777.
\bibitem{Wikle} Wikle, C.K. (2015), Modern perspectives on statistics for spatio-temporal data. WIREs Comput Stat, 7: 86-98. https://doi.org/10.1002/wics.1341
\bibitem{Baker} Baker et al. "Analyzing Stochastic Computer Models: A Review with Opportunities." Statist. Sci. 37 (1) 64 - 89, February 2022. https://doi.org/10.1214/21-STS822
\bibitem{Kennedy} Kennedy, M.C. and O'Hagan, A. (2001), Bayesian calibration of computer models. Journal of the Royal Statistical Society: Series B (Statistical Methodology), 63: 425-464. https://doi.org/10.1111/1467-9868.00294
\bibitem{Canadell} Canadell et al, 2021: Global Carbon and
other Biogeochemical Cycles and Feedbacks. In Climate Change 2021: The Physical Science Basis. Contribution of
Working Group I to the Sixth Assessment Report of the Intergovernmental Panel on Climate Change. Cambridge University Press,
Cambridge, United Kingdom and New York, NY, USA, pp. 673–816, doi:10.1017/9781009157896.007.
\bibitem{Swallow} Swallow et al, Challenges in estimation, uncertainty quantification and elicitation for pandemic modelling, Epidemics, Volume 38, 2022, 100547, ISSN 1755-4365, https://doi.org/10.1016/j.epidem.2022.100547.
\bibitem{Stuart} A. M. Stuart (2010). Inverse problems: A Bayesian perspective. Acta Numerica, 19, pp 451-559 doi:10.1017/
S0962492910000061.
\bibitem{Sun} Sun et al, Synthesizing simulation and field data of solar irradiance. Stat Anal Data Min: The ASA Data Sci Journal. 2019; 12: 311– 324. https://doi.org/10.1002/sam.11414
\end{thebibliography}



\end{document}



